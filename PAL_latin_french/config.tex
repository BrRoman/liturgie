\documentclass[twoside]{article}

% Géométrie de la page :
\usepackage[twoside]{geometry}
\geometry{paperwidth=14.85cm, paperheight=21cm, inner=0.7cm, outer=1.2cm, tmargin=1cm, bmargin=1.3cm, includehead}

% Choix d'une police principale :
\usepackage{luatextra}
\setmainfont{Arno Pro}

% Redéfinition de la taille des polices :
%\renewcommand{\small}{\fontsize{11}{12}\selectfont}

% Césures etc. :
\usepackage[latin, french]{babel}
\selectlanguage{french}

% Mise en forme des titres de sections :
\usepackage[explicit]{titlesec}
\titleformat{\section}{}{}{0cm}{
    \fontsize{11}{12}\selectfont
    \begin{center}
    \makebox[5.9cm][c]{\centering\textsc{\textbf{#1}}}
    \end{center}
    \vspace{0.2cm}
}
\titlespacing{\section}{0cm}{0cm}{-.5cm}

% Textes en parallèle :
%\usepackage{parallel}
\usepackage{paracol}

% Entêtes et pieds de pages :
\usepackage{fancyhdr}
\pagestyle{fancy}
\fancyhead{}
\fancyhead[CE]{\fontsize{11}{12}\selectfont\textsc{Missel romain 1962 - Supplément}}
\fancyhead[CO]{\fontsize{11}{12}\selectfont\textsc{\rightmark}}
\fancyfoot[CE,CO]{\fontsize{11}{12}\selectfont\thepage}
\renewcommand{\sectionmark}[1]{\markright{#1}}
\renewcommand{\headrulewidth}{0.3pt}
\renewcommand{\footrulewidth}{0pt}
\renewcommand{\headrule}{\vbox to 6pt{\hbox to\headwidth{\hrulefill}\vss}}
\setlength{\parindent}{0cm}
\setlength{\headsep}{0.2cm} % Distance entre le header et le corps du texte.
\setlength{\footskip}{0.6cm} % Distance entre le footer et le corps du texte.
% Redéfinition du style {plain} (seulement pied de page) :
\fancypagestyle{plain}{
\fancyhf{}% Clear all.
\fancyfoot[CE,CO]{\fontsize{11}{12}\selectfont\thepage}
\renewcommand{\headrulewidth}{0pt}
\renewcommand{\headrule}{}
\setlength{\headsep}{0cm}
}
% Redéfintion du style {headings} (seulement entête) :
\fancypagestyle{headings}{
\fancyhf{}% Clear all.
\fancyhead[CE]{\fontsize{11}{12}\selectfont\textsc{Benedictiones}}
\fancyhead[CO]{\fontsize{11}{12}\selectfont\textsc{\rightmark}}
\setlength{\footskip}{0cm}
}

% Pour forcer l'usage des césures (cf. mail Thierry Masson) :
\pretolerance = -1
\tolerance = 7000

% Pour augmenter l'approche des caractères :
\usepackage{soul}

% Gregorio :
\usepackage[autocompile]{gregoriotex}
%\grechangedim{commentaryraise}{.4cm}{scalable}
%\grechangestyle{modeline}{\small\scshape}

% Table des matières :
\usepackage{tocloft}
% Pour supprimer les numéros de section dans la table des matières,
% on les met dans des boîtes tempo qu'on n'utilise jamais.
% cf. https://tex.stackexchange.com/questions/71123/remove-section-number-toc-entries-with-tocloft/71136#71136
\makeatletter
\renewcommand{\cftsecpresnum}{\begin{lrbox}{\@tempboxa}}
\renewcommand{\cftsecaftersnum}{\end{lrbox}}
\makeatother

% Lettrines :
\usepackage{lettrine}

% Various (fonts) :
\usepackage{fontspec}
\usepackage{calc}
\usepackage{pifont}
\frenchbsetup{ThinColonSpace=true}
\newfontfamily{\GregPlantin}[BoldFont = GregPlantin Bold,ItalicFont = GregPlantin Italic,BoldItalicFont = GregPlantin Bolditalic]{GregPlantin Regular}
\newfontfamily{\PlantinStd}{Plantin Std}
\newfontfamily{\GaramondPremierPro}[Numbers=OldStyle]{Garamond Premier Pro}
%\newfontfamily{\GaramondPremierProCaption}[Numbers=OldStyle]{GaramondPremrPro-Capt}
%\newfontfamily{\GaramondPremierProMediumCaption}[Numbers=OldStyle]{GaramondPremrPro-MedCapt}
%\newfontfamily{\GaramondPremierProItCaption}[Numbers=OldStyle]{GaramondPremrPro-ItCapt}
%\newfontfamily{\GaramondPremierProIt}{GaramondPremrPro-It}
%\newfontfamily{\GaramondPremierProMediumIt}{GaramondPremrPro-MedIt}
\newfontfamily{\FlavGaramond}{FlavGaramond}
\definecolor{rougeliturgique}{cmyk}{0.15,1,1,0}
\renewcommand{\Rbar}{\textbf{\color{rougeliturgique}\GregPlantin\symbol{164}}}
\renewcommand{\Vbar}{\textbf{\color{rougeliturgique}\GregPlantin\symbol{8730}}}
\renewcommand{\GreDagger}{\textrm{\color{rougeliturgique}\FlavGaramond \symbol{8224}}}
\catcode`\®=\active
\def®{\Rbar}
\catcode`\√=\active
\def√{\Vbar}
\catcode`\©=\active
\def©{\hspace{-1.2ex}}
\catcode`\†=\active
\def†{\GreDagger}
\makeatletter
\def\accentaigucaractere{\makebox[0pt][c]{´}}
\newcommand\accentaigu[1]{\setlength{\@tempdima}{\widthof{#1}}\hbox{#1\kern-0.5\@tempdima\accentaigucaractere\kern0.5\@tempdima}}
\makeatother
\def\espacefine{\hspace{0.035cm}}
\catcode`\ã=\active% Tilde-a
\defã{\accentaigu{æ}}
\catcode`\õ=\active% Tilde-o
\defõ{\accentaigu{œ}}
\catcode`\Ï=\active% Alt-j
\defÏ{\accentaigu{y}}
\catcode`\™=\active% Alt-Maj-8
\def™{\textit{\color{rougeliturgique}T.P.}}
\catcode`\Ø=\active% Alt-Maj-$ (dollar)
\defØ{\textit{\color{rougeliturgique}Ps.}}
%\catcode`\™=\active% Alt-Maj-t
%\defı{\textup{\color{rougeliturgique}N.}}
\catcode`\€\active% Alt-* (sinon les * des \vspace* sont actives)
\def€{\GreStar}
\def\GreStar{
{\color{rougeliturgique}\tiny\raisebox{1.5ex}{\ding{72}}}
  \relax
}
%\grechangestyle{initial}{\fontsize{28}{36}\color{rougeliturgique}\PlantinStd }
%\newbox\scorebox
%\gresetheadercapture{commentary}{grecommentary}{string}
%\grechangestaffsize{12}
%\grechangedim{afterinitialshift}{2.2mm}{fixed}
%\grechangedim{beforeinitialshift}{3mm}{fixed}
%\let\grevanillacommentary\grecommentary
%\def\grecommentary#1{\grevanillacommentary{#1\kern 0.3mm}}
%\gresetbarspacing{new}
%\grechangedim{bar@maior@standalone@notext}{0.3 cm}{scalable}
%\grechangedim{spacebeforeeolcustos}{0.3 cm}{scalable}
%\grechangedim{baselineskip}{40pt plus 5 pt minus 5 pt}{scalable}


%%%%%%%%%%%%%%%%%
% Symboles spéciaux :
%%%%%%%%%%%%%%%%%

% Antienne :
%\catcode`\ø=\active
%\defø{{\fontspec{FlavGaramond} \symbol{8721}}}
% Verset :
\catcode`\ß=\active
\defß{{\fontspec{FlavGaramond} \symbol{8730}}}
% Croix de Malte:
\catcode`\+=\active
\def+{{\color{rougeliturgique}{\fontspec{Menlo}{\fontsize{18}{24}\selectfont \symbol{10016}}}}}


%%%%%%%%%%%%%%%%%
% Hyphenations :
%%%%%%%%%%%%%%%%%

\hyphenation{al-le-lu-ia}


%%%%%%%%%%%%%%%%%%%%%%%%%%%%%%%
% Commandes et environnements :
%%%%%%%%%%%%%%%%%%%%%%%%%%%%%%%

% Espace fine :
\DeclareRobustCommand{\mynobreakthinspace}{%
\leavevmode\nobreak\hspace{0.08em}}
\def~{\mynobreakthinspace{}}

% Style pour les références des partitions :
%\grechangestyle{commentary}{\color{rougeliturgique}\itshape\fontsize{9}{8}\selectfont}

%% Environnement Boîte (espace avant, contenu) :
%\newenvironment{ParBox}[2]{
%    \setlength{\parindent}{0cm}
%    \begin{center}
%    \parbox[t]{14.85cm}{\vspace{#1} #2}
%    \end{center}
%    \par
%}

% Style de paragraphe TitreA :
\newenvironment{TitreA}[1]{
    \setlength{\parindent}{0cm}
    \setlength{\leftskip}{0cm}
    \fontsize{16}{18}\selectfont
    \setlength{\parskip}{-0.3cm}
    \begin{center}
    {\color{rougeliturgique}\MakeUppercase{#1}}
    \end{center}
}

% Style de paragraphe TitreB :
\newenvironment{TitreB}[1]{
    \setlength{\parindent}{0cm}
    \setlength{\leftskip}{0cm}
    \setlength{\parskip}{-0.3cm}
    \fontsize{10}{12}\selectfont
    \begin{center}
    \textsc{#1}
    \end{center}
}

% Style de paragraphe TitreC :
\newenvironment{TitreC}[1]{
    \setlength{\parindent}{0cm}
    \setlength{\leftskip}{0cm}
    \setlength{\parskip}{0cm}
    \fontsize{10}{10}\selectfont
    \begin{center}
    {\color{rougeliturgique}\textsc{#1}}
    \end{center}
}

% Style de paragraphe Normal :
\newenvironment{Normal}[1]{
    \setlength{\parindent}{0cm}
    \setlength{\leftskip}{0cm}
    \setlength{\parskip}{0cm}
    \selectlanguage{latin}
    \fontsize{14}{16}\selectfont#1\par
    \vspace{0.1cm}
}

% Style de paragraphe Rubrique :
%\newenvironment{Rubrique}[1]{
%    \setlength{\parindent}{0cm}
%    \setlength{\leftskip}{0cm}
%    \setlength{\parskip}{0cm}
%    \fontsize{9}{11}\selectfont
%    {\color{rougeliturgique}\textit{#1}}
%    \vspace{0.1cm}
%}

\newenvironment{Date}[1]{
    \setlength{\parindent}{0cm}
    \setlength{\leftskip}{0cm}
    \setlength{\parskip}{0cm}
    \selectlanguage{latin}
    \fontsize{14}{16}\selectfont
    \begin{center}
        \textsc{#1}
    \end{center}
    \vspace{-.5cm}
}

\newenvironment{Name}[1]{
    \setlength{\parindent}{0cm}
    \setlength{\leftskip}{0cm}
    \setlength{\parskip}{0cm}
    \selectlanguage{latin}
    \fontsize{16}{18}\selectfont
    \begin{center}
        \MakeUppercase{#1}
    \end{center}
    \vspace{-.5cm}
}

\newenvironment{Rang}[1]{
    \setlength{\parindent}{0cm}
    \setlength{\leftskip}{0cm}
    \setlength{\parskip}{0cm}
    \selectlanguage{latin}
    \fontsize{14}{16}\selectfont
    \begin{center}
        {\color{rougeliturgique}\textsc{#1}}
    \end{center}
    \vspace{-.7cm}
}

\newenvironment{Latin}[1]{
    \setlength{\parindent}{0cm}
    \setlength{\leftskip}{0cm}
    \setlength{\parskip}{0cm}
    \selectlanguage{latin}
    \fontsize{13}{14}\selectfont
    #1
}

\newenvironment{French}[1]{
    \setlength{\parindent}{0cm}
    \setlength{\leftskip}{0cm}
    \setlength{\parskip}{0cm}
    \selectlanguage{french}
    \fontsize{12}{14}\selectfont
    #1
}

% Chapeau :
\newcommand{\Hat}[1]{
    \begin{center}
        {\fontsize{12}{14}\selectfont\color{rougeliturgique}#1}
    \end{center}
    \vspace{-.3cm}
}

% Références :
\newcommand{\Ref}[1]{
    {\fontsize{12}{14}\selectfont\textit{\color{rougeliturgique}#1}}
}

% Chapeau rouge avec référence :
\newcommand{\HatWithRefRed}[2]{
    \vspace{.3cm}
    {\fontsize{12}{14}\selectfont\color{rougeliturgique}#1}\hfill\Ref{#2}
}

% Chapeau noir avec référence latin :
\newcommand{\HatWithRefBlackLatin}[2]{
    \vspace{.3cm}
    {\fontsize{13}{14}\selectfont#1}\par\hfill\Ref{#2}
    \vspace{.5cm}
}

% Chapeau noir avec référence français :
\newcommand{\HatWithRefBlackFrench}[2]{
    \vspace{.3cm}
    {\fontsize{12}{14}\selectfont#1}\par\hfill\Ref{#2}
    \vspace{.5cm}
}

% Lettrines rouges [Lettrine, reste du mot, coeff. de dilatation au-dessus de la 1ère ligne, coeff. d'élévation au-dessus de la 1ère ligne] :
\newcommand\DropCapRed[4]{\vspace{-\baselineskip}\lettrine[realheight, nindent=0.5em, slope=-.5em, loversize=#3, lraise=#4]{{\color{rougeliturgique}#1}}{#2}}

% Lettrines noires [Lettrine, reste du mot, coeff. de dilatation au-dessus de la 1ère ligne, coeff. d'élévation au-dessus de la 1ère ligne] :
\newcommand\DropCapBlack[4]{\vspace{-\baselineskip}\lettrine[realheight, nindent=0.5em, slope=-.5em, loversize=#3, lraise=#4]{#1}{#2}}

% Textes en rouge :
\newcommand\Red[1]{{\color{rougeliturgique}#1}}

% Chapeau Évangile latin (Évangéliste, référence) :
\newcommand\EvgLatin[2]{\HatWithRefBlackLatin{+\hspace{0.3cm}\Red{S}equéntia sancti Evangélii secúndum #1.}{#2}}

% Chapeau Évangile français (Évangéliste, référence) :
\newcommand\EvgFrench[2]{\HatWithRefBlackFrench{\Red{L}ecture du saint Évangile selon saint #1.}{#2}}

% Ligne de séparation :
\newcommand\Ligne{
\vspace{0.2cm}
\begin{center}
\greseparator{2}{10}
\end{center}
}

