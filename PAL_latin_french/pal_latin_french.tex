\documentclass[twoside]{article}

% Géométrie de la page :
\usepackage{geometry}
\geometry{paperwidth=14.85cm, paperheight=21cm, inner=1cm, outer=1.2cm, tmargin=1cm, bmargin=1.2cm, includehead}

% Choix d'une police principale :
\usepackage{luatextra}
\setmainfont{Arno Pro}

% Redéfinition de la taille des polices :
%\renewcommand{\small}{\fontsize{8}{10}\selectfont}

% Césures etc. :
\usepackage[latin, french]{babel}
\selectlanguage{french}

% Mise en forme des titres de sections :
\usepackage[explicit]{titlesec}
\titleformat{\section}{}{}{0cm}{
    \fontsize{11}{12}\selectfont
    \begin{center}
    \makebox[5.9cm][c]{\centering\textsc{\textbf{#1}}}
    \end{center}
    \vspace{0.2cm}
}
\titlespacing{\section}{0cm}{0cm}{-.5cm}

% Textes en parallèle :
%\usepackage{parallel}
\usepackage{paracol}

% Entêtes et pieds de pages :
\usepackage{fancyhdr}
\pagestyle{fancy}
\fancyhead{}
\fancyhead[CE]{\fontsize{8}{10}\selectfont\textsc{Benedictiones}}
\fancyhead[CO]{\fontsize{8}{10}\selectfont\textsc{\rightmark}}
\fancyfoot[CE,CO]{\fontsize{8}{10}\selectfont\thepage}
\renewcommand{\sectionmark}[1]{\markright{#1}}
\renewcommand{\headrulewidth}{0.3pt}
\renewcommand{\footrulewidth}{0pt}
\renewcommand{\headrule}{\vbox to 6pt{\hbox to\headwidth{\hrulefill}\vss}}
\setlength{\parindent}{0cm}
\setlength{\headsep}{0.2cm} % Distance entre le header et le corps du texte.
\setlength{\footskip}{0.6cm} % Distance entre le footer et le corps du texte.
% Redéfinition du style {plain} (seulement pied de page) :
\fancypagestyle{plain}{
\fancyhf{}% Clear all.
\fancyfoot[CE,CO]{\fontsize{8}{10}\selectfont\thepage}
\renewcommand{\headrulewidth}{0pt}
\renewcommand{\headrule}{}
\setlength{\headsep}{0cm}
}
% Redéfintion du style {headings} (seulement entête) :
\fancypagestyle{headings}{
\fancyhf{}% Clear all.
\fancyhead[CE]{\fontsize{8}{10}\selectfont\textsc{Benedictiones}}
\fancyhead[CO]{\fontsize{8}{10}\selectfont\textsc{\rightmark}}
\setlength{\footskip}{0cm}
}

% Pour forcer l'usage des césures (cf. mail Thierry Masson) :
\pretolerance = -1
\tolerance = 2000

% Pour augmenter l'approche des caractères :
\usepackage{soul}

% Gregorio :
\usepackage[autocompile]{gregoriotex}
%\grechangedim{commentaryraise}{.4cm}{scalable}
%\grechangestyle{modeline}{\small\scshape}

% Table des matières :
\usepackage{tocloft}
% Pour supprimer les numéros de section dans la table des matières,
% on les met dans des boîtes tempo qu'on n'utilise jamais.
% cf. https://tex.stackexchange.com/questions/71123/remove-section-number-toc-entries-with-tocloft/71136#71136
\makeatletter
\renewcommand{\cftsecpresnum}{\begin{lrbox}{\@tempboxa}}
\renewcommand{\cftsecaftersnum}{\end{lrbox}}
\makeatother


% Various (fonts) :
\usepackage{fontspec}
\usepackage{calc}
\usepackage{pifont}
\frenchbsetup{ThinColonSpace=true}
\newfontfamily{\GregPlantin}[BoldFont = GregPlantin Bold,ItalicFont = GregPlantin Italic,BoldItalicFont = GregPlantin Bolditalic]{GregPlantin Regular}
\newfontfamily{\PlantinStd}{Plantin Std}
\newfontfamily{\GaramondPremierPro}[Numbers=OldStyle]{Garamond Premier Pro}
%\newfontfamily{\GaramondPremierProCaption}[Numbers=OldStyle]{GaramondPremrPro-Capt}
%\newfontfamily{\GaramondPremierProMediumCaption}[Numbers=OldStyle]{GaramondPremrPro-MedCapt}
%\newfontfamily{\GaramondPremierProItCaption}[Numbers=OldStyle]{GaramondPremrPro-ItCapt}
%\newfontfamily{\GaramondPremierProIt}{GaramondPremrPro-It}
%\newfontfamily{\GaramondPremierProMediumIt}{GaramondPremrPro-MedIt}
\newfontfamily{\FlavGaramond}{FlavGaramond}
\definecolor{rougeliturgique}{cmyk}{0.15,1,1,0}
\renewcommand{\Rbar}{\textbf{\color{rougeliturgique}\GregPlantin\symbol{164}}}
\renewcommand{\Vbar}{\textbf{\color{rougeliturgique}\GregPlantin\symbol{8730}}}
\renewcommand{\GreDagger}{\textrm{\color{rougeliturgique}\FlavGaramond \symbol{8224}}}
\catcode`\®=\active
\def®{\Rbar}
\catcode`\√=\active
\def√{\Vbar}
\catcode`\©=\active
\def©{\hspace{-1.2ex}}
\catcode`\†=\active
\def†{\GreDagger}
\makeatletter
\def\accentaigucaractere{\makebox[0pt][c]{´}}
\newcommand\accentaigu[1]{\setlength{\@tempdima}{\widthof{#1}}\hbox{#1\kern-0.5\@tempdima\accentaigucaractere\kern0.5\@tempdima}}
\makeatother
\def\espacefine{\hspace{0.035cm}}
\catcode`\ã=\active% Tilde-a
\defã{\accentaigu{æ}}
\catcode`\õ=\active% Tilde-o
\defõ{\accentaigu{œ}}
\catcode`\Ï=\active% Alt-j
\defÏ{\accentaigu{y}}
\catcode`\™=\active% Alt-Maj-t
\def™{\textit{\color{rougeliturgique}T.P.}}
\catcode`\∏=\active% Alt-Maj-p
\def∏{\textit{\color{rougeliturgique}Ps.}}
\catcode`\ı=\active% Alt-Maj-n
\defı{\textup{\color{rougeliturgique}N.}}
\catcode`\€\active% Alt-* (sinon les * des \vspace* sont actives)
\def€{\GreStar}
\def\GreStar{
{\color{rougeliturgique}\tiny\raisebox{1.5ex}{\ding{72}}}
  \relax
}
%\grechangestyle{initial}{\fontsize{28}{36}\color{rougeliturgique}\PlantinStd }
%\newbox\scorebox
%\gresetheadercapture{commentary}{grecommentary}{string}
%\grechangestaffsize{12}
%\grechangedim{afterinitialshift}{2.2mm}{fixed}
%\grechangedim{beforeinitialshift}{3mm}{fixed}
%\let\grevanillacommentary\grecommentary
%\def\grecommentary#1{\grevanillacommentary{#1\kern 0.3mm}}
%\gresetbarspacing{new}
%\grechangedim{bar@maior@standalone@notext}{0.3 cm}{scalable}
%\grechangedim{spacebeforeeolcustos}{0.3 cm}{scalable}
%\grechangedim{baselineskip}{40pt plus 5 pt minus 5 pt}{scalable}

%%%%%%%%%%%%%%%%%%%%%%%%%%%%%%%
% Commandes et environnements :
%%%%%%%%%%%%%%%%%%%%%%%%%%%%%%%

% Espace fine :
\DeclareRobustCommand{\mynobreakthinspace}{%
\leavevmode\nobreak\hspace{0.08em}}
\def~{\mynobreakthinspace{}}

% Style pour les références des partitions :
%\grechangestyle{commentary}{\color{rougeliturgique}\itshape\fontsize{9}{8}\selectfont}

%% Environnement Boîte (espace avant, contenu) :
%\newenvironment{ParBox}[2]{
%    \setlength{\parindent}{0cm}
%    \begin{center}
%    \parbox[t]{14.85cm}{\vspace{#1} #2}
%    \end{center}
%    \par
%}

% Style de paragraphe TitreA :
\newenvironment{TitreA}[1]{
    \setlength{\parindent}{0cm}
    \setlength{\leftskip}{0cm}
    \fontsize{16}{18}\selectfont
    \setlength{\parskip}{-0.3cm}
    \begin{center}
    {\color{rougeliturgique}\MakeUppercase{#1}}
    \end{center}
}

% Style de paragraphe TitreB :
\newenvironment{TitreB}[1]{
    \setlength{\parindent}{0cm}
    \setlength{\leftskip}{0cm}
    \setlength{\parskip}{-0.3cm}
    \fontsize{10}{12}\selectfont
    \begin{center}
    \textsc{#1}
    \end{center}
}

% Style de paragraphe TitreC :
\newenvironment{TitreC}[1]{
    \setlength{\parindent}{0cm}
    \setlength{\leftskip}{0cm}
    \setlength{\parskip}{0cm}
    \fontsize{10}{10}\selectfont
    \begin{center}
    {\color{rougeliturgique}\textsc{#1}}
    \end{center}
}

% Style de paragraphe Normal :
\newenvironment{Normal}[1]{
    \setlength{\parindent}{0cm}
    \setlength{\leftskip}{0cm}
    \setlength{\parskip}{0cm}
    \selectlanguage{latin}
    \fontsize{9}{10}\selectfont#1\par
    \vspace{0.1cm}
}

% Style de paragraphe Rubrique :
\newenvironment{Rubrique}[1]{
    \setlength{\parindent}{0cm}
    \setlength{\leftskip}{0cm}
    \setlength{\parskip}{0cm}
    \fontsize{9}{11}\selectfont
    {\color{rougeliturgique}\textit{#1}}
    \vspace{0.1cm}
}

\newenvironment{Date}[1]{
    \setlength{\parindent}{0cm}
    \setlength{\leftskip}{0cm}
    \setlength{\parskip}{0cm}
    \selectlanguage{latin}
    \fontsize{9}{10}\selectfont
    \begin{center}
        {\color{rougeliturgique}\textsc{#1}}
    \end{center}
}

\newenvironment{Name}[1]{
    \setlength{\parindent}{0cm}
    \setlength{\leftskip}{0cm}
    \setlength{\parskip}{0cm}
    \selectlanguage{latin}
    \fontsize{9}{10}\selectfont
    \begin{center}
        #1
    \end{center}
}

% Rubriques à l'intérieur d'un texte :
%\newcommand{\RubriqueInside}[1]{
%{\fontsize{9}{11}\selectfont\textit{\color{rougeliturgique}#1}}
%}

% Ligne de séparation :
\newcommand\Ligne{
\vspace{0.2cm}
\begin{center}
\greseparator{2}{10}
\end{center}
}


%%%%%%%%%%%%%%%%%
% Symboles spéciaux :
%%%%%%%%%%%%%%%%%

% Antienne :
%\catcode`\ø=\active
%\defø{{\fontspec{FlavGaramond} \symbol{8721}}}
% Verset :
\catcode`\ß=\active
\defß{{\fontspec{FlavGaramond} \symbol{8730}}}
% Croix de Malte:
\catcode`\+=\active
\def+{{\fontspec{Menlo} \symbol{10016}}}

%

%%%%%%%%%%%%%%%%%
% Hyphenations :
%%%%%%%%%%%%%%%%%

\hyphenation{al-le-lu-ia}




\begin{document}

\setlength{\columnseprule}{0.5pt}
\colseprulecolor{rougeliturgique}

\begin{paracol}{2}

\vspace*{.3cm}
\sectionmark{7 décembre - saint Ambroise}
\Date{Die 7 decembris}
\switchcolumn
\vspace*{.3cm}
\Date{Le 7 décembre}
\switchcolumn*

\Name{Sancti Ambrosii}
\switchcolumn
\Name{Saint Ambroise}
\switchcolumn*

\Rang{Ep., Conf.}
\switchcolumn
\Rang{Év., Conf.}
\switchcolumn*

\Rang{et Eccl. Doctoris}
\switchcolumn
\Rang{et Docteur de l’Église}
\switchcolumn*

\vspace{.7cm}
\HatWithRefRed{Ant. ad Introitum.}{Sap 7, 7}
\switchcolumn
\vspace{.7cm}
\HatWithRefRed{Antienne d’introït.}{}
\switchcolumn*

\vspace{.5cm}
\Latin{\DropCapRed{O}{ptávi}{0}{0}, et datus est mihi sensus~:  et invocávi, et venit in me spíritus sapiéntiæ : et præpósui illam regnis et sédibus, et divítias nihil esse duxi in comparatióne illíus. √~\Ref{Ps 85, 1.} Inclína, Dómine, aurem tuam, et exáudi me, quóniam inops et pauper sum ego. √~Glória Patri. \Red{O}ptávi.}
\switchcolumn
\vspace{.5cm}
\French{\DropCapRed{J’}{ai}{0.1}{0} choisi, et l’intelligence m’a été donnée, et l’esprit de sagesse est venu en moi. Je l’ai placée avant les royaumes et les trônes, et j’ai estimé que les richesses n’étaient rien en comparaison d’elle. √~Inclinez votre oreille, Seigneur, et exaucez-moi, car je suis pauvre et sans ressource. √~Gloire au Père. \Red{J’}ai choisi.}
\switchcolumn*

\Hat{Oratio}
\switchcolumn
\Hat{Collecte}
\switchcolumn*

\vspace{.3cm}
\Latin{\DropCapRed{O}{mnípotens}{0}{0} sempitérne De\-us, qui hodiérnam festivitátem beáti Ambrósii Sacerdótis electióne consecrásti ; præsta pópulo tuo, ut cujus ánnua celebritáte devótis exsúltat obséquiis, ejus patrocínio tuæ pietátis consequátur auxília. Per Dóminum.}
\switchcolumn
\vspace{.3cm}
\French{\DropCapRed{D}{ieu}{0.1}{0} tout-puissant et éternel, qui avez consacré ce jour de fête par l’élection du bienheureux prêtre Ambroise, accordez à votre peuple qu’il obtienne les secours de votre bonté, grâce à la protection de celui qu’il honore chaque année de sa dévotion. Par Notre-Seigneur.}
\switchcolumn*

\HatWithRefBlackLatin{\Red{L}éctio Epístolæ beáti Pauli Apóstoli ad Ephésios.}{Ep 3, 2-11}
\switchcolumn
\HatWithRefBlackFrench{\Red{L}ecture de la lettre du bienheureux apôtre Paul aux Éphésiens.}{}
\switchcolumn*

\Latin{\DropCapRed{F}{ratres}{0}{0} : Audístis dispensatiónem  grátiæ Dei, quæ data est mihi in vobis : quóniam secúndum revelatiónem notum mihi factum est sacraméntum, sicut supra scripsi in brevi ; prout potéstis legéntes intellégere prudéntiam meam in mystério Christi, quod áliis generatiónibus non est ágnitum fíliis hóminum, sícuti nunc revelátum est sanctis apóstolis ejus et prophétis in Spíritu ; gentes esse cohéredes, et concorporáles, et compartícipes promissiónis ejus in Christo Jesu per Evangélium : cujus factus sum miníster, secúndum donum grátiæ Dei, quæ data est mihi secúndum operatiónem virtútis ejus. Mihi ómnium sanctórum mínimo data est grátia hæc, in géntibus evangelizáre investigábiles divítias Christi, et illumináre omnes, quæ sit dispensátio sacraménti abscónditi a sǽculis in Deo, qui ómnia creávit, ut innotéscat principátibus et potestátibus in cæléstibus per Ecclésiam multifórmis sapiéntia Dei ; secúndum præfinitiónem sæculórum, quam fecit in Christo Jesu Dómino nostro.}
\switchcolumn
\French{\DropCapRed{F}{rères}{0}{0} : Vous avez appris de quelle manière Dieu m’a donné la grâce de l’apostolat, pour l’exercer envers vous, m’ayant découvert par révélation ce mystère, dont je vous ai déjà écrit en peu de paroles, où vous pouvez connaître par la lecture que vous en ferez, quelle intelligence j’ai du mystère du Christ ; mystère qui n’a point été découvert aux enfants des hommes dans les autres temps, comme il est révélé maintenant par le Saint-Esprit à ses saints apôtres et aux prophètes. Ce mystère, c’est que que les païens sont appelés au même héritage que les Juifs ; qu’ils sont les membres d’un même corps, et qu’ils participent à la même promesse de Dieu en Jésus-Christ par l’Évangile, dont j’ai été fait le ministre par le don de la grâce de Dieu, qui m’a été conférée par l’opération de sa puissance. Car j’ai reçu, moi qui suis le plus petit d’entre tous les fidèles, cette grâce d’annoncer aux païens les richesses incompréhensibles de Jésus-Christ, et d’éclairer tous les hommes, en leur découvrant quelle est l’économie du mystère caché dès le commencement des siècles en Dieu, qui a créé toutes choses ; afin que les principautés et les puissances qui sont dans les cieux connaissent par l’Église la sagesse multiforme de Dieu, selon le dessein éternel qu’il a accompli par Jésus-Christ notre Seigneur.}
\switchcolumn*

\vspace{.5cm}
\Latin{\Red{Graduale. \textit{Gn 12, 1}.} Egrédere de terra tua, et de domo patris tui, et veni in terram, quam monstrábo tibi. √~\Ref{Ps 20, 3.} Desidérium cordis ejus tribuísti ei, et voluntáte labiórum ejus non fraudásti eum.}
\switchcolumn
\vspace{.5cm}
\French{\Red{Graduel.} Sors de ton pays et de la maison de ton père, et viens dans la terre, que je te montrerai. √~Vous lui avez accordé le désir de son cœur, et vous ne lui avez point refusé ce que voulaient ses lèvres.}
\switchcolumn*

\Latin{\Red{A}llelúia, allelúia. √~Diréctus est vir ínclitus ut Arium destrúeret, splendor Ecclésiæ, cláritas vatum, ínfulas dum gerit sǽculi acquisívit Paradísi. Allelúia.}
\switchcolumn
\French{\Red{A}lléluia, alléluia. √~Cet homme illustre a été choisi pour détruire Arius. Il est la splendeur de l’Église, la gloire des prophètes ; tandis qu’il porte les insignes du siècle, il acquiert ceux du Paradis. Alléluia.}
\switchcolumn*

\EvgLatin{Joánnem}{Jo 10, 11-16}
\switchcolumn
\EvgFrench{Jean}{}
\switchcolumn*

\Latin{\DropCapRed{I}{n}{0}{0} illo témpore : Dixit Jesus pharisǽis : Ego sum pastor bonus. Bonus pastor ánimam suam dat pro óvibus suis. Mercenárius autem, et qui non est pastor, cujus non sunt oves própriæ, videt lupum veniéntem, et dimíttit oves, et fugit : et lupus rapit, et dispérgit oves : mercenárius autem fugit, quia mercenárius est, et non pértinet ad eum de óvibus. Ego sum pastor bonus : et cognósco meas, et cognóscunt me meæ. Sicut novit me Pater, et ego agnósco Patrem : et ánimam meam pono pro óvibus meis. Et álias oves hábeo, quæ non sunt ex hoc ovíli : et illas opórtet me addúcere, et vocem meam áudient, et fiet unum ovíle, et unus pastor.}
\switchcolumn
\French{\DropCapRed{E}{n}{0}{0} ce temps-là, Jésus dit aux pharisiens : Je suis le bon Pasteur. Le bon pasteur donne sa vie pour ses brebis. Mais le mercenaire, qui n’est point pasteur, et à qui les brebis n’appartiennent pas, s’il voit venir le loup, abandonne les brebis et s’enfuit ; et le loup les ravit, et disperse le troupeau. Le mercenaire s’enfuit, parce qu’il est mercenaire, et qu’il ne se met point en peine des brebis. Moi, je suis le bon Pasteur : je connais mes brebis, et mes brebis me connaissent ; comme mon Père me connaît, et que je connais mon Père ; et je donne ma vie pour mes brebis. J’ai encore d’autres brebis qui ne sont pas de cette bergerie ; il faut aussi que je les amène. Elles écouteront ma voix, et il n’y aura qu’un troupeau, et qu’un Pasteur.}
\switchcolumn*

\vspace{.5cm}
\HatWithRefRed{Ant. ad Offertorium.}{Ez 34, 15-16}
\switchcolumn
\vspace{.5cm}
\HatWithRefRed{Antienne d’offertoire.}{}
\switchcolumn*

\Latin{Ego pascam oves meas, et ego eas accubáre fáciam, dicit Dóminus. Quod períerat requíram, et quod abjéctum erat redúcam, et quod confráctum fúerat alligábo, et quod infírmum fuerat consolidábo.}
\switchcolumn
\French{Je ferai paître mes brebis, et je les ferai reposer, dit le Seigneur. Ce qui s’était perdu je le rechercherai, ce qui avait été rejeté je le ramènerai, ce qui était brisé je le soignerai, et ce qui était infirme je le consoliderai.}
\switchcolumn*

\Hat{Secreta}
\switchcolumn
\Hat{Secrète}
\switchcolumn*

\vspace{.3cm}
\Latin{\DropCapRed{O}{mnípotens}{0}{0} sempitérne Deus,  múnera tuæ majestáti obláta, per intercessiónem beáti Ambrósii Confessóris tui atque Pontíficis, ad perpétuam nobis fac proveníre salútem. Per Dóminum.}
\switchcolumn
\vspace{.3cm}
\French{\DropCapRed{D}{ieu}{0.1}{0} tout-puissant et éternel, faites que les dons offerts à votre Majesté, par l’intercession du bienheureux Ambroise votre Confesseur et Pontife, servent à notre salut éternel. Par Notre-Seigneur.}
\switchcolumn*

\vspace{.5cm}
\HatWithRefRed{Ant. ad Comm.}{Si 50, 6-7}
\switchcolumn
\vspace{.5cm}
\HatWithRefRed{Antienne de communion.}{}
\switchcolumn*

\Latin{Quasi stella matutína in médio nébulæ, et quasi luna plena in diébus suis lucet, et quasi sol refúlgens, sic ille effúlsit in templo Dei.}
\switchcolumn
\French{Comme l’étoile du matin au milieu de la nuée, et comme la pleine lune  qui brille, et comme le soleil qui resplendit, ainsi il a brillé dans le temple de Dieu.}
\switchcolumn*

\Hat{Postcommunio}
\switchcolumn
\Hat{Postcommunion}
\switchcolumn*

\vspace{.3cm}
\Latin{\DropCapRed{S}{acraménta}{0}{0} salútis nostræ suscipiéntes, concéde, quǽsumus, omnípotens Deus : ut beáti Ambrósii Confessóris tui atque Pontíficis, nos úbique orátio ádjuvet ; in cujus veneratióne hæc tuæ obtúlimus majestáti. Per Dóminum.}
\switchcolumn
\vspace{.3cm}
\French{\DropCapRed{D}{ieu}{0.1}{0}tout-puissant, accordez-nous,  nous vous en prions : à nous qui recevons les mystères de notre salut, que la prière du bienheureux Ambroise, votre Confesseur et Pontife, en l’honneur de qui nous vous avons offert ce sacrifice, nous secoure en tout lieu. Par Notre-Seigneur.}
\switchcolumn*

%%%%%%%%%%%%%%%%%%

\vspace*{.3cm}
\sectionmark{12 décembre - Notre-Dame de Guadalupe}
\Date{Die 12 decembris}
\switchcolumn
\vspace*{.3cm}
\Date{Le 12 décembre}
\switchcolumn*

\Name{B.M.V. de Guadalupe}
\switchcolumn
\Name{N.-D. de guadalupe}
\switchcolumn*

\vspace{.7cm}
\HatWithRefRed{Ant. ad Introitum.}{Sedulius}
\switchcolumn
\vspace{.7cm}
\HatWithRefRed{Antienne d’introït.}{}
\switchcolumn*

\vspace{.5cm}
\Latin{\DropCapRed{S}{alve}{0}{0}, sancta Parens, eníxa puérpera Regem: qui cælum terrámque regit in sǽcula sæculórum. √~\Ref{Ps. 44, 2.} Eructávit cor meum verbum bonum : dico ego ópera mea Regi. √~Glória Patri. \Red{S}alve.}
\switchcolumn
\vspace{.5cm}
\French{\DropCapRed{S}{alut}{0.1}{0}, ô Mère, qui avez mis au monde le Roi, celui qui gouverne le ciel et la terre pour les siècles des siècles. √~Mon cœur a produit une belle parole, je dis mes œuvres au Roi. √~Gloire au Père. \Red{S}alut, ô Mère.}
\switchcolumn*

\Hat{Oratio}
\switchcolumn
\Hat{Collecte}
\switchcolumn*

\vspace{.3cm}
\Latin{\DropCapRed{D}{eus}{0}{0}, qui sub beatíssimæ Vírginis Maríæ singulári patrocínio constitútos, perpétuis benefíciis nos cumulári voluísti : præsta supplícibus tuis; ut cuius hódie commemoratióne lætámur in terris, eius conspéctu perfruámur in cælis. Per Dóminum.}
\switchcolumn
\vspace{.3cm}
\French{\DropCapRed{D}{ieu}{0.1}{0} qui avez voulu nous combler de bienfaits, nous qui sommes placés sous le glorieux patronage de la bienheureuse Vierge Marie, accordez à ceux qui vous supplient de jouir dans le ciel de la vue de celle que nous nous réjouissons de commémorer aujourd’hui sur la terre. Par N.-S.}
\switchcolumn*

\HatWithRefBlackLatin{\Red{L}éctio libri Sapiéntiæ.}{Si 24, 23-31}
\switchcolumn
\HatWithRefBlackFrench{\Red{L}ecture du Livre de la Sagesse.}{}
\switchcolumn*

\Latin{\DropCapRed{E}{go}{0}{0} quasi vitis fructificávi suavitátem odóris : et flores mei, fructus honóris et honestátis. Ego mater pulchræ dilectiónis, et timóris, et agnitiónis, et sanctæ spei. In me grátia omnis viæ et veritátis : in me omnis spes vitæ et virtútis. Transíte ad me, omnes qui concupíscitis me, et a genera- tiónibus meis implémini. Spíritus enim meus super mel dulcis, et heréditas mea super mel et favum. Memória mea in generatiónes sæculórum. Qui edunt me, adhuc esúrient : et qui bibunt me, adhuc sítient. Qui audit me, non con- fundétur: et qui operántur in me, non peccábunt. Qui elúcidant me, vitam ætérnam habébunt.}
\switchcolumn
\French{\DropCapRed{C}{omme}{0}{0} la vigne j’ai répandu une odeur suave, et mes fleurs sont des fruits d’honneur et de richesse. Je suis la mère du bel amour, de la crainte, de la connaissance et de la sainte espérance. En moi est la grâce de toute voie et de toute vérité, en moi toute espérance de vie et de vertu. Passez vers moi, vous tous qui me désirez, et rassasiez-vous de mes fruits. Car mon esprit est plus doux que le miel, et mon héritage plus que le miel et le rayon de miel. Ma mémoire demeure pour tous les siècles. Ceux qui me mangent auront encore faim, et ceux qui me boivent auront encore soif. Celui qui m’écoute ne sera pas confondu, et ceux qui travaillent avec moi ne pécheront point. Ceux qui me mettent en lumière auront la vie éternelle.}
\switchcolumn*

\vspace{.5cm}
\Latin{\Red{Graduale. \textit{Ct 6, 9}.} Quæ est ista, quæ progréditur quasi auróra consúrgens, pulchra ut luna, elécta ut sol ? √~\Ref{Si 50, 8.} Quasi arcus refúlgens inter nébulas glóriæ, et quasi flos rosárum in diébus vernis.}
\switchcolumn
\vspace{.5cm}
\French{\Red{Graduel.} Qui est celle-ci, qui s’avance comme l’aube qui se lève, belle comme la lune, éclatante comme le soleil ? √~Comme l’arc-en-ciel brillant parmi les nuées glorieuses, et comme la rose au printemps.}
\switchcolumn*

\Latin{\Red{A}llelúia, allelúia. \Ref{Ct 2, 12.} √~Flores apparuérunt in terra nostra, tempus putatiónis ádvenit. Allelúia.}
\switchcolumn
\French{\Red{A}lléluia, alléluia. √~Les fleurs sont apparues sur notre terre, le temps de la taille est venu. Alléluia.}
\switchcolumn*

\EvgLatin{Lucam}{Lc. 1, 39-47}
\switchcolumn
\EvgFrench{Luc}{}
\switchcolumn*

\Latin{\DropCapRed{I}{n}{0}{0} illo témpore : Exsúrgens María ábiit in montána cum festinatióne in civitátem Iuda: et intrávit in domum Zacharíæ, et salutávit Elísabeth. Et factum est, ut audívit salutatiónem Maríæ Elísabeth, exsultávit infans in útero eius : et repléta est Spíritu Sancto Elísabeth : et exclamávit voce magna, et dixit : Benedícta tu inter mulíeres, et benedíctus fructus ventris tui. Et unde hoc mihi, ut véniat Mater Dómini mei ad me ? Ecce enim, ut facta est vox salutatiónis tuæ in áuribus meis, exsultávit in gáudio infans in útero meo. Et beáta, quæ credidísti, quóniam perficiéntur ea, quæ dicta sunt tibi a Dómino. Et ait María : Magníficat ánima mea Dóminum: et exsultávit spíritus meus in Deo salutári meo.}
\switchcolumn
\French{\DropCapRed{E}{n}{0}{0} ce temps-là, Marie se mit en route ra- pidement vers les montagnes de Judée, en une ville de la tribu de Juda ; et étant entrée dans la maison de Zacharie, elle salua Élisabeth. Aussitôt qu’Élisabeth eut entendu la voix de Marie qui la saluait, son enfant tressaillit dans son sein, et Élisabeth fut remplie du Saint-Esprit. Alors élevant sa voix, elle s’écria : Vous êtes bénie entre toutes les femmes, et le fruit de votre sein est béni ; et d’où me vient ce bonheur, que la mère de mon Seigneur vienne vers moi ? Car votre voix n’a pas plutôt frappé mon oreille, lorsque vous m’avez saluée, que mon enfant a tressailli de joie dans mon sein. Et vous êtes bienheureuse d’avoir cru, parce que ce qui vous a été dit de la part du Seigneur, sera accompli. Alors Marie dit ces paroles : Mon âme glorifie le Seigneur, et mon esprit est ravi de joie en Dieu, mon Sauveur.}
\switchcolumn*

\vspace{.5cm}
\HatWithRefRed{Ant. ad Offertorium.}{2 Par 7, 16}
\switchcolumn
\vspace{.5cm}
\HatWithRefRed{Antienne d’offertoire.}{}
\switchcolumn*

\Latin{Elégi, et sanctificávi locum istum, ut sit ibi nomen meum, et permáneant óculi mei, et cor meum ibi cunctis diébus.}
\switchcolumn
\French{J’ai choisi et j’ai sanctifié ce lieu, pour que là soit mon Nom et que mes yeux et mon cœur y demeurent tous les jours.}
\switchcolumn*

\Hat{Secreta}
\switchcolumn
\Hat{Secrète}
\switchcolumn*

\vspace{.3cm}
\Latin{\DropCapRed{T}{ua}{0}{0}, Dómine, propitiatióne, et beátæ Maríæ semper Vírginis intercessióne, ad perpétuam, atque præséntem hæc oblátio nobis profíciat prosperitátem et pacem. Per Dóminum.}
\switchcolumn
\vspace{.3cm}
\French{\DropCapRed{P}{ar}{0.1}{0} votre bonté, Seigneur, et par l’inter- cession de la bienheureuse Marie toujours Vierge, que cette offrande serve à notre prospérité et notre paix, tant en ce monde que dans l’autre. Par Notre-Seigneur.}
\switchcolumn*

\vspace{.5cm}
\HatWithRefRed{Ant. ad Comm.}{Ps. 147, 20.}
\switchcolumn
\vspace{.5cm}
\HatWithRefRed{Antienne de communion.}{}
\switchcolumn*

\Latin{Non fecit táliter omni natióni : et iudícia sua non manifestávit eis.}
\switchcolumn
\French{Il n’a pas agi ainsi envers tous les peuples, et il ne leur a pas manifesté ses jugements.}
\switchcolumn*

\Hat{Postcommunio}
\switchcolumn
\Hat{Postcommunion}
\switchcolumn*

\vspace{.3cm}
\Latin{\DropCapRed{S}{umptis}{0}{0}, Dómine, salútis nostræ subsídiis : da, quǽsumus, beátæ Maríæ semper Vírginis patrocíniis nos ubíque prótegi ; in cuius veneratióne hæc tuæ obtúlimus maiestáti. Per Dóminum.}
\switchcolumn
\vspace{.3cm}
\French{\DropCapRed{N}{ous}{0.1}{0} avons reçu, Seigneur, les secours  de notre salut. Accordez-nous d’être partout protégés par les suffrages de la bienheureuse Marie toujours Vierge, en l’honneur de laquelle nous avons offert à votre Majesté ce sacrifice. Par Notre -Seigneur.}
\switchcolumn*

Die 14 decembris
\switchcolumn
Le 14 décembre
\switchcolumn*

Sancti Joannis a Cruce
\switchcolumn
Saint Jean de la Croix
\switchcolumn*

Conf. et Eccl. Doctoris
\switchcolumn
Conf. et Docteur de l’Église
\switchcolumn*

Ant. ad Introitum.\hfill Ga 6, 14
\switchcolumn
Antienne d’introït.
\switchcolumn*

Mihi autem absit gloriári, nisi  in cruce Dómini nostri Jesu Christi : per quem mihi mundus crucifíxus est, et ego mundo. √~Ps 118, 1. Beáti immaculáti in via : qui ambulant in lege Dómini. √~Glória Patri. Mihi autem.
\switchcolumn
Pour moi, que jamais je ne me glorifie, sinon en la Croix de Notre-Seigneur Jésus-Christ, par qui le monde est crucifié pour moi, et moi pour le monde. √~Heureux ceux qui sont sans tache dans le chemin, qui marchent dans la loi du Seigneur. √~Gloire au Père. Pour moi.
\switchcolumn*

Oratio
\switchcolumn
Collecte
\switchcolumn*

Deus, qui sanctum Ioánnem  Confessórem tuum atque Doctórem perféctæ sui abnegatiónis, et Crucis amatórem exímium effecísti : concéde ; ut, ejus imitatióni júgiter inhæréntes, glóriam assequámur ætérnam. Per Dñm.
\switchcolumn
Dieu qui avez inspiré à saint Jean, votre Confesseur et Docteur, un ardent amour du parfait renoncement à soi-même et de la Croix, accordez-nous d’obtenir la gloire éternelle en imitant sans cesse son exemple. Par Notre-Seigneur.
\switchcolumn*

Léctio epístolæ beáti Pauli Apóstoli ad Philippénses.
\switchcolumn
Lecture de la lettre du bienheureux Apôtre Paul aux Philippiens.
\switchcolumn*

Ph 3, 17-21 ; 4, 6-9
\switchcolumn

\switchcolumn*

Fratres : Imitatóres mei estóte, et observáte eos qui ita ámbulant, sicut habétis formam nostram. Multi enim ámbulant, quos sæpe dicébam vobis (nunc autem et flens dico) inimícos crucis Christi : quorum finis intéritus : quorum Deus venter est : et glória in confusióne ipsórum, qui terréna sápiunt. Nostra autem conversátio in cælis est : unde étiam Salvatórem exspectámus Dóminum nostrum Jesum Christum, qui reformábit corpus humilitátis nostræ, configurátum córpori claritátis suæ, secúndum operatiónem, qua étiam possit subjícere sibi ómnia. Nihil sollíciti sitis : sed in omni oratióne, et obsecratióne, cum gratiárum actióne petitiónes vestræ innotéscant apud Deum ; et pax Dei, quæ exsúperat omnem sensum, custódiat corda vestra, et intelligéntias vestras in Christo Jesu. De cétero fratres, quæcúmque sunt vera, quæcúmque púdica, quæcúmque justa, quæcúmque sancta, quæcúmque amabília, quæcúmque bonæ famæ, si qua virtus, si qua laus disciplínæ, hæc cogitáte. Quæ et didicístis, et accepístis, et audístis, et vidístis in me, hæc ágite : et Deus pacis erit vobíscum.
\switchcolumn
Frères, soyez mes imitateurs, et pro- posez-vous l’exemple de ceux qui se conduisent selon le modèle que vous avez vu en nous. Car il y en a beaucoup dont je vous ai souvent parlé, et dont je vous parle encore avec larmes, qui se conduisent en ennemis de la croix de Jésus-Christ, qui auront pour fin la damnation, qui font leur dieu de leur ventre, qui mettent leur gloire dans ce qui fait leur honte, et qui n’ont de goût que pour la terre. Mais pour nous, notre conversation est dans le ciel ; et c’est de là aussi que nous attendons le Sauveur, notre Seigneur Jésus-Christ, qui transformera notre corps misérable, afin de le rendre conforme à son corps glorieux, par cette vertu efficace par laquelle il peut s’assujettir toutes choses. Ne vous inquiétez de rien ; mais en quelque état que vous soyez, présentez à Dieu vos demandes par des supplications et des prières, accompagnées d’actions de grâces. Et que la paix de Dieu, qui surpasse toutes pensées, garde vos cœurs et vos esprits dans le Christ Jésus. Enfin, mes frères, que tout ce qui est véritable, tout ce qui est honnête, tout ce qui est juste, tout ce qui est saint, tout ce qui est aimable, tout ce qui est de bonne réputation, tout ce qui est vertueux, et tout ce qui est louable dans le règlement des mœurs, soit l’entretien de vos pensées. Pratiquez ce que vous avez appris et reçu de moi, ce que vous avez entendu dire de moi, et ce que vous avez vu en moi ; et le Dieu de paix sera avec vous.
\switchcolumn*

Graduale. Mt 16, 24. Qui vult post me veníre, ábneget semetípsum et tollat crucem suam, et sequátur me. √~Is 26, 9. Anima mea desiderávit te in nocte : sed et spíritu meo in præcórdiis meis de mane vigilábo ad te.
\switchcolumn
Graduel. Que celui qui veut venir derrière moi renonce à lui-même, qu’il porte sa croix et qu’il me suive. √~Mon âme vous a désiré dans la nuit ; au plus profond de mon cœur, mon esprit vous cherche dès le matin.
\switchcolumn*

Allelúia, allelúia. √~Si 51, 18.22. Quæsívi sapiéntiam in oratióne mea : multam invéni in meípso, et multum proféci in ea. Allelúia.
\switchcolumn
Alléluia, alléluia. √~J’ai cherché la sagesse dans mon oraison ; elle a abondé en moi, et j’ai beaucoup progressé grâce à elle. Alléluia.
\switchcolumn*

Sequéntia sancti Evangélii secúndum Lucam.
\switchcolumn
Lecture du saint évangile selon saint Luc.
\switchcolumn*

Lc 11, 33-36
\switchcolumn

\switchcolumn*

In illo témpore : Dixit Jesus discípulis suis : Nemo lucérnam accéndit, et in abscóndito ponit, neque sub módio : sed supra candelábrum, ut qui ingrediúntur, lumen vídeant. Lucérna córporis tui est óculus tuus. Si óculus tuus fúerit simplex, totum corpus tuum lúcidum erit : si autem nequam fúerit, étiam corpus tuum tenebrósum erit. Vide ergo ne lumen quod in te est, ténebræ sint. Si ergo corpus tuum totum lúcidum fúerit, non habens áliquam partem tenebrárum, erit lúcidum totum, et sicut lucérna fulgóris illuminábit te.
\switchcolumn
En ce temps-là, Jésus dit à ses disciples : Il n’y a personne qui, après avoir allumé une lampe, la mette dans un lieu caché ou sous un boisseau ; mais on la met sur le chandelier, afin que ceux qui entrent voient la lumière. Ton œil est la lampe de ton corps : si ton œil est simple et pur, tout ton corps sera éclairé ; s’il est mauvais, ton corps aussi sera ténébreux. Prends donc garde que la lumière qui est en toi ne soit elle-même ténèbres. Si donc ton corps est tout éclairé, n’ayant aucune partie ténébreuse, tout sera lumineux, et comme une lampe brillante il t’éclairera.
\switchcolumn*

Ant. ad Offertorium.\hfill Is 60, 19
\switchcolumn
Antienne d’offertoire.
\switchcolumn*

Erit tibi Dóminus in lucem sempitérnam, et Deus tuus in glóriam tuam.
\switchcolumn
Le Seigneur sera pour toi une lumière éternelle, et ton Dieu sera ta gloire.
\switchcolumn*

Secreta
\switchcolumn
Secrète
\switchcolumn*

Offérimus tibi, Dómine, hóstiam  laudis in honórem sancti Joánnis Confessóris tui atque Doctóris, qui assíduam crucis mortificatiónem in semetípso portans, tibi fuit hóstia grata, atque jucúnda : Qui vivis.
\switchcolumn
Nous vous offrons, Seigneur, l’hostie de louange en l’honneur de saint Jean, votre Confesseur et Docteur, lui qui, portant toujours en lui-même la mortification de la croix, vous fut une hostie agréable et suave. Vous qui vivez et régnez.
\switchcolumn*

Ant. ad Comm.\hfill Is 53, 11.10
\switchcolumn
Antienne de communion.
\switchcolumn*

Pro eo quod laborávit ánima ejus, vidébit semen longǽvum, et volúntas Dómini in manu ejus dirigétur.
\switchcolumn
Parce que son âme a peiné, il verra une descendance pour de nombreuses générations, et la volonté du Seigneur sera dans sa main.
\switchcolumn*

Postcommunio
\switchcolumn
Postcommunion
\switchcolumn*

Prǽbeant nobis, Dómine, divínum tua sancta fervórem, intercedénte sancto Joánne Confessóre tuo atque Doctóre : et præsta ; ut, sicut illum, dum hæc sacra mystéria perágeret, caritátis igne cǽlitus immísso, étiam extérius irradiáre fecísti : ita nos ejúsdem caritátis ígnibus succénsi, ad cæléstia júgiter aspirémus : Qui vivis.
\switchcolumn
Que vos sacrements, Seigneur, nous pro- curent un accroissement de ferveur, grâce à l’intercession de saint Jean, votre Confesseur et Docteur ; et accordez-nous que, de même que vous le rendiez, en lui envoyant du ciel le feu de la charité, rayonnant jusque dans son corps, tandis qu’il célébrait ces saints mystères, ainsi nous aspirions sans cesse aux réalités célestes, en brûlant du même feu d’amour. Vous qui vivez.
\switchcolumn*

Die 2 januarii
\switchcolumn
Le 2 janvier
\switchcolumn*

SS. Basilii Magni
\switchcolumn
S. Basile le Grand et
\switchcolumn*

et Gregorii Nazianzeni
\switchcolumn
S. Grégoire de Nazianze
\switchcolumn*

Ep. et Eccl. Doctorum
\switchcolumn
Évêques et Docteurs de l’Église
\switchcolumn*

Ant. ad Introitum.\hfill Ps 131, 16-17
\switchcolumn
Antienne d’introït.
\switchcolumn*

Sacerdótes Sion índuam salutári, et sancti ejus exsultatióne exsultábunt, dicit Dóminus : illuc prodúcam cornu David, parávi lucérnam Christo meo. √~Ibid., 1. Meménto, Dómine, David : et omnis mansuetúdinis ejus. √~Glória Patri. Sacerdótes Sion.
\switchcolumn
Je revêtirai du salut les prêtres de Sion, et  ses saints se réjouiront, dit le Seigneur : là je susciterai une force pour David, je préparerai une lampe pour mon Christ. √~Souvenez-vous, Seigneur, de David, et de toute sa douceur. √~Gloire au Père. Je revêtirai.
\switchcolumn*

Oratio
\switchcolumn
Collecte
\switchcolumn*

Deus, qui nos beatórum Basílii et  Gregórii Confessórum tuórum atque Pontíficum confessiónibus gloriósis circúmdas et prótegis : da nobis, et eórum imitatióne profícere, et intercessióne gaudére. Per Dóminum.
\switchcolumn
Ô Dieu qui nous entourez et nous pro- tégez par les vies glorieuses de vos Confesseurs Basile et Grégoire, accordez-nous de progresser en les imitant et de nous réjouir de leur intercession. Par Notre-Seigneur.
\switchcolumn*

Alia Oratio
\switchcolumn
Autre Collecte
\switchcolumn*

Ecclésiam tuam, Dómine, beatórum Basílii et Gregórii Confessórum tuórum atque Pontíficum contínua protectióne custódi : ut, sicut illos pastorális sollicitúdo gloriósos réddidit ; ita nos eórum intercéssio in tuo semper fáciat amóre fervéntes. Per Dóminum.
\switchcolumn
Conservez votre Église, Seigneur, grâce à la protection continuelle de vos Confesseurs et Pontifes, les bienheureux Basile et Grégoire ; de sorte que, de même que leur zèle pastoral les couvrit de gloire, ainsi leur intercession nous rende toujours fervents par votre amour. Par Notre-Seigneur.
\switchcolumn*

Léctio epístolæ beáti Pauli Apóstoli ad Hebrǽos.
\switchcolumn
Lecture de la lettre du bienheureux apôtre Paul aux Hébreux.
\switchcolumn*

Hb 13, 7-17
\switchcolumn

\switchcolumn*

Fratres : Mementóte præpositórum vestrórum, qui vobis locúti sunt verbum Dei : quorum intuéntes éxitum conversatiónis, imitámini fidem. Jesus Christus heri, et hódie : ipse et in sǽcula. Doctrínis váriis et peregrínis nolíte abdúci. Optimum est enim grátia stabilíre cor, non escis : quæ non profuérunt ambulántibus in eis. Habémus altáre, de quo édere non habent potestátem, qui tabernáculo desérviunt. Quorum enim animálium infértur sanguis pro peccáto in Sancta per pontíficem, horum córpora cremántur extra castra. Propter quod et Jesus, ut sanctificáret per suum sánguinem pópulum, extra portam passus est. Exeámus ígitur ad eum extra castra, impropérium ejus portántes. Non enim habémus hic manéntem civitátem, sed futúram inquírimus. Per ipsum ergo offerámus hóstiam laudis semper Deo, id est, fructum labiórum confiténtium nómini ejus. Beneficéntiæ autem et communiónis nolíte oblivísci : tálibus enim hóstiis promerétur Deus. Obedíte præpósitis vestris, et subjacéte eis. Ipsi enim pervígilant quasi ratiónem pro animábus vestris redditúri.
\switchcolumn
Frères, souvenez-vous de vos supérieurs, qui vous ont prêché la parole de Dieu ; et considérant quelle a été la fin de leur vie, imitez leur foi. Jésus-Christ était hier, il est aujourd’hui, et il sera le même dans tous les siècles. Ne vous laissez point emporter à une diversité d’opinions et à des doctrines étrangères. Car il est bon d’affermir son cœur par la grâce, au lieu de s’appuyer sur des discernements de nourritures, qui n’ont point servi à ceux qui les ont observés. Nous avons un autel dont les ministres du tabernacle n’ont pas le pouvoir de manger. Car les corps des animaux, dont le sang est porté par le pontife dans le sanctuaire pour l’expiation du péché, sont brûlés hors du camp. Et c’est pour cette raison que Jésus, devant sanctifier le peuple par son propre sang, a souffert hors de la porte de la ville. Sortons donc aussi hors du camp, et allons à lui en portant l’ignominie de sa croix. Car nous n’avons point ici de ville permanente ; mais nous cherchons celle où nous devons habiter un jour. Offrons donc par lui sans cesse à Dieu une hostie de louange, c’est-à-dire le fruit de lèvres qui rendent gloire à son nom. Souvenez-vous d’exercer la charité, et de faire part de vos biens aux autres : car c’est par de semblables hosties qu’on se rend Dieu favorable. Obéissez à vos supérieurs, et soyez soumis à leur autorité : car eux veillent, devant rendre compte de vos âmes.
\switchcolumn*

Graduale. Ps 106, 22.32. Sacrífi-cent Dómino sacrifícium laudis : et annúntient ópera ejus in exsultatióne. √~Et exáltent eum in ecclésia plebis : et in cáthedra seniórum laudent eum.
\switchcolumn
Graduel. Qu’ils offrent au Seigneur un sacrifice de louange, et qu’ils annoncent ses œuvres dans l’allégresse. √~Et qu’ils L’exaltent parmi l’assemblée du peuple, et qu’ils le louent dans la chaire des anciens.
\switchcolumn*

Allelúia, allelúia. √~2 Par 6, 41. Sacerdótes tui, Dómine Deus, índuant salútem, et sancti tui læténtur in bonis. Allelúia.
\switchcolumn
Alléluia, alléluia. √~Que vos prêtres, Seigneur Dieu, revêtent le salut, et que vos saints jubilent dans le bonheur. Alléluia.
\switchcolumn*

Sequéntia sancti Evangélii secúndum Marcum.
\switchcolumn
Lecture du saint évangile selon saint Marc.
\switchcolumn*

Mc 13, 33-37
\switchcolumn

\switchcolumn*

In illo témpore : Dixit Jesus discípulis suis : Vidéte, vigiláte, et oráte : nescítis enim quando tempus sit. Sicut homo qui péregre proféctus relíquit domum suam, et dedit servis suis potestátem cujúsque óperis, et janitóri præcépit ut vígilet. Vigiláte ergo (nescítis enim quando Dóminus domus véniat : sero, an média nocte, an galli cantu, an mane), ne, cum vénerit repénte, invéniat vos dormiéntes. Quod autem vobis dico, ómnibus dico : Vigiláte.
\switchcolumn
En ce temps-là, Jésus dit à ses disciples : Prenez garde à vous, veillez et priez ; parce que vous ne savez quand le temps viendra. Car il en sera comme d’un homme qui, s’en allant faire un voyage, laissa sa maison sous la conduite de ses serviteurs, marquant à chacun ce qu’il devait faire, et recommanda au portier de veiller. Veillez donc de même ; puisque vous ne savez pas quand le maître de la maison viendra, si ce sera le soir, ou à minuit, ou au chant du coq, ou le matin ; de peur que survenant tout d’un coup, il ne vous trouve endormis. Or, ce que je vous dis, je le dis à tous : Veillez.
\switchcolumn*

Ant. ad Offertorium.\hfill Ps 105, 3
\switchcolumn
Antienne d’offertoire.
\switchcolumn*

Beáti, qui custódiunt judícium, et fáciunt justítiam in omni témpore.
\switchcolumn
Heureux ceux qui gardent le droit et qui font en tout temps ce qui est juste.
\switchcolumn*

Secreta
\switchcolumn
Secrète
\switchcolumn*

Múnera nostra, Dómine, sacris altáribus offeréntes, quǽsumus cleméntiam tuam : ut éadem, suffragántibus beatórum Basílii et Gregórii Pontíficum méritis, et suprémam tibi glóriam operéntur, et ubérrimam nobis grátiam assequántur. Per Dóminum.
\switchcolumn
En apportant nos offrandes sur votre autel, Seigneur, nous prions votre bonté : que ces mêmes présents, grâce aux mérites des bienheureux pontifes Basile et Grégoire, vous rendent la plus grande gloire, et qu’ils nous obtiennent une grâce surabondante. Par Notre-Seigneur.
\switchcolumn*

Alia Secreta
\switchcolumn
Autre Secrète
\switchcolumn*

Hanc nostræ oblatiónis hóstiam,  Deus, gratam óculis tuæ majestátis effíciant beatórum Basílii et Gregórii Pontíficum expetíta suffrágia : qui digne in hoc sǽculo sacrifícia tibi ac preces in salútem pópuli obtulérunt. Per Dóminum.
\switchcolumn
Que les prières que nous implorons  des bienheureux Pontifes Basile et Grégoire, ô Dieu, rendent agréable aux yeux de votre Majesté nos offrandes que voici ; eux qui vous offrirent ici bas de dignes sacrifices et des prières pour le salut du peuple. Par Notre-Seigneur.
\switchcolumn*

Ant. ad Comm.\hfill Mc 13, 34
\switchcolumn
Antienne de communion.
\switchcolumn*

Homo péregre proféctus relíquit domum suam, et dedit servis suis potestátem cujúsque óperis, et janitóri præcépit ut vígilet.
\switchcolumn
Un homme partit en voyage. Il quitta sa maison, donna à chacun de ses serviteurs un pouvoir selon sa tâche, et prescrivit au portier de veiller.
\switchcolumn*

Postcommunio
\switchcolumn
Postcommunion
\switchcolumn*

Refectióne sacra enutrítos, fac  nos, omnípotens Deus, vestígiis beatórum Basílii et Gregórii Pontíficum semper insístere : qui studuérunt pérpeti devotióne te cólere, et indeféssa ómnibus caritáte profícere. Per Dóminum.
\switchcolumn
Accordez-nous la grâce, Dieu tout- puissant, à nous qui avons été nourris à la Table sainte, de marcher toujours sur les traces des bienheureux Pontifes Basile et Grégoire, eux qui eurent soin de toujours vous  honorer par une dévotion continuelle, et de grandir sans cesse dans une charité inépuisable envers tous. Par Notre-Seigneur.
\switchcolumn*

Alia Postcommunio
\switchcolumn
Autre Postcommunion
\switchcolumn*

Mensa cæléstis, omnípotens  Deus, intercedéntibus bea-tórum Basílii et Gregórii Pontíficum méritis, supérnas in ómnibus vires firmet et áugeat : ut et fídei donum íntegrum custodiámus, et per osténsum salútis trámitem ambulémus. Per Dóminum.
\switchcolumn
Que la Table céleste, Dieu tout-puissant, par les mérites des bienheureux Pontifes Basile et Grégoire, affermisse et augmente en tous la force qui vient d’en-haut, afin que nous gardions intact le don de la foi, et que nous marchions sur le chemin du salut qu’ils nous ont montré. Par Notre-Seigneur.
\switchcolumn*

Die 15 januarii
\switchcolumn
Le 15 janvier
\switchcolumn*

SS. Mauri et Placidi,
\switchcolumn
S. Maur et S. Placide
\switchcolumn*

Discipulorum
\switchcolumn
Disciples de
\switchcolumn*

S. P. N. Benedicti
\switchcolumn
N. P. S. Benoît
\switchcolumn*

Ant. ad Introitum.\hfill Ps 144, 10-11
\switchcolumn
Antienne d’introït.
\switchcolumn*

Confiteántur tibi, Dómine, ómnia  ópera tua, et sancti tui benedícant tibi : glóriam regni tui dicent, et poténtiam tuam loquéntur. √~Ibid., 1. Exaltábo te, Deus meus Rex : et benedícam nómini tuo in sǽculum, et in sǽculum sǽculi. √~Glória Patri. Confiteántur tibi.
\switchcolumn
Que toutes vos œuvres, Seigneur, vous  louent, et que vos saints vous bénissent ; qu’ils disent la gloire de votre règne, et qu’ils racontent votre puissance. √~Je vous exalterai, ô Dieu, mon Roi, et je bénirai votre nom à jamais, et pour les siècles des siècles. √~Gloire au Père. Que toutes vos œuvres.
\switchcolumn*

Oratio
\switchcolumn
Collecte
\switchcolumn*

Concéde, quǽsumus, omnípotens  Deus : ut ad meliórem vitam beatórum Mauri et Plácidi Confessórum tuórum exémpla nos próvocent ; quátenus, quorum memóriam ágimus, étiam actus imitémur. Per Dóminum.
\switchcolumn
Accordez-nous, nous vous en prions,  Dieu tout-puissant, que les exemples de vos bienheureux Confesseurs Maur et Placide nous excitent à mener une vie meilleure, de sorte que nous imitions les actions de ceux dont nous faisons mémoire. Par Notre-Seigneur.
\switchcolumn*

Alia oratio
\switchcolumn
Autre Collecte
\switchcolumn*

Deus, qui nos beatórum Mauri et  Plácidi Confessórum tuórum méritis et intercessióne lætíficas : concéde propítius ; ut, qui tua per eos benefícia póscimus, dono tuæ grátiæ consequámur. Per Dóminum.
\switchcolumn
Ô Dieu qui nous réjouissez par les mérites et l’intercession de vos bienheureux Confesseurs Maur et Placide, accordez-nous dans votre bonté, que le don de votre grâce nous fasse obtenir ces bienfaits que nous sollicitons par eux. Par Notre-Seigneur.
\switchcolumn*

Léctio libri Sapiéntiæ.
\switchcolumn
Lecture du Livre de la Sagesse.
\switchcolumn*

Si 2, 7-13
\switchcolumn

\switchcolumn*

Metuéntes Dóminum, sustinéte  misericórdiam ejus : et non deflectátis ab illo ne cadátis. Qui timétis Dóminum, crédite illi : et non evacuábitur merces vestra. Qui timétis Dóminum, speráte in illum : et in oblectatiónem véniet vobis misericórdia. Qui timétis Dóminum, dilígite illum : et illuminabúntur corda vestra. Respícite, fílii, natiónes hóminum : et scitóte quia nullus sperávit in Dómino, et confúsus est. Quis enim permánsit in mandátis ejus, et derelíctus est ? Aut quis invocávit eum, et despéxit illum ? Quóniam pius et miséricors est Deus, et remíttet in die tribulatiónis peccáta : et protéctor est ómnibus exquiréntibus se in veritáte.
\switchcolumn
Vous qui craignez le Seigneur,  attendez patiemment sa miséricorde, et ne vous détournez pas de lui, de peur que vous ne tombiez. Vous qui craignez le Seigneur, croyez en lui, et votre récompense ne sera pas anéantie. Vous qui craignez le Seigneur, espérez en lui, et sa miséricorde vous apportera la joie. Vous qui craignez le Seigneur, aimez-le, et vos cœurs seront illuminés. Considérez, mes enfants, les générations des hommes, et sachez que nul n’a espéré dans le Seigneur et a été confondu. Qui en effet a persévéré dans ses commandements et a été abandonné ? Ou qui l’a invoqué et a été méprisé ? Car le Seigneur est bon et miséricordieux, et il remet les péchés au jour de la tribulation, et il est le protecteur de tous ceux qui le cherchent en vérité.
\switchcolumn*

Graduale. Ps 30, 24-25. Dilígite Dóminum, omnes sancti ejus, quóniam veritátem requíret Dóminus, et retríbuet abundánter faciéntibus supérbiam. √~Viríliter ágite, et confortétur cor vestrum, omnes qui sperátis in Dómino.
\switchcolumn
Graduel. Aimez le Seigneur, vous tous ses saints, car le Seigneur recherche la fidélité, et il châtiera sévèrement les superbes. √~Agissez virilement et que votre cœur se fortifie, vous tous qui espérez dans le Seigneur.
\switchcolumn*

Allelúia, allelúia. √~Ps 9, 11. Sperent in te qui novérunt nomen tuum : quóniam non dereliquísti quæréntes te, Dómine. Allelúia.
\switchcolumn
Alléluia, alléluia. √~Que ceux qui connaissent votre nom espèrent en vous, car vous n’abandonnez pas ceux qui vous cherchent, Seigneur. Alléluia.
\switchcolumn*

Sequéntia sancti Evangélii secúndum Matthǽum.
\switchcolumn
Lecture du saint évangile selon saint Matthieu.
\switchcolumn*

Mt 14, 28-33
\switchcolumn

\switchcolumn*

In illo témpore : Respóndens Petrus, dixit : Dómine, si tu es, jube me ad te veníre super aquas. At ipse ait : Veni. Et descéndens Petrus de navícula, ambulábat super aquam ut veníret ad Jesum. Videns vero ventum válidum, tímuit : et cum cœpísset mergi, clamávit dicens : Dómine, salvum me fac. Et contínuo Jesus exténdens manum, apprehéndit eum, et ait illi : Módicæ fídei, quare dubitásti ? Et cum ascendíssent in navículam, cessávit ventus. Qui autem in navícula erant, venérunt, et adoravérunt eum, dicéntes : Vere Fílius Dei es.
\switchcolumn
En ce temps-là, Pierre prenant la parole dit : Seigneur, si c’est toi, ordonne que je vienne à toi sur les eaux. Et Jésus lui dit : « Viens ! ». Et Pierre, descendant de la barque, marchait sur l’eau pour venir à Jésus. Mais voyant le vent violent il eut peur, et comme il commençait à s’enfoncer, il cria : « Seigneur, sauve-moi ! ». Et aussitôt Jésus, étendant la main, le tira vers lui et lui dit : « Homme de peu de foi, pourquoi as-tu douté ? ». Et comme ils montaient dans la barque, le vent tomba. Or, ceux qui étaient dans la barque vinrent et l’adorèrent en disant : « Vraiment tu es le Fils de Dieu. »
\switchcolumn*

Ant. ad Offertorium.\hfill Ps 67, 4
\switchcolumn
Antienne d’offertoire.
\switchcolumn*

Justi epuléntur, et exsúltent in conspéctu Dei, et delecténtur in lætítia.
\switchcolumn
Que les justes festoient et exultent en présence de Dieu, et qu’ils se délectent dans la joie.
\switchcolumn*

Secreta
\switchcolumn
Secrète
\switchcolumn*

Hóstias ad altáre tuum offeréntibus,  Dómine, da nobis illum pietátis afféctum, quem beátis Mauro et Plácido Confessóribus tuis infudísti : ut pura mente ac férvido corde rei sacræ attendámus, et sacrifícium tibi plácitum nobísque profícuum immolémus. Per Dóminum.
\switchcolumn
Donnez-nous, Seigneur, à nous qui of- frons des hosties sur votre autel, cet esprit de piété que vous avez accordé aux bienheureux Maur et Placide vos Confesseurs ; de sorte que nous assistions aux saints Mystères avec un esprit pur et un cœur fervent, et que nous immolions un sacrifice qui vous soit agréable et qui nous soit profitable. Par Notre-Seigneur.
\switchcolumn*

Alia secreta
\switchcolumn
Autre Secrète
\switchcolumn*

Múnera, Dómine, obláta sanctífica :  et, intercedéntibus beátis Mauro et Plácido Confessóribus tuis, nos per hæc a peccatórum nostrórum máculis emúnda. Per Dóminum.
\switchcolumn
Sanctifiez, Seigneur, les dons que nous  offrons, et par l’intercession des bienheureux Maur et Placide, vos Confesseurs, purifiez-nous par ces mêmes offrandes des souillures de nos péchés. Par Notre-Seigneur.
\switchcolumn*

Ant. ad Comm.\hfill Lc 12, 37
\switchcolumn
Antienne de communion.
\switchcolumn*

Beáti servi illi, quos, cum vénerit Dóminus, invénerit vigilántes : amen dico vobis, quod præcínget se, et fáciet illos discúmbere, et tránsiens ministrábit illis.
\switchcolumn
Bienheureux ces serviteurs que le Maître, quand il viendra, trouvera veillants. Amen je vous le dis, il se ceindra, les fera mettre à table et les servira chacun à son tour.
\switchcolumn*

Postcommunio
\switchcolumn
Postcommunion
\switchcolumn*

Tríbuat nobis, omnípotens Deus,  suffragántibus beatórum Mauri et Plácidi Confessórum tuórum précibus, reféctio sacra subsídium : ut et castitátis mundítiam observémus in córpore, et lumen veritátis exhibeámus in ópere. Per Dóminum.
\switchcolumn
Que ce repas saint, Dieu tout-puissant,  nous fortifie par les prières des bienheureux Maur et Placide, vos Confesseurs, en sorte que nous conservions la pureté de la chasteté dans nos corps, et que nous fassions resplendir dans nos œuvres la lumière de la vérité. Par Notre-Seigneur.
\switchcolumn*

Alia Postcommunio
\switchcolumn
Autre Postcommunion
\switchcolumn*

Súpplices te rogámus, omnípotens  Deus : ut, quos tuis réficis sacraméntis, intercedéntibus beátis Mauro et Plácido Confessóribus tuis, tibi étiam plácitis móribus dignánter tríbuas deservíre. Per Dóminum.
\switchcolumn
Nous vous demandons en suppliant,  Dieu tout-puissant : donnez à ceux que vous restaurez par vos sacrements de vous servir à leur tour dignement par une sainte vie, grâce à l’intercession des bienheureux Maur et Placide, vos Confesseurs. Par Notre-Seigneur.
\switchcolumn*

Die 22 januarii
\switchcolumn
Le 22 janvier
\switchcolumn*

S. Vincentii
\switchcolumn
S. Vincent
\switchcolumn*

Martyris
\switchcolumn
Martyr
\switchcolumn*

Ant. ad Introitum.\hfill Ps 63, 11
\switchcolumn
Antienne d’introït.
\switchcolumn*

Lætábitur justus in Dómino, et  sperábit in eo : et laudabúntur omnes recti corde. √~Ibid., 2. Exáudi, Deus, oratiónem meam cum déprecor : a timóre inimíci éripe ánimam meam. √~Glória Patri. Lætábitur.
\switchcolumn
Le juste se réjouira dans le Seigneur et il  espérera en lui, et tous les cœurs droits seront loués. √~Exaucez, ô Dieu, ma prière lorsque je prie ; délivrez mon âme de la crainte de l’ennemi. √~Gloire au Père. Le juste se réjouira.
\switchcolumn*

Oratio
\switchcolumn
Collecte
\switchcolumn*

Adésto, quǽsumus, Dómine,  supplicatiónibus nostris : ut, qui ex iniquitáte nostra reos nos esse cognóscimus, beáti Vincéntii Mártyris tui intercessióne liberémur. Per Dóminum.
\switchcolumn
Soyez attentif, nous vous en prions,  Seigneur, à nos supplications, afin que nous, qui nous reconnaissons coupables à cause de notre iniquité, nous soyons libérés par l’intercesion du bienheureux Vincent, votre martyr. Par Notre-Seigneur.
\switchcolumn*

Léctio Isaíæ Prophétæ.
\switchcolumn
Lecture du prophète Isaïe.
\switchcolumn*

Is 43, 1-5
\switchcolumn

\switchcolumn*

Hæc dicit Dóminus creans te, Jacob,  et formans te, Israel : Noli timére, quia redémi te, et vocávi te nómine tuo : meus es tu. Cum transíeris per aquas, tecum ero, et flúmina non opérient te ; cum ambuláveris in igne, non comburéris, et flamma non ardébit in te. Quia ego Dóminus Deus tuus, Sanctus Israel, salvátor tuus, dedi propitiatiónem tuam Ægýptum, Æthiópiam et Saba pro te. Ex quo honorábilis factus es in óculis meis, et gloriósus, ego diléxi te, et dabo hómines pro te, et pópulos pro ánima tua. Noli timére, quia ego tecum sum : Dóminus Deus tuus.
\switchcolumn
Voici ce que dit le Seigneur qui te crée,  ô Jacob, et qui te forme, ô Israël : Ne crains pas, car je t’ai racheté et je t’ai appelé par ton nom. Tu es à moi. Quant tu passeras par les flots, je serai avec toi, et les fleuves ne pourront te recouvrir. Quand tu marcheras au milieu du feu, tu ne seras pas brûlé, et la flamme n’aura point de prise sur toi. Car je suis le Seigneur ton Dieu, le Saint d’Israël, ton Sauveur. J’ai donné pour ta propitiation l’Égypte, l’Éthiopie et Saba pour toi. Depuis que tu as été précieux et glorieux à mes yeux, je t’ai aimé, et je donnerai des hommes pour toi, des peuples pour ton âme. Ne crains pas car je suis avec toi, moi le Seigneur ton Dieu.
\switchcolumn*

Graduale. Ps 20, 4. Posuísti, Dómine, in cápite ejus corónam de lápide pretióso. √~Ibid., 3. Desidérium ánimæ ejus tribuísti ei : et voluntáte labiórum ejus non fraudásti eum.
\switchcolumn
Graduel. Vous avez placé sur sa tête, Seigneur, une couronne de pierres précieuses. √~Vous lui avez accordé le désir de son âme, et vous ne l’avez point frustré de ce que vous ont demandé ses lèvres.
\switchcolumn*

Allelúia, allelúia. √~Ps 111, 1. Beátus vir, qui timet Dóminum : in mandátis ejus cupit nimis. Allelúia.
\switchcolumn
Alléluia, alléluia. √~Heureux l’homme qui craint le Seigneur, son désir est tout entier dans ses commandements. Alléluia.
\switchcolumn*

Post Septuagesimam, omissis Alléluia et versu sequenti, dicitur :
\switchcolumn
Après la Septuagésime, on omet l’Alléluia et son verset, et l’on dit le Trait :
\switchcolumn*

Tractus. Ps 20, 3-4. Desidérium ánimæ ejus tribuísti ei : et voluntáte labiórum ejus non fraudásti eum. √~Quóniam prævenísti eum in benedictiónibus dulcédinis. √~Posuísti in cápite ejus corónam de lápide pretióso.
\switchcolumn
Trait. Vous lui avez accordé le désir de son âme, et vous ne l’avez point frustré de ce que vous ont demandé ses lèvres. √~Car vous l’avez prévenu de douces bénédictions. √~Vous avez placé sur sa tête une couronne de pierres précieuses.
\switchcolumn*

Sequéntia sancti Evangélii secúndum Joánnem.
\switchcolumn
Lecture du saint évangile selon saint Jean
\switchcolumn*

Jo 12, 24-26
\switchcolumn

\switchcolumn*

In illo témpore : Dixit Jesus discípulis  suis : Amen, amen dico vobis, nisi granum fruménti cadens in terram mórtuum fúerit, ipsum solum manet : si autem mórtuum fúerit, multum fructum affert. Qui amat ánimam suam, perdet eam ; et qui odit ánimam suam in hoc mundo, in vitam ætérnam custódit eam. Si quis mihi minístrat, me sequátur, et ubi sum ego, illic et miníster meus erit. Si quis mihi ministráverit, honorificábit eum Pater meus.
\switchcolumn
En ce temps-là, Jésus dit à ses disciples :  En vérité, en vérité je vous le dis : si le grain de froment tombant en terre ne meurt, il demeure seul ; mais s’il meurt, il porte beaucoup de fruit. Celui qui aime sa vie la perdra ; mais celui qui hait sa vie dans ce monde la conserve pour la vie éternelle. Si quelqu’un me sert, qu’il me suive : et là où je serai, là aussi sera mon serviteur. Si quelqu’un me sert, mon Père l’honorera.
\switchcolumn*

Ant. ad Offertorium.\hfill Ps 8, 6-7
\switchcolumn
Antienne d’offertoire.
\switchcolumn*

Glória et honóre coronásti eum : et constituísti eum super ópera mánuum tuárum, Dómine.
\switchcolumn
Vous l’avez couronné de gloire et d’honneur, et vous l’avez placé sur les œuvres de vos mains, Seigneur.
\switchcolumn*

Secreta
\switchcolumn
Secrète
\switchcolumn*

Munéribus nostris, quǽsumus,  Dómine, precibúsque suscéptis : et cæléstibus nos munda mystériis, et cleménter exáudi. Per Dóminum.
\switchcolumn
Nous vous en prions, Seigneur : en ac- ceptant nos offrandes et nos prières, purifiez-nous par les mystères célestes, et exaucez-nous avec bonté. Par Notre-Seigneur.
\switchcolumn*

Ant. ad Comm.\hfill Mt 16, 24
\switchcolumn
Antienne de communion.
\switchcolumn*

Qui vult veníre post me, ábneget semetípsum, et tollat crucem suam, et sequátur me.
\switchcolumn
Celui qui veut venir derrière moi, qu’il renonce à lui-même, qu’il prenne sa croix et qu’il me suive.
\switchcolumn*

Postcommunio
\switchcolumn
Postcommunion
\switchcolumn*

Quǽsumus, omnípotens Deus : ut,  qui cæléstia aliménta percépimus, intercedénte beáto Vincéntio Mártyre tuo, per hæc contra ómnia advérsa muniámur. Per Dóminum.
\switchcolumn
Nous vous en prions, Seigneur, nous qui  avons reçu les aliments célestes : que par eux et par l’intercession du bienheureux Vincent, votre martyr, nous soyons défendus contre toute adversité. Par Notre-Seigneur.
\switchcolumn*

Die 24 januarii
\switchcolumn
Le 24 janvier
\switchcolumn*

S. Francisci Salesii
\switchcolumn
S. François de Sales
\switchcolumn*

Ep., Conf.
\switchcolumn
Év., Conf. et
\switchcolumn*

et Ecclesiæ Doctoris
\switchcolumn
Docteur de l’Église
\switchcolumn*

Ant. ad Introitum.\hfill Si 45, 8 et 9
\switchcolumn
Antienne d’introït.
\switchcolumn*

Státuit ei testaméntum ætérnum, et dedit illi sacerdótium gentis : beatificávit illum in glória, et coronávit eum in vasis virtútis. √~Ps 118, 103. Quam dúlcia fáucibus meis elóquia tua, super mel ori meo ! √~Glória Patri. Státuit ei.
\switchcolumn
Il l’installa par une alliance éternelle, et il  lui a donné le sacerdoce du peuple. Il l’a rendu bienheureux dans la gloire, et il l’a couvert d’un vêtement glorieux. √~Combien douces à mon palais sont tes paroles, plus que le miel à ma bouche ! √~Gloire au Père. Il l’installa.
\switchcolumn*

Oratio
\switchcolumn
Collecte
\switchcolumn*

Deus, qui ad animárum salútem  beátum Francíscum Confessórem tuum atque Pontíficem ómnibus ómnia factum esse voluísti : concéde propítius ; ut caritátis tuæ dulcédine perfúsi, ejus dirigéntibus mónitis ac suffragántibus méritis, ætérna gáudia consequámur. Per Dóminum.
\switchcolumn
Ô Dieu qui, pour le salut des âmes, avez  voulu que le bienheureux François, votre Confesseur et Pontife, se fasse tout à tous, accordez-nous dans votre bonté, que remplis de la douceur de votre amour, nous obtenions, dirigés par ses enseignements et soutenus par ses mérites, les joies éternelles. Par Notre-Seigneur.
\switchcolumn*

Léctio epístolæ beáti Pauli Apóstoli ad Ephésios.
\switchcolumn
Lecture de la lettre du bienheureux apôtre Paul aux Éphésiens.
\switchcolumn*

Ep 3, 7-21
\switchcolumn

\switchcolumn*

Fratres : Factus sum miníster  secúndum donum grátiæ Dei, quæ data est mihi secundum operatiónem virtútis ejus. Mihi ómnium sanctórum mínimo data est grátia hæc, in géntibus evangelizáre investigábiles divítias Christi, et illumináre omnes, quæ sit dispensátio sacraménti abscónditi a sǽculis in Deo, qui ómnia creávit : ut innotéscat principátibus et potestátibus in cæléstibus per Ecclésiam, multifórmis sapiéntia Dei, secúndum præfinitiónem sæculórum, quam fecit in Christo Jesu Dómino nostro : in quo habémus fidúciam, et accéssum in confidéntia per fidem ejus. Propter quod peto ne deficiátis in tribulatiónibus meis pro vobis : quæ est glória vestra. Hujus rei grátia flecto génua mea ad Patrem Dómini nostri Jesu Christi, ex quo omnis patérnitas in cælis et in terra nominátur, ut det vobis secúndum divítias glóriæ suæ, virtúte corroborári per Spíritum ejus in interiórem hóminem, Christum habitáre per fidem in cordibus vestris : in caritáte radicáti, et fundáti, ut possítis comprehéndere cum ómnibus sanctis, quæ sit latitúdo, et longitúdo, et sublímitas, et profúndum : scire étiam supereminéntem sciéntiæ caritátem Christi, ut impleámini in omnem plenitúdinem Dei. Ei autem, qui potens est ómnia fácere superabundánter quam pétimus aut intellígimus, secúndum virtútem, quæ operátur in nobis : ipsi glória in Ecclésia, et in Christo Jesu, in omnes generatiónes sǽculi sæculórum. Amen.
\switchcolumn
Frères, j’ai été fait le ministre de l’Évan- gile par le don de la grâce de Dieu, qui m’a été conférée par l’opération de sa puissance. À moi qui suis le plus petit d’entre tous les saints, a été confiée cette grâce d’annoncer aux païens les richesses incompréhensibles de Jésus-Christ, et d’éclairer tous les hommes, en leur découvrant quelle est l’économie du mystère caché dès le commencement des siècles en Dieu, qui a créé toutes choses : afin que les principautés et les puissances qui sont dans les cieux connaissent par l’Église la sagesse multiforme de Dieu, selon le dessein éternel qu’il a accompli par Jésus-Christ notre Seigneur, en qui nous avons, par la foi en son nom, la liberté de nous approcher de Dieu avec confiance. C’est pourquoi je vous prie de ne point perdre courage en me voyant souffrir tant de maux pour vous, puisque c’est là votre gloire. C’est pour cela que je fléchis les genoux devant le Père de notre Seigneur Jésus-Christ, de qui toute paternité tire son nom au ciel et sur la terre : afin que, selon les richesses de sa gloire, il vous fortifie dans l’homme intérieur par son Saint-Esprit ; que Jésus-Christ habite par la foi dans vos cœurs, et que vous soyez enracinés et fondés dans la charité ; afin que vous puissiez comprendre, avec tous les saints, quelle est la largeur, la longueur, la hauteur et la profondeur, et connaître l’amour de Jésus-Christ qui surpasse toute connaissance ; afin que vous soyez remplis jusqu’à la plénitude de Dieu. À celui qui, par la puissance qui opère en nous, peut faire infiniment plus que tout ce que nous demandons ou pensons ; à lui soit la gloire dans l’Église par Jésus-Christ dans la succession de tous les âges et de tous les siècles. Amen.
\switchcolumn*

Graduale. Si 33, 18-19. Respícite quóniam non mihi soli laborávi, sed ómnibus exquiréntibus disciplínam. √~Audíte me, magnátes, et omnes pópuli, et rectóres ecclésiæ, áuribus percípite.
\switchcolumn
Graduel. Considérez que ce n’est pas pour moi seul que j’ai travaillé, mais aussi pour tous ceux qui recherchent l’instruction. √~Écoutez-moi, grands, et vous, tous les peuples, et tous les chefs de l’Église, écoutez.
\switchcolumn*

Allelúia, allelúia. √~Ps 32, 18. Ecce óculi Dómini super metuéntes eum : et in eis qui sperant super misericórdia ejus. Allelúia.
\switchcolumn
Alléluia, alléluia. √~Voici que les yeux du Seigneur sont sur ceux qui le craignent, et sur ceux qui espèrent en sa miséricorde. Alléluia.
\switchcolumn*

Post Septuagesimam, omissis Alléluia et versu sequenti, dicitur :
\switchcolumn
Après la Septuagésime, on omet l’Alléluia et son verset, et l’on dit le Trait :
\switchcolumn*

Tractus. Ps 33, 18-19. Gustáte, et vidéte quóniam suavis est Dóminus : beátus vir, qui sperat in eo. √~Pr 16, 23. Cor sapiéntis erúdiet os ejus : et lábiis ejus addet grátiam. √~Ibid. 17, 27. Qui moderátur sermónes suos, doctus et prudens est : et pretiósi spíritus vir erudítus.
\switchcolumn
Trait. Goûtez et voyez comme le Seigneur est bon. Heureux l’homme qui espère en lui. √~Le cœur du sage instruira sa bouche, et il donnera de la grâce à ses paroles. √~Celui qui mesure ses paroles est intelligent et prudent, et l’homme à l’esprit retenu est sage.
\switchcolumn*

Sequéntia sancti Evangélii secúndum Matthǽum.
\switchcolumn
Lecture du saint évangile selon saint Matthieu.
\switchcolumn*

Mt 5, 13-19
\switchcolumn

\switchcolumn*

In illo témpore : Dixit Jesus discípulis  suis : Vos estis sal terræ. Quod si sal evanúerit, in quo saliétur ? Ad níhilum valet ultra, nisi ut mittátur foras, et conculcétur ab homínibus. Vos estis lux mundi. Non potest cívitas abscóndi supra montem pósita. Neque accéndunt lucérnam, et ponunt eam sub módio, sed super candelábrum, ut lúceat ómnibus qui in domo sunt. Sic lúceat lux vestra coram homínibus, ut vídeant ópera vestra bona, et gloríficent Patrem vestrum, qui in cælis est. Nolíte putáre, quóniam veni sólvere legem, aut prophétas : non veni sólvere, sed adimplére. Amen quippe dico vobis, donec tránseat cælum et terra, iota unum, aut unus apex non præteríbit a lege, donec ómnia fiant. Qui ergo sólverit unum de mandátis istis mínimis, et docúerit sic hómines, mínimus vocábitur in regno cælórum : qui autem fécerit, et docúerit, hic magnus vocábitur in regno cælórum.
\switchcolumn
En ce temps-là, Jésus dit à ses disciples :  Vous êtes le sel de la terre. Si le sel s’affadit, avec quoi le salera-t-on ? Il n’est plus bon à rien, sinon à être jeté dehors et piétiné par les hommes. Vous êtes la lumière du monde. Une ville placée sur une montagne ne peut être cachée. Et quand on allume une lampe, ce n’est pas pour la placer sous le boisseau mais sur le candélabre, afin qu’elle brille pour tous ceux qui sont dans la maison. Qu’ainsi votre lumière brille devant les hommes, afin qu’ils voient vos bonnes œuvres et qu’ils glorifient votre Père qui est dans les cieux. Ne pensez pas que je sois venu abolir la loi ou les prophètes. Je ne suis pas venu abolir, mais accomplir. Amen je vous le dis : Jusqu’à ce que passent le ciel et la terre, pas un iota ou un trait de la loi ne passera : tout sera accompli. Celui donc qui abolira l’un de ces plus petits commandements et enseignera aux hommes à faire ainsi, sera appelé le plus petit dans le royaume des cieux. Mais celui qui les accomplira et les enseignera, celui-là sera appelé grand dans le royaume des cieux.
\switchcolumn*

Ant. ad Offertorium.\hfill Ap 2, 19
\switchcolumn
Antienne d’offertoire.
\switchcolumn*

Novi ópera tua, et fidem, et caritátem tuam, et ministérium, et patiéntiam tuam, et ópera tua novíssima plura prióribus.
\switchcolumn
Je connais tes œuvres, ta foi, ta charité, ton service, ta patience et tes dernières œuvres plus nombreuses que les premières.
\switchcolumn*

Secreta
\switchcolumn
Secrète
\switchcolumn*

Per hanc salutárem hóstiam,  quam offérimus tibi, Dómine, divíno illo Sancti Spiritus igne cor nostrum accénde : quo mitíssimum beáti Francísci ánimum mirabíliter inflammásti. Per Dóminum… in unitáte ejúsdem Spíritus Sancti.
\switchcolumn
Par cette hostie salutaire que nous vous  offrons, Seigneur, brûlez nos cœurs par ce même feu divin du Saint-Esprit, par lequel vous avez enflammé de manière admirable l’âme du très doux François. Par Notre-Seigneur… dans l’unité du même Saint-Esprit.
\switchcolumn*

Ant. ad Comm.\hfill 1 Co 9, 22
\switchcolumn
Antienne de communion.
\switchcolumn*

Factus sum infírmis infírmus, ut infírmos lucrifácererm : ómnibus ómnia factus sum, ut omnes fácerem salvos.
\switchcolumn
Je me suis fait infirme avec les infirmes, afin de gagner les infirmes ; je me suis fait tout à tous, afin de les sauver tous.
\switchcolumn*

Postcommunio
\switchcolumn
Postcommunion
\switchcolumn*

Concéde, quǽsumus, omnípotens  Deus : ut, per sacraménta quæ súmpsimus, beáti Francísci caritátem et mansuetúdinem imitántes in terris, glóriam quoque consequámur in cælis. Per Dóminum.
\switchcolumn
Accordez-nous, nous vous en prions,  Dieu tout-puissant, que par les sacrements que nous avons reçus, imitant sur la terre la charité et la douceur du bienheureux François, nous obtenions également la gloire dans le ciel. Par Notre-Seigneur.
\switchcolumn*

Die 10 februarii
\switchcolumn
Le 10 février
\switchcolumn*

S. Scholasticæ
\switchcolumn
S. Scholastique
\switchcolumn*

Virg., Sororis
\switchcolumn
Vierge, Sœur de
\switchcolumn*

S. P. N. Benedicti
\switchcolumn
N. P. S. Benoît
\switchcolumn*

Ant. ad Introitum.\hfill Ct 2, 10.11
\switchcolumn
Antienne d’introït.
\switchcolumn*

Surge, própera, amíca mea, colúmba  mea, et veni ; jam enim hiems tránsiit, imber ábiit et recéssit. √~Ps 54, 7. Quis dabit mihi pennas sicut colúmbæ ? et volábo et requiéscam. √~Glória Patri. Surge, própera.
\switchcolumn
Debout, hâte-toi, mon amie, ma co- lombe, et viens ; car l’hiver est désormais passé, les pluies s’en sont allées et se sont éloignées. √~Qui me donnera des ailes comme celles de la colombe ? Alors je volerai et j’irai me reposer. √~Gloire au Père. Debout, hâte-toi.
\switchcolumn*

Oratio
\switchcolumn
Collecte
\switchcolumn*

Deus, qui ánimam beátæ Vírginis  tuæ Scholásticæ, ad ostendéndam innocéntiæ viam, in colúmbæ spécie cœlum penetráre fecísti : da nobis ejus méritis et précibus ita innocénter vívere, ut ad ætérna mereámur gáudia perveníre. Per Dóminum.
\switchcolumn
Ô Dieu qui, pour nous montrer l’inno cence de sa vie, avez fait entrer aux cieux l’âme de la bienheureuse Vierge Scholastique sous l’apparence d’une colombe, donnez-nous  par ses mérites et ses prières de vivre d’une si manière si sainte que nous méritions de parvenir aux joies éternelles. Par Notre-Seigneur.
\switchcolumn*

Léctio libri Sapiéntiæ.
\switchcolumn
Lecture du livre de la Sagesse.
\switchcolumn*

Ct 8, 1-4
\switchcolumn

\switchcolumn*

Quis mihi det te fratrem meum  sugéntem úbera matris meæ, ut invéniam te foris, et deósculer te, et jam me nemo despíciat ? Apprehéndam te, et ducam in domum matris meæ : ibi me docébis, et dabo tibi póculum ex vino condíto et mustum malórum granatórum meórum. Læva ejus sub cápite meo, et déxtera illíus amplexábitur me. Adjúro vos fíliæ Jerúsalem, ne suscitétis, neque evigiláre faciátis diléctam, donec ipsa velit.
\switchcolumn
Qui fera de toi mon frère, tétant les seins  de ma mère, afin que je te trouve au dehors, que je t’embrasse, et que désormais nul ne me méprise ? Je te prendrai et je te conduirai à la maison de ma mère. Là tu m’enseigneras et je te donnerai une coupe de vin aromatisé, et le suc nouveau de mes grenades. Sa main gauche est sous ma tête, et sa droite m’entoure. Je vous adjure, filles de Jérusalem, de ne point réveiller, ni de faire s’éveiller la bien-aimée, jusqu’à ce qu’elle-même le veuille.
\switchcolumn*

Graduale. Ct 8, 7-8. Aquæ multæ non potuérunt exstínguere caritátem, neque flúmina óbruent illam. √~Si déderit homo omnem substántiam domus suæ pro dilectióne, quasi nihil despíciet eam.
\switchcolumn
Graduel. Les eaux nombreuses n’ont pu éteindre l’amour, et les fleuves ne la recouvriront point. √~Si quelqu’un donnait toute la richesse de sa maison pour acheter l’amour, il la mépriserait comme n’étant rien.
\switchcolumn*

Allelúia, allelúia. √~Ct 2, 16-17. Diléctus meus mihi et ego illi, qui páscitur inter lília, donec aspíret dies et inclinéntur umbræ. Allelúia.
\switchcolumn
Alléluia, alléluia. √~Mon Bien-aimé est à moi et je suis à lui. Il se repaît parmi les lys, jusqu’à ce que le jour décline et que s’inclinent les ombres. Alléluia.
\switchcolumn*

Post Septuagesimam, omissis Alléluia et versu sequenti, dicitur :
\switchcolumn
Après la Septuagésime, on omet l’Alléluia et son verset, et l’on dit le Trait :
\switchcolumn*

Tractus. Ct 3, 4. Invéni quem díligit ánima mea, ténui eum, nec dimíttam. √~Ct 1, 2-3. Oleum effúsum nomen tuum : ideo adolescéntulæ dilexérunt te. √~Trahe me, post te currémus in odórem unguentórum tuórum. √~Exsultábimus et lætábimur in te, mémores úberum tuórum super vinum ; recti díligunt te.
\switchcolumn
Trait. J’ai trouvé celui qu’aime mon âme, je le tiens et je ne le lâcherai point. √~Ton nom est une huile répandue, c’est pourquoi les jeunes filles t’aiment. √~Attire-moi, derrière toi nous courrons à l’odeur de tes parfums. √~Nous exulterons et nous nous réjouirons en toi, nous souvenant de tes seins plus que du vin. Les cœurs droits t’aiment.
\switchcolumn*

Sequentia
\switchcolumn
Séquence
\switchcolumn*

Emicat merídies,
\switchcolumn
Le midi brille avec splendeur,
\switchcolumn*

Et beáta réquies
\switchcolumn
Ainsi que le repos bienheureux
\switchcolumn*

Vírgini Scholásticæ
\switchcolumn
Pour la Vierge Scholastique.
\switchcolumn*

Intrat in cubícula ;
\switchcolumn
Elle entre dans la chambre nuptiale,
\switchcolumn*

Sponsi petit óscula,
\switchcolumn
Elle réclame les baisers de l’Époux,
\switchcolumn*

Quem amávit únice.
\switchcolumn
Qu’elle aima uniquement.
\switchcolumn*

Quantis cum gemítibus
\switchcolumn
Par combien de gémissements
\switchcolumn*

Cordis et ardóribus
\switchcolumn
Et de brûlant désirs de son cœur
\switchcolumn*

Hæc diléctum quǽsiit.
\switchcolumn
N’a-t-elle pas cherché son Bien-aimé !
\switchcolumn*

Movit cœlos lácrimis,
\switchcolumn
Elle ébranle les cieux par ses larmes,
\switchcolumn*

Imbribúsque plúrimis
\switchcolumn
Et ramollit le cœur de son frère
\switchcolumn*

Pectus fratris mólliit.
\switchcolumn
Par une pluie abondante.
\switchcolumn*

O grata collóquia,
\switchcolumn
Ô les heureux entretiens
\switchcolumn*

Cum cœlórum gáudia
\switchcolumn
Lorsque Benoît explique
\switchcolumn*

Benedíctus éxplicat !
\switchcolumn
Les joies du ciel !
\switchcolumn*

Ardent desidéria,
\switchcolumn
Son désir brûle,
\switchcolumn*

Mentis et suspíria
\switchcolumn
Et l’Époux Vierge réveille en elle
\switchcolumn*

Virgo sponsus éxcitat.
\switchcolumn
Les soupirs de son esprit.
\switchcolumn*

Veni formosíssima,
\switchcolumn
Viens ma toute belle,
\switchcolumn*

Sponsa dilectíssima,
\switchcolumn
Mon épouse bien aimée,
\switchcolumn*

Veni, coronáberis.
\switchcolumn
Viens, tu seras couronnée.
\switchcolumn*

Dórmies in líliis,
\switchcolumn
Tu reposeras parmi les lys,
\switchcolumn*

Afflues delíciis,
\switchcolumn
Un torrent de délices te remplira,
\switchcolumn*

Et inebriáberis.
\switchcolumn
Et tu seras enivrée.
\switchcolumn*

O colúmba vírginum,
\switchcolumn
Ô colombe parmi les vierges,
\switchcolumn*

Quæ de ripis flúminum
\switchcolumn
Toi qui des rives du fleuve
\switchcolumn*

Adis aulam glóriæ ;
\switchcolumn
Gagnes le séjour de la gloire,
\switchcolumn*

Trahe nos odóribus,
\switchcolumn
Attire-nous par tes parfums,
\switchcolumn*

Pasce et ubéribus
\switchcolumn
Nourris-nous du lait
\switchcolumn*

Immortális grátiæ. Amen
\switchcolumn
De la grâce immortelle. Amen
\switchcolumn*

Sequéntia sancti Evangélii secúndum Matthǽum.
\switchcolumn
Lecture du saint évangile selon saint Matthieu.
\switchcolumn*

Mt 25, 1-13
\switchcolumn

\switchcolumn*

In illo témpore : Dixit Jesus discípulis  suis parábolam hanc : Símile erit regnum cælórum decem virgínibus : quæ accipiéntes lámpades suas exiérunt óbviam sponso et sponsæ. Quinque autem ex eis erant fátuæ, et quinque prudéntes : sed quinque fátuæ, accéptis lampádibus, non sumpsérunt oleum secum : prudéntes vero accepérunt oleum in vasis suis cum lampádibus. Moram autem faciénte sponso, dormitavérunt omnes et dormiérunt. Média autem nocte clamor factus est : Ecce sponsus venit, exíte óbviam ei. Tunc surrexérunt omnes vírgines illæ, et ornavérunt lámpades suas. Fátuæ autem sapiéntibus dixérunt : Date nobis de óleo vestro, quia lampádes nostræ extinguúntur. Respondérunt prudéntes, dicéntes : Ne forte non suffíciat nobis, et vobis, ite pótius ad vendéntes, et émite vobis. Dum autem irent émere, venit sponsus : et quæ parátæ erant, intravérunt cum eo ad núptias, et clausa est janua. Novíssime vero véniunt et relíquæ vírgines, dicéntes : Dómine, Dómine, áperi nobis. At ille respóndens, ait : Amen dico vobis, néscio vos. Vigiláte ítaque, quia néscitis diem, neque horam.
\switchcolumn
En ce temps-là, Jésus dit à ses disciples :  Le royaume des cieux sera semblable à dix vierges, qui ayant pris leurs lampes, s’en allèrent au-devant de l’époux et de l’épouse. Cinq d’entre elles étaient folles, et cinq étaient sages. Les cinq qui étaient folles, ayant pris leurs lampes, ne prirent point d’huile avec elles. Les sages, au contraire, prirent de l’huile dans leurs vases avec leurs lampes. Mais l’époux tardant à venir, elles s’assoupirent toutes, et s’endormirent. Or, au milieu de la nuit, on entendit un grand cri : Voici l’époux qui vient ; allez au-devant de lui. Aussitôt toutes ces vierges se levèrent et préparèrent leurs lampes. Mais les folles dirent aux sages : Donnez-nous de votre huile, parce que nos lampes s’éteignent. Les sages leur répondirent : De peur que ce que nous avons ne suffise pas pour nous et pour vous, allez plutôt chez les marchands et achetez-vous-en. Mais pendant qu’elles allaient en acheter, l’époux vint, et celles qui étaient prêtes entrèrent avec lui aux noces, et la porte fut fermée. Enfin les autres vierges vinrent aussi, et lui dirent : Seigneur, seigneur, ouvrez-nous ! Mais il leur répondit : Je vous le dis en vérité, je ne vous connais point. Veillez donc, parce que vous ne savez ni le jour ni l’heure.
\switchcolumn*

Ant. ad Offertorium.\hfill Ps 17, 14-16
\switchcolumn
Antienne d’offertoire.
\switchcolumn*

Intónuit Dóminus de cœlo ; et Altíssimus dedit vocem suam : grando et carbónes ignis : fúlgura multiplicávit, et conturbávit eos, et apparuérunt fontes aquárum.
\switchcolumn
Le Seigneur a fait éclater son tonnerre du haut des cieux, et le Très-Haut a fait retentir sa voix, de la grêle et des charbons ardents ; il a multiplié les éclairs et a jeté le trouble parmi [mes ennemis] ; et des sources d’eau sont apparues.
\switchcolumn*

Secreta
\switchcolumn
Secrète
\switchcolumn*

Súscipe, quǽsumus, Dómine,  desidéria supplicántium cum oblatiónibus hostiárum, ut interveniénte beáta Scholástica Vírgine tua, quæ te sincéro diléxit amóre, fides in nobis et cáritas augeátur. Per Dóminum.
\switchcolumn
Recevez, nous vous en prions, Seigneur,  les désirs de ceux qui vous supplient avec l’offrande de ces hosties, afin que, par l’intercession de la bienheureuse Scholastique votre Vierge, qui vous aima d’un amour sincère, la foi et la charité croissent en nous. Par Notre-Seigneur.
\switchcolumn*

Ant. ad Comm.\hfill Jo 15, 7
\switchcolumn
Antienne de communion.
\switchcolumn*

Si manséritis in me, et verba mea in vobis mánserint, quodcúmque voluéritis, petétis, et fiet vobis, dicit Dóminus.
\switchcolumn
Si vous demeurez en moi, et si mes paroles demeurent en vous, demandez tout ce que vous voulez, et cela arrivera pour vous, dit le Seigneur.
\switchcolumn*

Postcommunio
\switchcolumn
Postcommunion
\switchcolumn*

Famíliam tuam, quǽsumus, Dómine, spiritáli cibo satiátam, beátæ Vírginis tuæ Scholásticæ méritis propítius réspice : ut, sicut ipsíus précibus, ad obtinéndum quod optábat, imbrem cǽlitus descéndere fecísti ; ita ejus supplicatiónibus ariditátem cordis nostri supérnæ dignéris grátiæ rore perfúndere. Per Dóminum.
\switchcolumn
Nous vous en prions par les mérites de votre bienheureuse Vierge Scholastique, Seigneur : regardez avec bonté votre famille, rassasiée par cette nourriture spirituelle, afin que, de même qu’à sa prière vous avez fait descendre des cieux la pluie qu’elle demandait, ainsi par ses supplications vous daigniez arroser l’aridité de notre cœur par la rosée de la grâce d’en-haut. Par Notre-Seigneur.
\switchcolumn*

Die 21 Martii
\switchcolumn
Le 21 mars
\switchcolumn*

In transitu
\switchcolumn
Trépas de
\switchcolumn*

S. P. N. Benedicti
\switchcolumn
N. P. S. Benoît
\switchcolumn*

Ant. ad Introitum.
\switchcolumn
Antienne d’introït.
\switchcolumn*

Gaudeámus omnes in Dómino,  diem festum celebrántes sub honóre sancti Benedícti Abbátis : de cujus solemnitáte gaudent Angeli et colláudant Fílium Dei. √~Ps 47, 2. Magnus Dóminus et laudábilis nimis : in civitáte Dei nostri, in monte sancto ejus. √~Glória Patri. Gaudeámus.
\switchcolumn
Réjouissons-nous tous dans le Seigneur,  en célébrant ce jour de fête en l’honneur du saint abbé Benoît. Les anges se réjouissent de cette solennité, et ils louent le Fils de Dieu. √~Grand est le Seigneur et très digne de louanges, dans la cité de notre Dieu, sur sa montagne sainte. √~Gloire au Père. Réjouissons-nous.
\switchcolumn*

Oratio
\switchcolumn
Collecte
\switchcolumn*

Omnípotens sempitérne Deus,  qui hodiérna die carnis edúctum ergástulo sanctíssimum Confessórem tuum Benedíctum sublevásti ad cœlum : concéde, quǽsumus, hæc festa tuis fámulis celebrántibus cunctórum véniam delictórum, ut, qui exsultántibus ánimis ejus claritáti congáudent, ipso apud te interveniénte consociéntur et méritis. Per Dóminum.
\switchcolumn
Dieu tout-puissant et éternel, qui au- jourd’hui avez tiré votre très saint Confesseur Benoît de la prison de la chair pour l’élever au ciel, accordez à vos serviteurs qui célèbrent ce jour de fête, nous vous en prions, le pardon de tous leurs péchés ; que ceux qui d’un cœur plein de joie s’associent à la gloire de saint Benoît soient associés également, par son intercession, à ses mérites. Par Notre-Seigneur.
\switchcolumn*

Léctio libri Sapiéntiæ.
\switchcolumn
Lecture du livre de la Sagesse.
\switchcolumn*

Sir 50, 1-14
\switchcolumn

\switchcolumn*

Ecce sacérdos magnus, qui in vita  sua suffúlsit domum, et in diébus suis corroborávit templum. Templi étiam altitúdo ab ipso fundáta est, duplex ædificátio, et excélsi paríetes templi. In diébus ipsíus emanavérunt pútei aquárum, et quasi mare adimpléti sunt supra modum. Qui curávit gentem suam, et liberávit eam a perditióne. Qui præváluit amplificáre civitátem, qui adéptus est glóriam in conversatióne gentis : et ingréssum domus et átrii amplificávit. Quasi stella matutína in médio nébulæ, et quasi luna plena, in diébus suis lucet ; et quasi sol refúlgens, sic ille effúlsit in templo Dei. Quasi arcus refúlgens inter nébulas glóriæ, et quasi flos rosárum in diébus vernis, et quasi lília quæ sunt in tránsitu aquæ, et quasi thus rédolens in diébus æstátis ; quasi ignis effúlgens et thus ardens in igne ; quasi vas auri sólidum ornátum omni lápide pretióso ; quasi olíva púllulans, et cypréssus in altitúdinem se extóllens. Circa illum coróna fratrum : quasi plantátio cedri in monte Líbano, sic circa illum stetérunt quasi rami palmæ ; et omnes fílii Aaron in glória sua.
\switchcolumn
Voici le grand prêtre [Simon, fils d’Onias] qui a soutenu la maison et fortifié le temple pendant sa vie. C’est par lui que fut fondée la hauteur double, le haut contrefort de l’enceinte du Temple. En son temps  les puits des eaux se remplirent comme la mer, au delà de toute mesure. Il guérit son peuple et le libéra de la perdition. Il réussit à agrandir la cité, il a obtenu la gloire avec son peuple autour de lui, et il a agrandi l’entrée du temple et du parvis. Il brille comme l’étoile du matin au milieu de la nuée, et comme la lune en son plein. Comme le soleil resplendissant, ainsi a-t-il brillé dans le temple de Dieu, comme l’arc-en-ciel brillant parmi les nuées glorieuses, comme la rose au printemps, comme les lis au bord des eaux, comme l’encens qui répand son parfum en été, comme le feu brillant et l’encens brûlant dans le feu, comme un solide vase d’or orné de toutes les pierres précieuses, comme l’olivier qui étend ses pousses et le cyprès qui s’élève ; ainsi était-il lorsqu’il prenait le vêtement de gloire et qu’il revêtait ses superbes ornements ; en montant l’autel saint, il remplissait de gloire l’enceinte du sanctuaire, lorsqu’il recevait les portions des sacrifices de la main des prêtres, et qu’il se tenait près de l’autel, avec autour de lui une couronne de frères. Comme une plantation de cèdres sur le Liban, ainsi se tenaient autour de lui comme des branches de palmier tous les fils d’Aaron dans leur gloire.
\switchcolumn*

Graduale. Ps 20, 4-5. Dómine, prævenísti eum in benedictiónibus dulcédinis : posuísti in cápite ejus corónam de lápide pretióso. √~Vitam pétiit a te, et tribuísti ei longitúdinem diérum in sǽculum sǽculi.
\switchcolumn
Graduel. Seigneur, vous l’avez prévenu de vos plus douces bénédictions ; vous avez placé sur sa tête une couronne de pierres précieuses. √~Il vous a demandé la vie, et vous lui avez accordé la longueur des jours pour les siècles des siècles.
\switchcolumn*

Tractus. Ps 111, 1-3. Beátus vir, qui timet Dóminum : in mandátis ejus cupit nimis. √~Potens in terra erit semen ejus : generátio rectórum benedicétur. √~Glória et divítiæ in domo ejus : et justítia ejus manet in sǽculum sǽculi.
\switchcolumn
Trait. Heureux l’homme qui craint le Seigneur, qui met tout son désir dans l’accomplissement de ses commandements. √~Puissante sur la terre sera sa descendance ; la générationt des justes sera bénie. √~Gloire et richesses dans sa maison ; sa justice demeure pour les siècles des siècles.
\switchcolumn*

Sequentia
\switchcolumn
Séquence
\switchcolumn*

Læta quies magni Ducis,Dona ferens novæ lucis,Hódie recólitur.
\switchcolumn
Nous célébrons aujourd’hui l’heureux jour où le grand Patriarche part pour son repos, jour qui nous apporte les dons d’une lumière nouvelle.
\switchcolumn*

Charis datur piæ menti,Corde sonet in ardéntiQuidquid foris prómitur.
\switchcolumn
La grâce est donnée à l’âme pieuse ; que retentisse dans un cœur brûlant tout ce qui est chanté extérieurement.
\switchcolumn*

Hunc per callem OriéntisAdmirémur ascendéntisPatriárchæ spéciem.
\switchcolumn
Admirons l’apparition du Patriarche, montant par le chemin de l’Orient.
\switchcolumn*

Amplum semen magnæ prolisIllum fecit instar solis,Abrahæ persímilem.
\switchcolumn
L’ample descendance de sa grande famille le rend égal au soleil, en tout semblable à Abraham.
\switchcolumn*

Corvum cernis ministrántem,Hinc Elíam latitántemSpecu nosce párvulo.
\switchcolumn
Tu vois le corbeau qui le sert ; reconnais donc en lui Élie qui se cache dans une petite grotte.
\switchcolumn*

Elíseus dignoscátur,Cum secúris revocáturDe torréntis álveo.
\switchcolumn
On reconnaît Élisée, lorsque la hache est rappelée du lit du torrent.
\switchcolumn*

Illum Joseph candor morum,Illum Jacob futurórumMens effécit cónscia.
\switchcolumn
La pureté de ses mœurs a fait de lui Joseph, et son esprit connaissant l’avenir l’identifie à Jacob.
\switchcolumn*

Ipse memor suæ gentis,Nos perdúcat in manéntisSemper Christi gáudia. Amen.
\switchcolumn
Se souvenant de sa famille, qu’il nous conduise vers les joies du Christ qui demeure à jamais. Amen.
\switchcolumn*

Sequéntia sancti Evangélii secúndum Matthǽum.
\switchcolumn
Lecture du saint évangile selon saint Matthieu.
\switchcolumn*

Mt 19, 27-29
\switchcolumn

\switchcolumn*

In illo témpore : Dixit Petrus ad  Jesum : Ecce nos relíquimus ómnia, et secúti sumus te ; quid ergo erit nobis ? Jesus autem dixit illis : Amen dico vobis, quod vos, qui secúti estis me, in regeneratióne, cum séderit Fílius hóminis in sede majestátis suæ, sedébitis et vos super sedes duódecim, judicántes duódecim tribus Israel. Et omnis, qui relíquerit domum, vel fratres, aut soróres, aut patrem, aut matrem, aut uxórem, aut fílios, aut agros propter nomen meum, céntuplum accípiet, et vitam ætérnam possidébit.
\switchcolumn
En ce temps-là, Pierre dit à Jésus : Voi- ci que nous avons tout quitté et nous t’avons suivi. Qu’y aura-t-il donc pour nous ? Jésus leur dit : Amen je vous le dis : vous qui m’avez suivi, lorsque le Fils de l’homme, au jour de la régénération, trônera sur le trône de sa majesté, vous trônerez vous aussi sur douze trônes, jugeant les douze tribus d’Israël. Et quiconque aura laissé une maison, des frères, des sœurs, un père, une mère, une femme, des fils ou des champs pour mon nom, il recevra le centuple, et il possèdera la vie éternelle.
\switchcolumn*

Ant. ad Offertorium.\hfill Ps 20, 3-4
\switchcolumn
Antienne d’offertoire.
\switchcolumn*

Desidérium ánimæ ejus tribuísti ei, Dómine, et voluntáte labiórum ejus non fraudásti eum : posuísti in cápite ejus corónam de lápide pretióso.
\switchcolumn
Vous lui avez accordé, Seigneur, le désir de son âme, et vous ne l’avez pas privé de ce que ses lèvres ont désiré ; vous avez placé sur sa tête une couronne de pierres précieuses.
\switchcolumn*

Secreta
\switchcolumn
Secrète
\switchcolumn*

Oblátis, Dómine, ad honórem  sanctíssimi Confessóris tui Benedícti placáre munéribus : et ipsíus intervéntu fámulis tuis tríbue indulgéntiam peccatórum. Per Dóminum.
\switchcolumn
Regardez avec bonté, Seigneur, les offrandes que nous vous présentons en l’honneur de votre très saint Confesseur Benoît, et par son intercession, accordez à vos serviteurs le pardon de leurs péchés. Par Notre-Seigneur.
\switchcolumn*

Præfatio
\switchcolumn
Préface
\switchcolumn*

Vere dignum et justum est, æquum et salutáre, nos tibi semper et ubíque grátias ágere : Dómine sancte, Pater omnípotens, ætérne Deus. Qui beatíssimum Confessórem tuum Benedíctum, Ducem et Magístrum cǽlitus edóctum, innumerábili multitúdini filiórum statuísti. Quem et ómnium justórum spíritu replétum, et extra se raptum, lúminis tuæ splendóre collustrásti. Ut in ipsa luce visiónis intímæ, mentis laxáto sinu, quam angústa essent ómnia inferióra deprehénderet. Per Christum Dóminum nostrum. Quaprópter profúsis gáudiis, totus in orbe terrárum monachórum cœtus exsúltat. Sed et supérnæ virtútes, atque angélicæ potestátes, hymnum glóriæ tuæ cóncinunt, sine fine dicéntes. Sanctus.
\switchcolumn
Il est vraiment digne et juste, équitable et salutaire, de toujours et partout vous rendre grâces, Seigneur saint, Père tout-puissant, Dieu éternel, qui avez établi votre bienheureux Confesseur Benoît comme chef et maître d’une multitude innombrable de fils, en l’instruisant d’en-haut. Vous l’avez rempli de l’esprit de tous les justes, et l’emportant hors de lui-même, vous l’avez illuminé de la splendeur de votre lumière. L’esprit dilaté, il comprit dans la lumière de cette intime vision combien petites sont toutes les choses d’ici-bas. Par le Christ Notre-Seigneur. C’est pourquoi dans cette effusion de joie, l’assemblée des moines exulte par toute la terre. De même les Vertus d’en-haut et les Puissances angéliques chantent l’hymne de votre gloire, disant sans fin : Saint.
\switchcolumn*

Ant. ad Comm.\hfill Lc 12, 42
\switchcolumn
Antienne de communion.
\switchcolumn*

Fidélis servus, et prudens, quem constítuit Dóminus super famíliam suam : ut det illis in témpore trítici mensúram.
\switchcolumn
Voici le serviteur fidèle et sage que le Seigneur a placé sur sa famille, pour qu’il leur donne leur mesure de blé en temps opportun.
\switchcolumn*

Postcommunio
\switchcolumn
Postcommunion
\switchcolumn*

Percéptis, Dómine Deus noster,  salutáribus sacraméntis, humíliter deprecámur : ut intercedénte sanctíssimo Benedícto Confessóre tuo, quæ pro illíus veneránda gérimus solemnitáte, nobis profíciant ad salútem. Per Dóminum.
\switchcolumn
Après avoir reçu, Seigneur notre Dieu,  les sacrements de notre salut, nous vous prions humblement : que par l’intercession de votre très saint Confesseur Benoît, l’action que nous accomplissons pour célébrer sa solennité serve à notre salut. Par Notre-Seigneur.
\switchcolumn*

Die 21 Aprilis
\switchcolumn
Le 21 avril
\switchcolumn*

S. Anselmi
\switchcolumn
S. Anselme
\switchcolumn*

Ep., Conf.
\switchcolumn
Év., Conf.
\switchcolumn*

et Eccl. Doctoris
\switchcolumn
et Docteur de l’Église
\switchcolumn*

Ant. ad Introitum.\hfill Si 15, 5
\switchcolumn
Antienne d’introït.
\switchcolumn*

In médio Ecclésiæ apéruit os ejus :  et implévit eum Dóminus spíritu sapiéntiæ et intellectus : stolam glóriæ índuit eum, allelúia, allelúia. √~Ps 91, 2. Bonum est confitéri Dómino : et psállere nómini tuo, Altíssime. √~Glória Patri. In médio.
\switchcolumn
Au milieu de l’Église, le Seigneur a ouvert sa bouche et il l’a rempli de l’esprit de sagesse et d’intelligence ; il l’a revêtu du vêtement de gloire. √~Il est bon de célébrer le Seigneur, et de psalmodier pour votre Nom, ô Très-Haut. √~Gloire au Père. Au milieu.
\switchcolumn*

Oratio
\switchcolumn
Collecte
\switchcolumn*

Ecclésiam tuam, quǽsumus,  Dómine, benígnus illústra : ut beáti Ansélmi Confessóris tui atque Pontíficis illumináta doctrínis, ad dona pervéniat sempitérna. Per Dóminum.
\switchcolumn
Éclairez votre Église, nous vous en  prions, Seigneur, afin qu’éclairée par la doctrine du bienheureux Anselme, votre Confesseur et Pontife, elle parvienne aux dons éternels. Par Notre-Seigneur.
\switchcolumn*

Léctio epístolæ beáti Pauli Apóstoli ad Timótheum.
\switchcolumn
Lecture de la lettre du bienheureux Apôtre Paul à Timothée.
\switchcolumn*

2 Tm 4, 1-8
\switchcolumn

\switchcolumn*

(Alia Epistola ad lib., vide infra)
\switchcolumn
(Autre épître au choix : voir plus bas)
\switchcolumn*

Testíficor coram Deo et Jesu  Christo, qui judicatúrus est vivos et mórtuos, per advéntum ipsíus et regnum ejus : prǽdica verbum, insta opportúne, importúne : árgue, óbsecra, íncrepa in omni patiéntia et doctrína. Erit enim tempus, cum sanam doctrínam non sustinébunt, sed ad sua desidéria coacervábunt sibi magístros, pruriéntes áuribus, et a veritáte quidem audítum avértent, ad fábulas autem converténtur. Tu vero vígila, in ómnibus labóra, opus fac evangelístæ, ministérium tuum imple. Sóbrius esto. Ego enim jam delíbor, et tempus resolutiónis meæ instat. Bonum certámen certávi, cursum consummávi, fidem servávi. In réliquo repósita est mihi coróna justítiæ, quam reddet mihi Dóminus in illa die, justus judex : non solum autem mihi, sed et iis, qui díligunt advéntum ejus.
\switchcolumn
Je te conjure donc devant Dieu, et devant  Jésus-Christ, qui jugera les vivants et les morts, par son avènement glorieux et par son royaume, d’annoncer la parole. Insiste à temps et à contre-temps ; reprends, supplie, menace, sans te lasser jamais de les tolérer et de les instruire. Car il viendra un temps où les hommes ne pourront plus souffrir la saine doctrine. Au contraire, ayant une extrême démangeaison d’entendre ce qui les flatte, ils auront recours à une foule de docteurs propres à satisfaire leurs désirs ; et fermant l’oreille à la vérité, ils l’ouvriront à des fables. Mais toi, veille continuellement ; travaille en toute chose ; fais œuvre d’évangéliste ; remplis tous les devoirs de ton ministère ; sois sobre. Car pour moi, je suis comme une victime qui a déjà reçu l’aspersion pour être sacrifiée ; et le temps de ma délivrance s’approche. J’ai bien combattu ; j’ai achevé ma course ; j’ai gardé la foi. Il ne me reste qu’à attendre la couronne de justice qui m’est réservée, que le Seigneur comme un juste juge me rendra en ce grand jour, et non-seulement à moi, mais encore à tous ceux qui aiment son avènement.
\switchcolumn*

Allelúia, allelúia. √~Si 45, 9. Amávit eum Dóminus, et ornávit eum : stolam glóriæ induit eum.
\switchcolumn
Alléluia, alléluia. √~Le Seigneur l’a aimé et l’a orné : il l’a revêtu du vêtement de gloire.
\switchcolumn*

Allelúia. √~Os 14, 6. Justus germinábit sicut lilium : et florébit in ætérnum ante Dóminum.
\switchcolumn
Alléluia. √~Le juste germera comme le lys, et il fleurira à jamais devant le Seigneur.
\switchcolumn*

Sequéntia sancti Evangélii secúndum Matthǽum.
\switchcolumn
Lecture du saint évangile selon saint Matthieu.
\switchcolumn*

Mt 5, 13-19
\switchcolumn

\switchcolumn*

In illo témpore : Dixit Jesus discípulis  suis : Vos estis sal terræ. Quod si sal evanúerit, in quo saliétur ? Ad níhilum valet ultra, nisi ut mittátur foras, et conculcétur ab homínibus. Vos estis lux mundi. Non potest cívitas abscóndi supra montem pósita. Neque accéndunt lucérnam, et ponunt eam sub módio, sed super candelábrum, ut lúceat ómnibus qui in domo sunt. Sic lúceat lux vestra coram homínibus, ut vídeant ópera vestra bona, et gloríficent Patrem vestrum, qui in cælis est. Nolite putáre, quóniam veni sólvere legem, aut prophétas : non veni sólvere, sed adimplére. Amen quippe dico vobis, donec tránseat cælum et terra, iota unum, aut unus apex non præteríbit a lege, donec ómnia fiant. Qui ergo sólverit unum de mandátis istis mínimis, et docúerit sic hómines, mínimus vocábitur in regno cælórum : qui autem fécerit, et docúerit, hic magnus vocábitur in regno cælórum.
\switchcolumn
En ce temps-là, Jésus dit à ses disciples :  Vous êtes le sel de la terre. Si le sel s’affadit, avec quoi le salera-t-on ? Il n’est plus bon à rien, sinon à être jeté dehors et piétiné par les hommes. Vous êtes la lumière du monde. Une ville placée sur une montagne ne peut être cachée Et quand on allume une lampe, ce n’est pas pour la placer sous le boisseau mais sur le candélabre, afin qu’elle brille pour tous ceux qui sont dans la maison. Qu’ainsi votre lumière brille devant les hommes, afin qu’ils voient vos bonnes œuvres et qu’ils glorifient votre Père qui est dans les cieux. Ne pensez pas que je sois venu abolir la loi ou les prophètes. Je ne suis pas venu abolir, mais accomplir. Amen je vous le dis : Jusqu’à ce que passent le ciel et la terre, pas un iota ou un trait de la loi ne passera : tout sera accompli. Celui donc qui abolira l’un de ces plus petits commandements et enseignera aux hommes à faire ainsi, sera appelé le plus petit dans le royaume des cieux. Mais celui qui les accomplira et les enseignera, celui-là sera appelé grand dans le royaume des cieux.
\switchcolumn*

Ant. ad Offertorium.\hfill Ps 91, 13
\switchcolumn
Antienne d’offertoire.
\switchcolumn*

Justus ut palma florébit : sicut cedrus, quæ in Líbano est, multiplicábitur, allelúia.
\switchcolumn
Le juste fleurira comme le palmier ; il se multipliera comme le cèdre du Liban, alléluia.
\switchcolumn*

Secreta
\switchcolumn
Secrète
\switchcolumn*

Sancti Ansélmi Pontíficis tui atque  Doctóris, Dómine, pia non desit orátio : quæ et múnera nostra concíliet ; et tuam nobis indulgéntiam semper obtíneat. Per Dóminum.
\switchcolumn
Que la prière de saint Anselme, votre  Pontife et Docteur, Seigneur, ne nous fasse jamais défaut ; qu’elle rende agréables nos offrandes et nous obtienne toujours votre indulgence. Par Notre-Seigneur.
\switchcolumn*

Ant. ad Comm.\hfill Lc 12, 42
\switchcolumn
Antienne de communion.
\switchcolumn*

Fidélis servus et prudens, quem constítuit Dóminus super famíliam suam : ut det illis in témpore trítici mensúram, allelúia.
\switchcolumn
Voici le serviteur fidèle et prudent que son seigneur a établi sur sa famille, afin de leur donner en temps opportun leur mesure de blé, alléluia.
\switchcolumn*

Postcommunio
\switchcolumn
Postcommunion
\switchcolumn*

Ut nobis, Dómine, tua sacrifícia  dent salútem : beátus Ansélmus Póntifex tuus et Doctor egrégius, quǽsumus, precátor accédat. Per Dóminum.
\switchcolumn
Pour que vos sacrifices, Seigneur, nous  donnent le salut, que le bienheureux Anselme, votre Pontife et Docteur éminent, nous secoure par sa prière. Par Notre-Seigneur.
\switchcolumn*

Item alia Epistola pro Doctoribus :
\switchcolumn
Autre épître pour les Docteurs :
\switchcolumn*

Léctio libri Sapiéntiæ.
\switchcolumn
Lecture du Livre de la Sagesse.
\switchcolumn*

Si 39, 6-14
\switchcolumn

\switchcolumn*

Justus cor suum tradet ad  vigilándum dilúculo ad Dóminum, qui fecit illum, et in conspéctu Altíssimi deprecábitur. Apériet os suum in oratióne et pro delíctis suis deprecábitur. Si enim Dóminus magnus volúerit, spíritu intellegéntiæ replébit illum : et ipse tamquam imbres mittet elóquia sapiéntiæ suæ, et in oratióne confitébitur Dómino : et ipse díriget consílium ejus et disciplínam, et in abscónditis suis consiliábitur. Ipse palam faciet disciplínam doctrínæ suæ, et in lege testaménti Dómini gloriábitur. Collaudábunt multi sapiéntiam ejus, et usque in sǽculum non delébitur. Non recédet memória ejus, et nomen ejus requirétur a generatióne in generatiónem. Sapiéntiam ejus enarrábunt gentes, et laudem ejus enuntiábit ecclésia.
\switchcolumn
Le juste occupera son cœur dès le matin  à se tourner vers le Seigneur qui le créa, et en présence du Très-Haut il fera monter sa prière. Il ouvrira sa bouche dans la prière, et il suppliera pour ses péchés. Car si le Seigneur grand le veut, il le remplira de l’esprit d’intelligence. Quant à lui, il répandra comme une pluie les paroles de sa sagesse, et dans sa prière il célébrera le Seigneur. Il acquerra la droiture du jugement et de la connaissance, et il méditera sur ses mystères cachés. Il fera paraître l’instruction qu’il a reçue, et il se glorifiera dans la loi de l’alliance du Seigneur. Beaucoup loueront sa sagesse, et il ne sera pas effacé à jamais. Sa mémoire ne sera pas oubliée, et son nom sera recherché de génération en génération. Les peuples raconteront sa sagesse, et l’assemblée prononcera sa louange.
\switchcolumn*

dfsq
\switchcolumn
S. L.-M. Grignion de Montfort
\switchcolumn*

Die 28 Aprilis
\switchcolumn
Le 28 avril
\switchcolumn*

S. Ludovici Mariæ
\switchcolumn
S. Louis-Marie
\switchcolumn*

Grignion a Montfort
\switchcolumn
Grignion de Montfort
\switchcolumn*

Confessoris
\switchcolumn
Confesseur
\switchcolumn*

Ant. ad Introitum.\hfill Is 52, 7
\switchcolumn
Antienne d’introït.
\switchcolumn*

Quam pulchri super montes pedes annuntiántis et prædicántis pacem, annuntiántis bonum, prædicántis salútem, dicéntis Sion : Regnábit Deus tuus, allelúia, allelúia. √~Ps 48, 2. Audíte hæc, omnes gentes : áuribus percípite, omnes qui habitátis orbem. √~Glória Patri. Quam pulchri.
\switchcolumn
Qu’ils sont beaux sur les montagnes, les pieds de celui qui annonce et prêche la paix, qui annonce le bien, qui prêche le salut, qui dit à Sion : Ton Dieu règne, alléluia, alléluia ! √~Écoutez cela, tous les peuples, prêtez l’oreille, vous tous qui habitez la terre. √~Gloire au Père. Qu’ils sont beaux.
\switchcolumn*

Oratio
\switchcolumn
Collecte
\switchcolumn*

Deus, qui sanctum Ludovícum  Maríam regni Unigéniti Fílii tui præcónem exímium effecísti, et géminam per eum famíliam religiósam in Ecclésia tua suscitásti : concéde propítius ; ut, ipsíus mónitis et exémplo, eídem dilécto Fílio tuo, sub suávi jugo beatíssimæ Vírginis et Matris ejus, perénniter servíre valeámus : Qui tecum vivit.
\switchcolumn
Ô Dieu, qui avez fait de saint Louis- Marie un héraut éminent du règne de votre Fils unique, et par lui avez suscité une double famille religieuse dans votre Église, accordez-nous dans votre bonté, que, selon ses enseignements et son exemple, nous puissions servir toujours, sous le joug suave de la bienheureuse Vierge sa Mère, votre Fils bien aimé. Lui qui vit et règne avec vous.
\switchcolumn*

Léctio epístolæ beáti Pauli Apóstoli ad Corínthios.
\switchcolumn
Lecture de la première lettre du bienheureux Apôtre Paul aux Corinthiens.
\switchcolumn*

1 Co 1, 17-25
\switchcolumn

\switchcolumn*

Fratres : Non misit me Christus  baptizáre, sed evangelizáre : non in sapiéntia verbi, ut non evacuétur crux Christi. Verbum enim crucis pereúntibus quidem stultítia est : iis autem qui salvi fiunt, id est nobis, Dei virtus est. Scriptum est enim : Perdam sapiéntiam sapiéntium, et prudéntiam prudéntium reprobábo. Ubi sapiens ? ubi scriba ? ubi conquisítor hujus sǽculi ? Nonne stultam fecit Deus sapiéntiam hujus mundi ? Nam quia in Dei sapiéntia non cognóvit mundus per sapiéntiam Deum : plácuit Deo per stultítiam prædicatiónis salvos fácere credéntes. Quóniam et Judǽi signa petunt, et Græci sapiéntiam quærunt : nos autem prædicámus Christum crucifíxum : Judǽis quidem scándalum, géntibus autem stultítiam, ipsis autem vocátis Judǽis atque Græcis, Christum Dei virtútem, et Dei sapiéntiam : quia quod stultum est Dei, sapiéntius est homínibus : et quod infírmum est Dei, fórtius est homínibus.
\switchcolumn
Frères : Le Christ ne m’a pas envoyé pour  baptiser, mais pour évangéliser, non pas avec la sagesse de la parole, pour que la croix de Jésus-Christ ne soit pas rendue vaine. Car la parole de la croix est une folie pour ceux qui se perdent ; mais pour ceux qui se sauvent, c’est-à-dire pour nous, elle est force de Dieu. Car il est écrit : Je détruirai la sagesse des sages, et je rejetterai la science des savants. Où est le sage ? Où est le scribe ? Où est le savant de ce monde ? Dieu n’a-t-il pas rendu folle la sagesse de ce monde ? Car Dieu, voyant que le monde, avec la sagesse humaine, ne l’avait point connu dans les ouvrages de sa sagesse divine, il lui a plu de sauver les croyants par la folie de la prédication. Car les Juifs demandent des signes, et les païens cherchent la sagesse. Mais nous, nous prêchons le Christ crucifié, scandale pour les Juifs et folie pour les païens, mais pour ceux qui sont appelés, Juifs et païens, il est force de Dieu et sagesse de Dieu. Parce que ce qui est folie de Dieu est plus sage que les hommes ; et ce qui est faiblesse de Dieu est plus fort que les hommes.
\switchcolumn*

Allelúia, allelúia. √~1 Co 1, 23.24. Nos autem prædicámus Christum crucifíxum, Dei virtútem et Dei sapiéntiam.
\switchcolumn
Alléluia, alléluia. √~Nous prêchons le Christ crucifié, force de Dieu et sagesse de Dieu.
\switchcolumn*

Allelúia. √~Si 3, 5.6. Sicut qui thesaurízat, ita et qui honoríficat Matrem suam : et in die oratiónis suæ exaudiétur. Allelúia.
\switchcolumn
Alléluia. √~Comme celui qui amasse un trésor, ainsi celui qui honore sa Mère. Au jour de sa prière il sera exaucé. Alléluia.
\switchcolumn*

Sequéntia sancti Evangélii secúndum Joánnem.
\switchcolumn
Lecture du saint évangile selon saint Jean.
\switchcolumn*

Jo 19, 25-27
\switchcolumn

\switchcolumn*

In illo témpore : Stabant juxta crucem  Jesu mater ejus, et soror matris ejus, María Cléophæ, et María Magdaléne. Cum vidísset ergo Jesus matrem, et discípulum stantem, dicit matri suæ : Múlier, ecce fílius tuus. Deínde dicit discípulo : Ecce mater tua. Et ex illa hora accépit eam discípulus in sua.
\switchcolumn
En ce temps-là, près de la croix de Jésus  se tenaient sa mère et la sœur de sa mère, Marie de Cléophas, ainsi que Marie Madeleine. Donc, quand Jésus vit sa mère et le disciple qui se tenait là, il dit à sa mère : Femme, voici ton fils. Ensuite il dit au disciple : Voici ta mère. Et depuis cette heure-là, le disciple la prit chez lui.
\switchcolumn*

Ant. ad Offertorium.\hfill Ps 115, 16-17
\switchcolumn
Antienne d’offertoire.
\switchcolumn*

O Dómine, quia ego servus tuus : ego servus tuus et fílius ancíllæ tuæ. Dirupísti víncula mea : tibi sacrificábo hóstiam laudis, allelúia.
\switchcolumn
Ô Seigneur, je suis votre serviteur, et le fils de votre servante. Vous avez rompu mes liens, je vous offrirai en sacrifice l’hostie de louange, alléluia.
\switchcolumn*

Secreta
\switchcolumn
Secrète
\switchcolumn*

Múnera altári tuo, Dómine,  superpósita, sancto Ludovíco María intercedénte, propítius réspice, ac nos quoque per beatíssimam Vírginem Maríam hóstias tibi placéntes effícere dignáre. Per Dóminum.
\switchcolumn
Regardez avec bonté, Seigneur, par l’intercession de saint Louis-Marie, les offrandes placées sur votre autel, et daignez faire de nous aussi, par la bienheureuse Vierge Marie, des hosties qui vous soient agréables. Par Notre-Seigneur.
\switchcolumn*

Ant. ad Comm.\hfill Si 3, 5.6
\switchcolumn
Antienne de communion.
\switchcolumn*

Sicut qui thesaurízat, ita et qui honoríficat Matrem suam : et in die oratiónis suæ exaudiétur, allelúia.
\switchcolumn
Comme celui qui amasse un trésor, ainsi celui qui honore sa Mère. Au jour de sa prière il sera exaucé, alléluia.
\switchcolumn*

Postcommunio
\switchcolumn
Postcommunion
\switchcolumn*

Grátia tua nos, Dómine, non  derelínquat : quæ, sancto Ludovíco María intercedénte, et sacræ nos déditos fáciat servitúti, et tuam nobis, per Vírginem Matrem, opem semper acquírat. Per Dóminum.
\switchcolumn
Que votre grâce, Seigneur, ne nous  abandonne pas. Par l’intercession de saint Louis-Marie, qu’elle nous consacre entièrement au saint esclavage, et que, par la Vierge Mère, elle nous obtienne toujours votre secours. Par Notre-Seigneur.
\switchcolumn*

fds
\switchcolumn
SS. Abbés de Cluny
\switchcolumn*

Die 11 Maii
\switchcolumn
Le 11 mai
\switchcolumn*

SS. Odonis, Majoli,Odilonis, Hugonis,
\switchcolumn
SS. Odon, Maïeul,Odilon, Hugues,
\switchcolumn*

et B. Petri Venerabilis,
\switchcolumn
B. Pierre le Vénérable,
\switchcolumn*

Abbatum Cluniacensium
\switchcolumn
Abbés de Cluny
\switchcolumn*


\switchcolumn

\switchcolumn*

Ant. ad Introitum.\hfill Mt 25, 34
\switchcolumn
Antienne d’introït.
\switchcolumn*

Veníte, benedícti Patris mei, percípite regnum, allelúia : quod vobis parátum est ab orígine mundi, allelúia, allelúia. √~Ps 95, 1. Cantáte Dómino cánticum novum : cantáte Dómino omnis terra. √~Glória Patri. Veníte.
\switchcolumn
Venez, les bénis de mon Père, recevez le royaume, alléluia, qui vous a été préparé depuis l’origine du monde, alléluia, alléluia ! √~Chantez au Seigneur un chant nouveau, chantez au Seigneur, toute la terre. √~Gloire au Père. Venez.
\switchcolumn*

Oratio
\switchcolumn
Collecte
\switchcolumn*

Deus, qui nos Sanctórum tuórum et  solemnitáte lætíficas, et imitatióne súscitas ad proféctum : præsta, ut quos venerámur offício, étiam piæ conversatiónis sequámur exémplo. Per Dóminum.
\switchcolumn
Ô Dieu, qui nous réjouissez par la solen- nité de vos saints, et nous poussez au progrès par leur imitation, accordez-nous de suivre par l’exemple d’une sainte vie ceux que nous vénérons par cet office. Par Notre-Seigneur.
\switchcolumn*

Lectio libri Sapiéntiæ.
\switchcolumn
Lecture du Livre de la Sagesse.
\switchcolumn*

Si 17, 6-13
\switchcolumn

\switchcolumn*

Creávit illis Deus sciéntiam spíritus,  sensu implevit cor illórum, et mala et bona osténdit illis. Pósuit oculum suum super corda illórum, osténdere illis magnália óperum suórum : ut nomen sanctificatiónis colláudent, et gloriári in mirabílibus illíus ; ut magnália enárrent óperum ejus. Addidit illis disciplínam, et legem vitæ hæreditávit illos. Testaméntum ætérnum constítuit cum illis, et justítiam et judícia sua osténdit illis. Et magnália honóris ejus vidit óculus illórum, et honórem vocis audiérunt aures illórum. Et dixit illis : Atténdite ab omni iníquo. Et mandávit illis unicuíque de próximo suo. Viæ illórum coram ipso sunt semper : non sunt abscónsæ ab óculis ipsíus.
\switchcolumn
Dieu a créé en eux la science de l’esprit,  il a rempli leur cœur d’intelligence, et il leur a montré le mal et le bien. Il a mis son œil sur leur cœur, pour leur montrer la grandeur de ses œuvres, afin qu’ils louent ensemble son saint Nom, et qu’ils le glorifient dans ses merveilles, pour qu’ils racontent la grandeur de ses œuvres. Il a augmenté en eux la science, et il les a fait hériter de la loi de vie. Il a passé avec eux une alliance éternelle, et il leur a montré la justice et ses jugements. Leur œil a vu la grandeur de sa gloire, et leurs oreilles ont entendu la majesté de sa voix. Et il leur a dit : Gardez-vous de toute iniquité. Il leur a commandé de s’occuper chacun de son prochain. Leur voies sont toujours devant lui, elles ne sont pas cachées à ses yeux.
\switchcolumn*

Allelúia, allelúia. √~Ps 88, 6. Confitébuntur cæli mirabília tua, Dómine : étenim veritátem tuam in ecclésia sanctórum.
\switchcolumn
Alléluia, alléluia. √~Les cieux célèbreront vos merveilles, Seigneur, et votre vérité dans l’assemblée des saints.
\switchcolumn*

Allelúia. √~Os 14, 6. Justus germinábit sicut lílium : et florébit in ætérnum ante Dóminum. Allelúia.
\switchcolumn
Alléluia. √~Le juste germera comme le lys, et il fleurira à jamais devant le Seigneur. Alléluia.
\switchcolumn*

Sequéntia sancti Evangélii secúndum Joánnem
\switchcolumn
Lecture du saint évangile selon saint Jean.
\switchcolumn*

Jo 15, 5-11
\switchcolumn

\switchcolumn*

In illo témpore, dixit Jesus discípulis  suis : Ego sum vitis, vos pálmites : qui manet in me, et ego in eo, hic fert fructum multum, quia sine me nihil potéstis fácere. Si quis in me non mánserit, mittétur foras sicut palmes, et aréscet, et cólligent eum, et in ignem mittent, et ardet. Si manséritis in me, et verba mea in vobis mánserint, quodcúmque voluéritis petétis, et fiet vobis. In hoc clarificátus est Pater meus, ut fructum plúrimum afferátis, et efficiámini mei discípuli. Sicut diléxit me Pater, et ego diléxi vos. Manéte in dilectióne mea. Si præcépta mea servavéritis, manébitis in dilectióne mea, sicut et ego Patris mei præcépta servávi, et máneo in ejus dilectióne. Hæc locútus sum vobis : ut gáudium meum in vobis sit, et gáudium vestrum impleátur.
\switchcolumn
En ce temps-là, Jésus dit à ses disciples :  Je suis la vigne, vous les sarments. Celui qui demeure en moi, et moi en lui, celui-là porte beaucoup de fruit, car sans moi vous ne pouvez rien faire. Si quelqu’un ne demeure pas en moi, il sera jeté dehors comme un sarment, il se dessèchera, on le ramassera, on le jettera au feu et il brûlera. Si vous demeurez en moi, et si mes paroles demeurent en vous, demandez tout ce que vous voulez et cela vous sera accordé. Ce qui glorifie mon Père, c’est que vous portiez beaucoup de fruit et que vous deveniez mes disciples. Comme le Père m’a aimé, moi aussi je vous ai aimés. Demeurez dans mon amour. Si vous gardez mes préceptes, vous demeurerez dans mon amour, de même que moi j’ai gardé les préceptes de mon Père et je demeure dans son amour. Je vous ai dit cela pour que ma joie soit en vous et que votre joie soit entière.
\switchcolumn*

Ant. ad Offertorium.\hfill Ps 89, 14
\switchcolumn
Antienne d’offertoire.
\switchcolumn*

Repléti sumus mane misericórdia tua : et exsultávimus et delectáti sumus, allelúia.
\switchcolumn
Nous sommes remplis de votre miséricorde dès le matin, nous exultons et nous nous réjouissons, alléluia.
\switchcolumn*

Secreta
\switchcolumn
Secrète
\switchcolumn*

Sacrifícium, Dómine, quǽsumus,  beatórum Odónis, Májoli, Odilónis et Hugónis precátio sancta concíliet, ut quorum honóre solémniter exhibétur, eórum méritis efficiátur accéptum. Per Dóminum.
\switchcolumn
Que par la prière sainte des bienheureux  Odon, Maïeul, Odilon et Hugue, Seigneur, ce sacrifice trouve grâce devant vous. Offert solennellement en leur honneur, qu’il vous devienne agréable par leurs mérites. Par Notre-Seigneur.
\switchcolumn*

Ant. ad Comm.\hfill Jo 15, 5
\switchcolumn
Antienne de communion.
\switchcolumn*

Ego sum vitis vera et vos pálmites, qui manet in me, et ego in eo, hic fert fructum multum. Allelúia, allelúia.
\switchcolumn
Je suis la vraie vigne, et vous les sarments. Celui qui demeure en moi, et moi en lui, celui-là porte beaucoup de fruit, alléluia, alléluia.
\switchcolumn*

Postcommunio
\switchcolumn
Postcommunion
\switchcolumn*

Grátias tibi reférimus, Dómine,  qui nos et cæléstis participatióne Sacraménti, et tuórum réficis celebritáte justórum. Per Dóminum.
\switchcolumn
Nous vous rendons grâces, Seigneur, car  vous nous restaurez par la participation à ce sacrement et par la célébration de vos saints. Par Notre-Seigneur.
\switchcolumn*

Die 30 Maii
\switchcolumn
Le 30 mai
\switchcolumn*

S. Ioannæ de Arc
\switchcolumn
S. Jeanne d’Arc
\switchcolumn*

Virginis
\switchcolumn
Vierge
\switchcolumn*

Ant. ad Introitum.\hfill Ex 15, 1-2
\switchcolumn
Antienne d’introït.
\switchcolumn*

Cantémus Dómino : glorióse enim  magnificátus est. Fortitúdo mea et laus mea Dóminus, et factus est mihi in salútem (T. P. allelúia, allelúia). √~Ps 97, 1. Cantáte Dómino cánticum novum, quia mirabília fecit. √~Glória Patri. Cantémus.
\switchcolumn
Chantons pour le Seigneur, car il a été  glorieusement magnifié. Ma force et ma louange, c’est le Seigneur, et il s’est fait mon salut (T. P. Alléluia, alléluia). √~Chantez au Seigneur un chant nouveau, car il a fait des merveilles. √~Gloire au Père. Chantons pour le Seigneur.
\switchcolumn*

Oratio
\switchcolumn
Collecte
\switchcolumn*

Deus, qui beátam Joánnam  Vírginem ad fidem ac pátriam tuéndam mirabíliter suscitásti : da, quǽsumus, ejus intercessióne ; ut Ecclésia tua, hóstium superátis insídiis, perpétua pace fruátur. Per Dóminum.
\switchcolumn
Ô Dieu, qui d’une manière admirable  avez suscité la bienheureuse vierge Jeanne pour protéger la foi et la patrie, accordez-nous, nous vous en prions, par son intercession, que votre Église, en triomphant des pièges de l’ennemi, jouisse d’une paix éternelle. Par Notre-Seigneur.
\switchcolumn*

Lectio libri Sapiéntiæ.
\switchcolumn
Lecture du Livre de la Sagesse.
\switchcolumn*

Sap 8, 9-15
\switchcolumn

\switchcolumn*

Propósui sapiéntiam addúcere mihi  ad convivéndum, sciens quóniam mecum communicábit de bonis, et erit allocútio cogitatiónis et tǽdii mei. Habébo propter hanc claritátem ad turbas, et honórem apud senióres júvenis ; et acútus invéniar in judício, et in conspéctu poténtium admirábilis ero, et fácies príncipum mirabúntur me : tacéntem me sustinébunt, et loquéntem me respícient, et sermocinánte me plura, manus ori suo impónent. Prætérea habébo per hanc immortalitátem, et memóriam ætérnam his qui post me futúri sunt relínquam. Dispónam pópulos, et natiónes mihi erunt súbditæ : timébunt me audiéntes reges horréndi. In multitúdine vidébor bonus, et in bello fortis.
\switchcolumn
Je me suis proposé d’amener la Sagesse  à vivre avec moi, sachant qu’elle me communiquera ses biens, et qu’elle sera la consolation de ma pensée et de mon ennui. Grâce à elle, j’acquerrai la gloire auprès de la multitude, et, bien que jeune, j’aurai de l’honneur auprès des vieillards. Je serai trouvé pénétrant dans les jugement, et en présence des puissants je serai admirable, et la face des princes me regardera avec étonnement. Quand je me tairai, ils attendront patiemment, et quand je parlerai, ils me regarderont, et quand je discourrai sur plusieurs sujets, ils mettront la main sur la bouche. Outre cela, j’aurai par elle l’immortalité, et laisserai une mémoire éternelle à ceux qui doivent venir après moi. Je gouvernerai des peuples, et des nations me seront soumises. Les rois les plus redoutables me craindront lorsqu’ils m’entendront ; au milieu de la multitude je me montrerai bon, et vaillant dans la guerre.
\switchcolumn*

Tempore paschali :
\switchcolumn
Au Temps pascal :
\switchcolumn*

Allelúia, allelúia. √~Judith 15, 11. Fecísti viríliter, et confortátum est cor tuum : manus Dómini confortávit te, et ídeo eris benedícta in ætérnum.
\switchcolumn
Alléluia, alléluia. √~Tu as agi virilement, et ton cœur s’est fortifié. La main du Seigneur t’a confortée, et c’est pourquoi tu seras bénie à jamais.
\switchcolumn*

Allelúia. √~Ibid. 8, 29. Nunc ergo ora pro nobis, quóniam múlier sancta es, et timens Deum. Allelúia.
\switchcolumn
Alléluia. Maintenant donc, prie pour nous, car tu es une femme sainte, qui craint Dieu. Alléluia.
\switchcolumn*

Extra tempus paschale :
\switchcolumn
Hors du Temps pascal :
\switchcolumn*

Graduale. Iudic. 5, 8-11. Nova bella elégit Dóminus, et portas hóstium ipse subvértit. √~Ubi collísi sunt currus, et hóstium suffocátus est exércitus, ibi narréntur justítiæ Dómini, et cleméntia ejus in fortes Israel.
\switchcolumn
Graduel. Le Seigneur a choisi de nouvelles guerres, et il a lui-même renversé les portes des ennemis. √~Là où les chars se sont brisés, et où l’armée ennemie a été étouffée, que soient racontées les justices du Seigneur, et sa clémence envers les forts d’Israël.
\switchcolumn*

Allelúia, allelúia. √~Judith 13, 17-18. Laudáte Dóminum Deum nostrum, qui non deséruit sperántes in se, et in me ancílla sua adimplévit misericórdiam suam, quam promísit dómui Israel. Allelúia.
\switchcolumn
Alléluia, alléluia. √~Louez le Seigneur notre Dieu, lui qui n’a pas délaissé ceux qui espèrent en lui, et qui a accompli en moi, sa servante, sa miséricorde, qu’il avait promise à la maison d’Israël. Alléluia.
\switchcolumn*

Sequéntia sancti Evangélii secúndum Matthǽum.
\switchcolumn
Lecture du saint évangile selon saint Matthieu.
\switchcolumn*

Mt 16, 24-27
\switchcolumn

\switchcolumn*

In illo témpore, dixit Jesus discípulis  suis : Si quis vult post me veníre, ábneget semetípsum, et tollat crucem suam, et sequátur me. Qui enim volúerit ánimam suam salvam fácere, perdet eam : qui autem perdíderit ánimam suam propter me, invéniet eam. Quid enim prodest hómini, si mundum univérsum lucrétur, ánimæ vero suæ detriméntum patiátur ? Aut quam dabit homo commutatiónem pro ánima sua ? Fílius enim hóminis ventúrus est in glória Patris sui cum ángelis suis : et tunc reddet unicuíque secúndum ópera ejus.
\switchcolumn
En ce temps-là, Jésus dit à ses disciples :  Si quelqu’un veut venir apès moi, qu’il renonce à lui-même, qu’il prenne sa croix et qu’il me suive. Car celui qui voudra sauver sa vie la perdra, mais celui qui la perdra pour moi la trouvera. Que sert à l’homme de gagner le monde entier si c’est au détriment de son âme ? Ou que donnera un homme en échange de son âme ? Car le Fils de l’Homme va venir dans la gloire de son Père avec ses anges, et alors il rendra à chacun selon ses œuvres.
\switchcolumn*

Ant. ad Offertorium.\hfill Judith 15, 10
\switchcolumn
Antienne d’offertoire.
\switchcolumn*

Benedixérunt eam omnes una voce, dicéntes : Tu glória Jerúsalem, tu lætítia Israel, tu honorificéntia pópuli nostri (T. P. allelúia).
\switchcolumn
Ils la bénirent tous d’une seule voix, en disant : Tu es la gloire de Jérusalem, la joie d’Israël, l’honneur de notre peuple (T. P. alléluia).
\switchcolumn*

Secreta
\switchcolumn
Secrète
\switchcolumn*

Hæc hóstia salutáris, Dómine, illam  nobis in rebus árduis cónferat fortitúdinem, cujus beáta Joánna, sub tanta discríminum varietáte, tam insígnia prǽbuit exémpla : ut, ad inimícos repelléndos, étiam belli perícula subíre non dubitáverit. Per Dóminum.
\switchcolumn
Que cette hostie salutaire, Seigneur,  nous confère dans les difficultés cette même force dont la bienheureuse Jeanne, au milieu de tant de dangers, donna des exemples si merveilleux, au point de ne pas hésiter à se livrer aux dangers de la guerre pour repousser l’ennemi. Par Notre-Seigneur.
\switchcolumn*

Ant. ad Comm.\hfill Ps 22, 4
\switchcolumn
Antienne de communion.
\switchcolumn*

Si ambulávero in médio umbræ mortis, non timébo mala, quóniam tu mecum es, Dómine Jesu (T. P. allelúia).
\switchcolumn
Même si je marche au milieu de l’ombre de la mort, je ne crains pas le mal, car tu es avec moi, Seigneur Jésus (T. P. alléluia).
\switchcolumn*

Postcommunio
\switchcolumn
Postcommunion
\switchcolumn*

Cælésti pane reféctos, qui tóties beátam Joánnam áluit ad victóriam, præsta, quǽsumus, omnípotens Deus ; ut hoc salútis aliméntum de inimícis nostris victóres nos effíciat. Per Dóminum.
\switchcolumn
Que cet aliment salutaire, nous vous en prions, Dieu tout puissant, nous rende victorieux sur nos ennemis, puisque nous avons été rassasiés de ce pain céleste qui nourrit tant de fois la bienheureuse Jeanne avant la victoire. Par Notre-Seigneur.
\switchcolumn*

Die 13 junii
\switchcolumn
Le 13 juin
\switchcolumn*

S. Antonii de Padua
\switchcolumn
S. Antoine de Padoue
\switchcolumn*

Conf. et Eccl. Doct.
\switchcolumn
Conf. et Docteur de l’Église
\switchcolumn*

Ant. ad Introitum.\hfill Si 15, 5
\switchcolumn
Antienne d’introït.
\switchcolumn*

In médio Ecclésiæ apéruit os ejus :  et implévit eum Dóminus spíritu sapiéntiæ et intelléctus : stolam glóriæ índuit eum. √~Ps 91, 2. Bonum est confitéri Dómino : et psállere nómini tuo, Altíssime. √~Glória Patri. In médio.
\switchcolumn
Au milieu de l’Église, le Seigneur a  ouvert sa bouche et il l’a rempli de l’esprit de sagesse et d’intelligence ; il l’a revêtu du vêtement de gloire. √~Il est bon de célébrer le Seigneur, et de psalmodier pour votre Nom, ô Très-Haut. Gloire au Père. Au milieu.
\switchcolumn*

Oratio
\switchcolumn
Collecte
\switchcolumn*

Ecclésiam tuam, Deus, beáti  Antónii Confessóris tui atque Doctóris solémnitas votíva lætíficet : ut spirituálibus semper muniátur auxíliis, et gáudiis pérfrui mereátur ætérnis. Per Dóminum.
\switchcolumn
Que la fête du bienheureux Antoine,  votre Confesseur et Docteur, ô Dieu, réjouisse votre Église, afin qu’elle soit toujours munie de secours spirituels et mérite de jouir des joies éternelles. Par Notre-Seigneur.
\switchcolumn*

Léctio libri Sapiéntiæ.
\switchcolumn
Lecture du Livre de la Sagesse.
\switchcolumn*

Sap 7, 7-15
\switchcolumn

\switchcolumn*

Optávi, et datus est mihi sensus ; et invocávi, et venit in me spíritus sapiéntiæ : et præpósui illam regnis et sédibus, et divítias nihil esse duxi in comparatióne illíus. Nec comparávi illi lápidem pretiósum, quóniam omne aurum in comparatióne illíus aréna est exígua, et tamquam lutum æstimábitur argéntum in conspéctu illíus. Super salútem et spéciem diléxi illam, et propósui pro luce habére illam, quóniam inextinguíbile est lumen illíus. Venérunt autem mihi ómnia bona páriter cum illa, et innumerábilis honéstas per manus illíus ; et lætátus sum in ómnibus, quóniam antecedébat me ista sapiéntia, et ignorábam quóniam horum ómnium mater est. Quam sine fictióne dídici, et sine invídia commúnico, et honestátem illíus non abscóndo. Infinítus enim thesáurus est homínibus ; quo qui usi sunt, partícipes facti sunt amicítiæ Dei, propter disciplínæ dona commendáti. Mihi autem dedit Deus dícere ex senténtia, et præsúmere digna horum quæ mihi dantur : quóniam ipse sapiéntiæ dux est, et sapiéntium emendátor.
\switchcolumn
J’ai choisi, et l’intelligence m’a été donnée, et l’esprit de sagesse est venu en moi. Je l’ai mise avant les royaumes et les trônes, et j’ai compté pour rien les richesses auprès d’elle. Je ne lui ai pas comparé les pierres précieuses, car tout l’or auprès d’elle n’est qu’un peu de sable, et l’argent sera compté comme de la boue en sa présence. Je l’ai aimée plus que la santé et la beauté, je me suis proposé de l’avoir pour lampe, car sa lumière ne s’éteint pas. Tous les biens me sont venus ensemble avec elle, et des richesses innombrables par ses mains, et je me suis réjoui en tout, car cette sagesse marchait devant moi, et j’ignorais qu’elle est la mère de toutes ces choses. Je l’ai apprise sans déguisement, sans envie je la communique, et je ne cache point ses richesses. Car elle est un trésor infini pour les hommes. Ceux qui en ont usé sont devenus participants de l’amitié de Dieu, recommandables par les dons de la science. À moi Dieu a donné de parler en sentences, et d’avoir des pensées dignes de ce qui m’a été donné. Car c’est lui le maître de la sagesse, et le réformateur des sages.
\switchcolumn*

Graduale. √~Si 24, 3-4. In médio pópuli sui exaltábitur, et in plenitúdine sancta admirábitur. √~In multitúdine electórum habébit laudem, et inter benedíctos benedicétur.
\switchcolumn
Graduel. Au milieu de son peuple il sera exalté, et il sera admirable dans l’assemblée sainte. √~Dans la multitude des élus il sera loué, et il sera béni parmi les bénis.
\switchcolumn*

Allelúia, allelúia. √~Si 48, 15. In vita sua fecit monstra, et in morte mirabília operátus est. Allelúia.
\switchcolumn
Alléluia, alléluia. √~Pendant sa vie il fit des prodiges, et dans sa mort il a opéré des merveilles. Alléluia.
\switchcolumn*

Sequéntia sancti Evangélii secúndum Matthǽum.
\switchcolumn
Lecture du saint Évangile selon saint Matthieu.
\switchcolumn*

Mt 5, 13-19
\switchcolumn

\switchcolumn*

In illo témpore : Dixit Jesus discípulis  suis : Vos estis sal terræ. Quod si sal evanúerit, in quo saliétur ? Ad níhilum valet ultra, nisi ut mittátur foras, et conculcétur ab homínibus. Vos estis lux mundi. Non potest cívitas abscóndi supra montem pósita. Neque accéndunt lucérnam, et ponunt eam sub módio, sed super candelábrum, ut lúceat ómnibus qui in domo sunt. Sic lúceat lux vestra coram homínibus, ut vídeant ópera vestra bona, et gloríficent Patrem vestrum, qui in cælis est. Nolite putáre, quóniam veni sólvere legem, aut prophétas : non veni sólvere, sed adimplére. Amen quippe dico vobis, donec tránseat cælum et terra, iota unum, aut unus apex non præteríbit a lege, donec ómnia fiant. Qui ergo sólverit unum de mandátis istis mínimis, et docúerit sic hómines, mínimus vocábitur in regno cælórum : qui autem fécerit, et docúerit, hic magnus vocábitur in regno cælórum.
\switchcolumn
En ce temps-là, Jésus dit à ses disciples :  Vous êtes le sel de la terre. Si le sel s’affadit, avec quoi le salera-t-on ? Il n’est plus bon à rien, sinon à être jeté dehors et piétiné par les hommes. Vous êtes la lumière du monde. Une ville placée sur une montagne ne peut être cachée. Et quand on allume une lampe, ce n’est pas pour la placer sous le boisseau mais sur le candélabre, afin qu’elle brille pour tous ceux qui sont dans la maison. Qu’ainsi votre lumière brille devant les hommes, afin qu’ils voient vos bonnes œuvres et qu’ils glorifient votre Père qui est dans les cieux. Ne pensez pas que je sois venu abolir la loi ou les prophètes. Je ne suis pas venu abolir, mais accomplir. Amen je vous le dis : Jusqu’à ce que passent le ciel et la terre, pas un iota ou un trait de la loi ne passera : tout sera accompli. Celui donc qui abolira l’un de ces plus petits commandements et enseignera aux hommes à faire ainsi, sera appelé le plus petit dans le royaume des cieux. Mais celui qui les accomplira et les enseignera, celui-là sera appelé grand dans le royaume des cieux.
\switchcolumn*

Ant. ad Offertorium.\hfill Si 49, 1.2
\switchcolumn
Antienne d’offertoire.
\switchcolumn*

Memória ejus in compositióne odóris ; in omni ore quasi mel indulcábitur memória ejus.
\switchcolumn
Sa mémoire est une mixture d’encens ; dans toutes les bouches elle sera comme la douceur du miel.
\switchcolumn*

Secreta
\switchcolumn
Secrète
\switchcolumn*

Hóstias tibi, Dómine, in beáti  Antónii Confessóris tui atque Doctóris solemnitáte offeréntes : te súpplices deprecámur, ut sicut ipsum cæléstibus donis cumulásti ; ita nos fácias tuo amóre fervéntes. Per Dóminum.
\switchcolumn
En vous offrant ces hosties, Seigneur, en  la solennité du bienheureux Antoine votre Confesseur et Docteur, nous vous supplions de nous rendre fervents dans votre amour, comme vous l’avez lui-même rempli de dons célestes. Par Notre-Seigneur.
\switchcolumn*

Ant. ad Comm.\hfill Si 51, 30
\switchcolumn
Antienne de communion.
\switchcolumn*

Dedit mihi Dóminus linguam mércedem meam, et in ipsa laudábo eum.
\switchcolumn
Le Seigneur m’a donné en récompense une langue avec laquelle je le louerai.
\switchcolumn*

Postcommunio
\switchcolumn
Postcommunion
\switchcolumn*

Divínis, Dómine, munéribus satiáti,  quǽsumus ; ut beáti Antónii Confessóris tui atque Doctóris méritis et intercessióne, salutáris sacrifícii sentiámus efféctum. Per Dóminum.
\switchcolumn
Rassasiés par vos dons sacrés, Seigneur,  nous vous prions, par les mérites et l’intercession du bienheureux Antoine votre confesseur et docteur, de nous faire sentir les effets de ce sacrifice salutaire. Par Notre-Seigneur.
\switchcolumn*

Die 6 julii
\switchcolumn
Le 6 juillet
\switchcolumn*

S. Mariæ Goretti
\switchcolumn
S. Maria Goretti
\switchcolumn*

Virg. et mart.
\switchcolumn
Vierge et martyre
\switchcolumn*

Ant. ad Introitum\hfill Ps 118, 95-96
\switchcolumn
Antienne d’introït.
\switchcolumn*

Me exspéctant peccatóres ut  perdant me : ad præscrípta tua atténdo : omnis perfectiónis vidi esse términum : latíssime patet mandátum tuum. √~Ibid., 1. Beáti quorum immaculáta est via : qui ámbulant in lege Dómini. √~Glória Patri. Me exspéctant peccatóres.
\switchcolumn
Les pécheurs m’attendent pour me  perdre : je prends garde à vos commandements : je vois qu’il y a un terme à toute perfection : votre commandement s’étend largement. √~Heureux ceux dont la voie est immaculée, qui marchent dans la loi du Seigneur. √~Gloire au Père. Les pécheurs m’attendent.
\switchcolumn*

Oratio
\switchcolumn
Collecte
\switchcolumn*

Deus, qui fámulæ tuæ Maríæ in  ténera ætáte victóriam martýrii contulísti : da nobis, quǽsumus, ejus patrocínio in mandátis tuis constántiam ; qui dedísti certánti Vírgini corónam. Per Dóminum.
\switchcolumn
Dieu qui avez donné à votre servante  Marie la victoire du martyre à un âge encore tendre, donnez-nous par son secours, nous vous en prions, la constance dans l’observation de vos commandements, vous qui avez accordé la couronne à celle qui combattait. Par Notre-Seigneur.
\switchcolumn*

Léctio epístolæ beáti Pauli Apóstoli ad Corínthios.
\switchcolumn
Lecture de la lettre de saint Paul Apôtre aux Corinthiens.
\switchcolumn*

1 Co 1, 26-29 ; 2, 14
\switchcolumn

\switchcolumn*

Fratres : Vidéte vocatiónem vestram,  quia non multi sapiéntes secúndum carnem, non multi poténtes, non multi nóbiles : sed quæ stulta sunt mundi elégit Deus, ut confúndat sapiéntes : et infírma mundi elégit Deus, ut confúndat fórtia : et ignobília mundi et contemptibília elégit Deus, et ea quæ non sunt, ut ea quæ sunt destrúeret : ut non gloriétur omnis caro in conspéctu ejus. Animális autem homo non pércipit ea quæ sunt Spíritus Dei : stultítia enim est illi, et non potest intellégere, quia spirituáliter examinátur.
\switchcolumn
Frères, voyez votre vocation : il y a peu  de sages selon la chair, peu de puissants, peu de nobles. Mais Dieu a choisi ce qui est fou dans le monde pour confondre les sages. Et Dieu a choisi ce qui est faible dans le monde pour confondre ce qui est fort. Et Dieu a choisi ce qui est sans noblesse dans le monde, ce qui est méprisable, ce qui n’est rien, pour détruire ce qui est, afin que nulle chair ne se glorifie en sa présence. Mais l’homme animal ne perçoit pas ce qui est de l’Esprit de Dieu ; c’est une folie pour lui, et il ne peut le comprendre, car il faut l’examiner spirituellement.
\switchcolumn*

Graduale. √~Ps 70, 4 et 6. Deus meus, éripe me de manu iníqui, de pugno ímprobi et oppressóris. √~A ventre matris meæ eras protéctor meus.
\switchcolumn
Graduel. Mon Dieu, arrache-moi de la main de l’inique, du poing de l’homme méchant et de l’oppresseur. √~Depuis le ventre de ma mère, tu es mon protecteur.
\switchcolumn*

Allelúia, allelúia. √~Ibid., 6 et 7. In te sperávi semper. Tamquam prodígium appárui multis : tu enim fuísti adjútor meus fortis. Allelúia.
\switchcolumn
Alléluia, alléluia. √~En toi j’ai toujours espéré. Je suis apparu à beaucoup comme un prodige, car tu as été mon puissant soutien. Alléluia.
\switchcolumn*

Sequéntia sancti Evangélii secúndum Joánnem.
\switchcolumn
Lecture du saint évangile selon saint Jean.
\switchcolumn*

Jo 12, 23-25
\switchcolumn

\switchcolumn*

In illo témpore : Dixit Jesus discípulis  suis : Venit hora ut clarificétur Fílius hóminis. Amen, amen dico vobis, nisi granum fruménti cadens in terram, mórtuum fúerit, ipsum solum manet : si autem mórtuum fúerit, multum fructum affert. Qui amat ánimam suam, perdet eam : et qui odit ániman suam in hoc mundo, in vitam ætérnam custódit eam.
\switchcolumn
En ce temps-là, Jésus dit à ses disciples :  L’heure vient pour le Fils de l’homme d’être glorifié. Amen, amen je vous le dis, si le grain de blé tombant en terre ne meurt pas, il reste seul, mais s’il meurt il porte beaucoup de fruit. Celui qui aime  sa vie la perdra, et celui qui haît sa vie en ce monde la garde pour la vie éternelle.
\switchcolumn*

Ant. ad Offertorium.\hfill Ps 73, 19
\switchcolumn
Antienne d’offertoire.
\switchcolumn*

Ne tradíderis vúlturi vitam túrturis tui : vitam páuperum tuórum noli oblivísci in perpétuum.
\switchcolumn
Ne livre pas ta tourterelle au vautour. N’oublie pas à jamais la vie de tes pauvres.
\switchcolumn*

Secreta
\switchcolumn
Secrète
\switchcolumn*

Placatiónis tibi hóstiam, Dómine,  offérimus, qua beáta María fámula tua, prima jam ætáte, dídicit corpus suum hóstiam sanctam tibíque placéntem exhibére. Per Dóminum.
\switchcolumn
Nous vous offrons, Seigneur, l’hostie  d’apaisement, par laquelle, jeune encore, la bienheureuse Marie, votre servante, apprit à vous offrir son corps en hostie sainte et agréable à vos yeux. Par Notre-Seigneur.
\switchcolumn*

Ant. ad Comm.\hfill Is 33, 6
\switchcolumn
Antienne de communion.
\switchcolumn*

Timor Dómini ipse est thesáurus ejus.
\switchcolumn
La crainte du Seigneur est son trésor.
\switchcolumn*

Postcommunio
\switchcolumn
Postcommunion
\switchcolumn*

Cælésti pane reféctis, da, quǽsumus,  Dómine, eam in tuénda córporis et ánimæ castitáte fortitúdinem : quam fámulæ tuæ Maríæ mirabíliter contulísti. Per Dóminum.
\switchcolumn
À nous qui avons été réconfortés par le  pain céleste, accordez, Seigneur, nous vous en prions, cette force pour protéger la chasteté de nos corps et de nos âmes, que vous avez donnée merveilleusement à votre servante Marie. Par Notre-Seigneur.
\switchcolumn*

Die 11 julii
\switchcolumn
Le 11 juillet
\switchcolumn*

Solemnitas
\switchcolumn
Solennité de
\switchcolumn*

S. P. N. Benedicti
\switchcolumn
N. P. S. Benoît
\switchcolumn*

Ant. ad Introitum.\hfill Gn 12, 2
\switchcolumn
Antienne d’introït.
\switchcolumn*

Fáciam te in gentem magnam, et  benedícam tibi, et magnificábo nomen tuum, erísque benedíctus. √~Ps 102, 1. Bénedic, ánima mea, Dómino : et ómnia quæ intra me sunt nómino sancto ejus. √~Glória Patri. Fáciam te.
\switchcolumn
Je ferai de toi un grand peuple et je te  bénirai ; je rendrai grand ton nom et tu seras béni. √~Bénis le Seigneur, ô mon âme ; et tout ce qui est en moi son saint Nom. √~Gloire au Père. Je ferai de toi.
\switchcolumn*

Oratio
\switchcolumn
Collecte
\switchcolumn*

Deus, qui beatíssimum  Confessórem tuum Benedíctum, ómnium justórum spíritu replére dignátus es : concéde nobis fámulis tuis, ejus solemnitátem celebrántibus ; ut ejúsdem spíritu repléti, quod te donánte promísimus, fidéliter adimpleámus. Per Dóminum.
\switchcolumn
Ô Dieu qui avez daigné remplir votre  bienheureux confesseur Benoît de l’esprit de tous les justes, accordez-nous, à nous vos serviteurs qui célébrons sa solennité, que remplis de son esprit, nous accomplissions fidèlement ce que vous nous avez donné de promettre. Par Notre-Seigneur.
\switchcolumn*

Léctio libri Sapiéntiæ.
\switchcolumn
Lecture du livre de la Sagesse.
\switchcolumn*

Si 48, 1-2 ;49, 1-2
\switchcolumn

\switchcolumn*

Et surréxit quasi ignis, et verbum  ipsíus quasi fácula ardébat. Et quis potest sic simíliter gloriári tibi ? Qui sustulísti mórtuum ab ínferis de sorte mortis, in verbo Dómini Dei ; qui audis in Sina judícium, et ungis reges ad pæniténtiam, et prophétas facis successóres post te. Qui scriptus es in judíciis témporum leníre iracúndiam Dómini, conciliáre cor patris ad fílium, et restitúere tribus Jacob. Beáti sunt qui te vidérunt, et in amicítia tua decoráti sunt. Nam nos vita vívimus tantum, post mortem autem non erit tale nomen nostrum. Memória enim ejus in compositiónem odóris facta opus pigmentárii : in omni ore quasi mel indulcábitur ejus memória, et ut música in convívio vini.
\switchcolumn
Et il [Élie] se leva comme un feu, et sa  parole brûlait comme une torche. Qui peut se glorifier comme toi ? Par la parole du Seigneur Dieu tu as libéré un mort des enfers. Tu entends le jugement sur le Sinaï. Tu confères l’onction à des rois en vue du châtiment, et tu crées des prophètes pour te succéder. Il est écrit que tu dois adoucir la colère du Seigneur au temps du jugement, ramener le cœur du père vers le fils, et reconstituer les tribus de Jacob. Heureux ceux qui t’ont vu, et ont pu se glorifier de ton amitié. Car nous, nous ne vivons qu’en cette vie, et nous ne laisserons pas un tel nom après notre mort. [Josias] Car sa mémoire est une agréable odeur, composée par un parfumeur. Elle est comme du miel dans toute les bouches, et comme une musique dans un festin de vin.
\switchcolumn*

Graduale. √~Ps 20, 4-5. Dómine, prævenísti eum in benedictiónibus dulcédinis : posuísti in cápite ejus corónam de lápide pretióso. √~Vitam pétiit a te, et tribuísti ei longitúdinem diérum in sǽculum sǽculi.
\switchcolumn
Graduel. Seigneur, vous l’avez prévenu de douces bénédictions, vous avez mis sur sa tête une couronne de pierres précieuses. √~Il vous a demandé la vie, et vous lui avez accordé la longueur des jours pour les siècles des siècles.
\switchcolumn*

Allelúia, allelúia. √~Vir Dei Benedíctus ómnium justórum spíritu plenus fuit : ipse intercédat pro cunctis monásticæ professiónis.
\switchcolumn
Alléluia, alléluia. √~L’homme de Dieu Benoît était plein de l’esprit de tous les justes. Qu’il intercède pour tous ceux qui ont fait profession monastique.
\switchcolumn*

Sequentia
\switchcolumn
Séquence
\switchcolumn*

Læta dies magni Ducis,Dona ferens novæ lucis,Hodie recólitur.
\switchcolumn
Nous célébrons aujourd’hui l’heureux jour du grand Maître, jour qui nous apporte les dons d’une lumière nouvelle.
\switchcolumn*

Charis datur piæ menti,Corde sonet in ardéntiQuidquid foris prómitur.
\switchcolumn
La grâce est donnée à l’âme pieuse ; que retentisse dans un cœur brûlant tout ce qui est chanté extérieurement.
\switchcolumn*

Hunc per callem OriéntisAdmirémur ascendéntisPatriárchæ spéciem.
\switchcolumn
Admirons l’apparition du Patriarche, montant par le chemin de l’Orient.
\switchcolumn*

Amplum semen magnæ prolisIllum fecit instar solis,Abrahæ persímilem.
\switchcolumn
L’ample descendance de sa grande famille le rend égal au soleil, en tout semblable à Abraham.
\switchcolumn*

Corvum cernis ministrántem,Hinc Elíam latitántemSpecu nosce párvulo.
\switchcolumn
Tu vois le corbeau qui le sert, reconnais donc en lui Élie qui se cache dans une petite grotte.
\switchcolumn*

Elíseus dignoscátur,Cum secúris revocáturDe torréntis álveo.
\switchcolumn
On reconnaît Élisée, lorsque la hache est rappelée du lit du torrent.
\switchcolumn*

Illum Joseph candor morum,Illum Jacob futurórumMens effécit cónscia.
\switchcolumn
La pureté de ses mœurs a fait de lui Joseph, et son esprit connaissant l’avenir l’identifie à Jacob.
\switchcolumn*

Ipse memor suæ gentis,Nos perdúcat in manéntisSemper Christi gáudia. Amen. Allelúia.
\switchcolumn
Se souvenant de sa famille, qu’il nous conduise vers les joies du Christ qui demeure à jamais. Amen. Alléluia.
\switchcolumn*

Sequéntia sancti Evangélii secúndum Matthǽum.
\switchcolumn
Lecture du saint évangile selon saint Matthieu.
\switchcolumn*

Mt 19, 27-29
\switchcolumn

\switchcolumn*

In illo témpore : Dixit Petrus ad  Jesum : Ecce nos relíquimus ómnia, et secúti sumus te ; quid ergo erit nobis ? Jesus autem dixit illis : Amen dico vobis, quod vos, qui secúti estis me, in regeneratióne, cum séderit Fílius hóminis in sede majestátis suæ, sedébitis et vos super sedes duódecim, judicántes duódecim tribus Israel. Et omnis, qui relíquerit domum, vel fratres, aut soróres, aut patrem, aut matrem, aut uxórem, aut fílios, aut agros propter nomen meum, céntuplum accípiet, et vitam ætérnam possidébit.
\switchcolumn
En ce temps-là, Pierre dit à Jésus : Voici  que nous avons tout quitté et nous t’avons suivi. Qu’y aura-t-il donc pour nous ? Jésus leur dit : Amen je vous le dis : vous qui m’avez suivi, lorsque le Fils de l’homme, au jour de la régénération, trônera sur le trône de sa majesté, vous trônerez vous aussi sur douze trône, jugeant les douze tribus d’Israël. Et quiconque aura laissé une maison, des frères, des sœurs, un père, une mère, une femme, des fils ou des champs pour mon nom, il recevra le centuple, et il possèdera la vie éternelle.
\switchcolumn*

Credo.
\switchcolumn
Credo.
\switchcolumn*

Ant. ad offertorium.\hfill Ps 1, 3
\switchcolumn
Antienne d’offertoire.
\switchcolumn*

Tamquam lignum, quod plantátum est secus decúrsus aquárum, quod fructum suum dabit in témpore suo : et fólium ejus non défluet, et ómnia quæcúmque fáciet prosperabúntur.
\switchcolumn
Comme un arbre planté près des cours d’eau, qui donne du fruit en son temps. Ses feuilles ne tombent pas, et tout ce qu’il fait prospèrera.
\switchcolumn*

Secreta
\switchcolumn
Secrète
\switchcolumn*

Súscipe, omnípotens Deus, hæc  sacra múnera, quæ in beáti Patris nostri Benedícti Abbátis festivitáte tibi offérimus ; ut sicut illi amórem tuum exímium tribuísti, ita et in nobis ejus patrocínio divínæ caritátis flammas accéndas. Per Dóminum.
\switchcolumn
Recevez, Dieu tout-puissant, ces saintes  offrandes, que nous vous offrons en la fête de notre bienheureux Père Benoît, abbé ; de même que vous lui avez accordé un très grand amour pour vous, ainsi allumez en nous, par son intercession, les flammes de la divine charité. Par Notre-Seigneur.
\switchcolumn*

Præfatio
\switchcolumn
Préface
\switchcolumn*

Vere dignum et justum est, æquum et salutáre, nos tibi semper et ubíque grátias ágere : Dómine sancte, Pater omnípotens, ætérne Deus. Qui beatíssimum Confessórem tuum Benedíctum, Ducem et Magístrum cǽlitus edóctum, innumerábili multitúdini filiórum statuísti. Quem et ómnium justórum spíritu replétum, et extra se raptum, lúminis tuæ splendóre collustrásti. Ut in ipsa luce visiónis íntimæ, mentis laxáto sinu, quam angústa essent ómnia inferióra deprehénderet. Per Christum Dóminum nostrum. Quaprópter profúsis gáudiis, totus in orbe terrárum monachórum cœtus exsúltat. Sed et supérnæ virtútes, atque angélicæ potestátes, hymnum glóriæ tuæ cóncinunt, sine fine dicéntes : Sanctus.
\switchcolumn
Il est vraiment digne et juste, équitable et salutaire, de toujours et partout vous rendre grâces, Seigneur saint, Père tout-puissant, Dieu éternel, qui avez établi votre bienheureux confesseur Benoît comme chef et maître d’une multitude innombrable de fils, en l’instruisant d’en-haut. Vous l’avez rempli de l’esprit de tous les justes, et l’emportant hors de lui-même, vous l’avez illuminé de la splendeur de votre lumière. L’esprit dilaté, il comprit dans la lumière de cette intime vision combien petites sont toutes les choses d’ici-bas. Par le Christ Notre-Seigneur. C’est pourquoi dans cette effusion de joie, l’assemblée des moines exulte par toute la terre. De même les Vertus d’en-haut et les Puissances angéliques chantent l’hymne de votre gloire, disant sans fin : Saint.
\switchcolumn*

Ant. ad Comm.\hfill Si 44, 25-26
\switchcolumn
Antienne de communion.
\switchcolumn*

Benedictiónem ómnium géntium dedit illi Dóminus, et testaméntum confirmávit super caput ejus : agnóvit eum in benedictiónibus suis, et conservávit illi misericórdiam suam.
\switchcolumn
(Jacob) Le Seigneur lui a donné la bénédiction de tous les peuples, et il a confirmé son alliance sur sa tête. Il l’a connu dans ses bénédictions, et il lui a conservé sa miséricorde.
\switchcolumn*

Postcommunio
\switchcolumn
Postcommunion
\switchcolumn*

Divíni Sacraménti pasti delíciis,  te, Dómine, benedictiónum fons et orígo, súpplices exorámus ; ut per intercessiónem beatíssimi Patris nostri Benedícti, benedictiónis tuæ grátiam consequámur. Per Dóminum nostrum.
\switchcolumn
Nourris des délices du divin Sacrement,  nous vous adressons nos suppliantes prières, Seigneur, source et origine de toute bénédiction : par l’intercession de notre très saint Père Benoît, puissions-nous recevoir la grâce de votre bénédiction. Par Notre-Seigneur.
\switchcolumn*

Die 21 julii
\switchcolumn
21 juillet
\switchcolumn*

S. Laurentii de Brundusio
\switchcolumn
S. Laurent de Brindisi
\switchcolumn*

Conf. et Eccl. Doct.
\switchcolumn
Conf. et Doct. de l’Église
\switchcolumn*

Ant. ad Introitum.\hfill Si 42, 15-16
\switchcolumn
Antienne d’introït.
\switchcolumn*

In sermónibus Dómini ópera ejus :  sol illúminans per ómnia respéxit, et glória Dómini plenum est opus ejus. √~Ps 67, 2. Exsúrgat Deus, et dissipéntur inimíci ejus : et fúgiant, qui odérunt eum, a fácie ejus. √~Glória Patri. In sermónibus.
\switchcolumn
Dans les paroles du Seigneur sont ses ouvrages. Le brillant soleil regarde partout, et son œuvre est pleine de la gloire du Seigneur. √~Que Dieu se lève, et que ses ennemis soient dispersés, et que s’enfuient devant sa face ceux qui le haïssent. √~Gloire au Père. Dans les paroles du Seigneur.
\switchcolumn*

Oratio
\switchcolumn
Collecte
\switchcolumn*

Deus qui, ad árdua quæque pro  nóminis tui glória et animárum salúte, beáto Lauréntio Confessóri tuo atque Doctóri spíritum consílii et fortitúdinis contulísti : da nobis in eodem spíritu et agénda cognóscere ; et cógnita, ejus intercessióne, perfícere. Per Dóminum.
\switchcolumn
Dieu qui avez accordé au bienheureux  Laurent votre Confesseur un esprit de conseil et de force, afin qu’il se porte toujours au plus difficile pour la gloire de votre nom et le salut des âmes, donnez-nous, dans le même esprit, de connaître ce que nous devons faire, et d’accomplir, par son intercession, ce que nous aurons connu. Par Jésus-Christ.
\switchcolumn*

Léctio epístolæ beáti Pauli Apóstoli ad Corínthios
\switchcolumn
Lecture de la lettre du bienheureux Apôtre Paul aux Corinthiens.
\switchcolumn*

2 Co 5, 14-21
\switchcolumn

\switchcolumn*

Fratres : Cáritas Christi urget nos :  æstimántes hoc, quóniam si unus pro ómnibus mórtuus est, ergo omnes mórtui sunt : et pro ómnibus mórtuus est Christus : ut, et qui vivunt, jam non sibi vivant, sed ei qui pro ipsis mórtuus est et resurréxit. Itaque nos ex hoc néminem nóvimus secúndum carnem. Et si cognóvimus secúndum carnem Christum, sed nunc jam non nóvimus. Si qua ergo in Christo nova creatúra, vétera transiérunt : ecce facta sunt ómnia nova. Omnia autem ex Deo, qui nos reconciliávit sibi per Christum : et dedit nobis ministérium reconciliatiónis, quóniam quidem Deus erat in Christo mundum reconcílians sibi, non réputans illis delícta ipsórum, et pósuit in nobis verbum reconciliatiónis. Pro Christo ergo legatióne fúngimur, tamquam Deo exhortánte per nos. Obsecrámus pro Christo, reconciliámini Deo. Eum, qui non nóverat peccátum, pro nobis peccátum fecit, ut nos efficerémur justítia Dei in ipso.
\switchcolumn
Frères, l’amour du Christ nous presse, quand nous considérons que si un seul est mort pour tous, alors tous sont morts ; et le Christ est mort pour tous, afin que ceux qui vivent ne vivent plus pour eux-mêmes, mais pour celui qui est mort et qui est ressuscité pour eux. C’est pourquoi désormais nous ne connaissons plus personne selon la chair ; et si nous avons connu le Christ selon la chair, maintenant nous ne le connaissons plus ainsi. Si donc il y a dans le Christ une création nouvelle, les choses anciennes sont passées, voici que tout est devenu nouveau. Et tout vient de Dieu, qui nous a réconciliés avec lui-même par le Christ, et qui nous a confié le ministère de la réconciliation. Car c’est Dieu qui a réconcilié le monde avec lui dans le Christ, ne leur imputant point leurs péchés ; et c’est lui qui a mis en nous la parole de réconciliation. Nous nous acquittons donc pour le Christ de notre fonction d’ambassadeur, comme si Dieu exhortait par nous. Ainsi nous vous conjurons, au nom du Christ : réconciliez-vous avec Dieu. Pour nous il a fait péché celui qui ne connaissait point le péché, afin qu’en lui nous devenions justice de Dieu.
\switchcolumn*

Graduale. Ex 15, 2-3. Fortitúdo mea et laus mea Dóminus, et factus est mihi in salútem : iste Deus meus, et glorificábo eum. √~Dóminus quasi vir pugnátor : omnípotens nomen ejus.
\switchcolumn
Graduel. Ma force et ma louange, c’est le Seigneur, et il s’est fait mon salut. C’est mon Dieu et je le glorifierai. √~Le Seigneur est comme un guerrier, Tout-puissant est son nom.
\switchcolumn*

Allelúia, allelúia. √~Si 46, 6 Invocávit Altissímum poténtem in oppugnándo inimícos úndique : et audívit illum magnus et sanctus Deus. Allelúia.
\switchcolumn
Alléluia, alléluia. √~Il a invoqué le Très-Haut qui est puissant pour combattre les ennemis de tous côtés, et le Dieu grand et saint l’a écouté. Alléluia.
\switchcolumn*

Sequéntia sancti Evangélii secúndum Lucam.
\switchcolumn
Lecture du saint évangile selon saint Luc.
\switchcolumn*

Lc 9, 1-6
\switchcolumn

\switchcolumn*

In illo témpore : Convocátis Jesus  duódecim Apóstolis, dedit illis virtútem et potestátem super ómnia dæmónia, et ut languóres curárent. Et misit illos prædicáre regnum Dei, et sanáre infírmos. Et ait ad illos : Nihil tuléritis in via, neque virgam, neque peram, neque panem, neque pecúniam, neque duas túnicas habeátis. Et in quamcúmque domum intravéritis, ibi manéte, et inde ne exeátis. Et quicúmque non recéperint vos : exeúntes de civitáte illa, étiam púlverem pedum vestrórum excútite in testimónium supra illos. Egréssi autem circuíbant per castélla evangelizántes, et curántes ubíque.
\switchcolumn
En ce temps-là, Jésus ayant appelé ses  douze apôtres, leur donna puissance et autorité sur tous les démons, avec le pouvoir de guérir les maladies. Puis il les envoya prêcher le royaume de Dieu, et guérir les  malades. Et il leur dit : Ne prenez rien pour le chemin, ni bâton, ni sac, ni pain, ni argent, et n’ayez point deux tuniques. Dans toute maison où vous entrerez, demeurez, et n’en sortez point. Quant à ceux qui ne vous recevront pas, sortant de leur ville secouez même la poussière de vos pieds, en témoignage contre eux. Étant partis, ils allaient de village en village, annonçant l’Évangile, et opérant partout des guérisons.
\switchcolumn*

Ant. ad offertorium.\hfill Is 49, 2
\switchcolumn
Antienne d’offertoire.
\switchcolumn*

Pósuit os meum quasi gládium acútum : in umbra manus suæ protéxit me, et pósuit me sicut sagíttam eléctam.
\switchcolumn
Il a fait de ma bouche un glaive aiguisé. À l’ombre de sa main il m’a protégé, et il a fait de moi une flèche choisie.
\switchcolumn*

Secreta
\switchcolumn
Secrète
\switchcolumn*

Ad cæléste convívium fac nos, Deus,  salutáribus pœniténtiæ lácrimis dignos accédere : quod beáto Lauréntio vitæ candor suavíssimum efficiébat. Per Dóminum.
\switchcolumn
Faites-nous, ô Dieu, approcher avec les  larmes salutaires d’une digne pénitence de ce banquet céleste, que sa vie pure rendait si suave au bienheureux Laurent. Par Jésus-Christ.
\switchcolumn*

Ant. ad Comm.\hfill Sap 8, 11
\switchcolumn
Antienne de communion.
\switchcolumn*

In conspéctu poténtium admirábilis ero, et fácies príncipum mirabúntur me.
\switchcolumn
En présence des puissants je serai admirable, et les visages des princes m’admireront.
\switchcolumn*

Postcommunio
\switchcolumn
Postcommunion
\switchcolumn*

Divinitátis tuæ, Dómine, sempitérna  fruitióne satiémur : quam beátus Lauréntius in sacro altáris mystério prægustábat. Per Dóminum.
\switchcolumn
Rassasiez-nous, Seigneur, de la  jouissance éternelle de votre divinité, que le bienheureux Laurent goûtait d’avance dans le mystère sacré de l’autel. Par Jésus-Christ.
\switchcolumn*

Die 26 julii
\switchcolumn
26 juillet
\switchcolumn*

SS. Joachim et Annæ
\switchcolumn
S. Joachim et S. Anne
\switchcolumn*

Parentum
\switchcolumn
Parents
\switchcolumn*

Beatæ Mariæ Virginis
\switchcolumn
de la B. V. Marie
\switchcolumn*

Ant. ad Introitum.
\switchcolumn
Antienne d’introït.
\switchcolumn*

Gaudeámus omnes in Dómino,  diem festum celebrántes sub honóre beatórum Jóachim et Annæ : de quorum solemnitáte gaudent Angeli et colláudant Fílium Dei. √~Ps 127, 1. Beáti omnes qui timent Dóminum, qui ámbulant in viis ejus. √~Glória Patri. Gaudeámus.
\switchcolumn
Réjouissons-nous tous dans le Seigneur,  en célébrant ce jour de fête en l’honneur des bienheureux Joachim et Anne ; les Anges se réjouissent de cette solennité et louent le Fils de Dieu. √~Heureux tous ceux qui craignent le Seigneur, qui marchent dans ses voies. √~Gloire au Père. Réjouissons-nous.
\switchcolumn*

Oratio
\switchcolumn
Collecte
\switchcolumn*

Deus qui sanctos Jóachim et  Annam, gloriósæ Matris Unigéniti tui paréntes elegísti, da nobis eórum précibus, in tuórum consórtio electórum cleméntiam tuam perpétuo collaudáre. Per eúmdem Dóminum.
\switchcolumn
Ô Dieu qui avez choisi les saints Joachim  et Anne pour être les parents de la glorieuse Mère de votre Fils unique, donnez-nous, par leurs prières, de pouvoir en la société de vos saints louer éternellement votre clémence. Par le même Jésus-Christ.
\switchcolumn*

Léctio libri Regum.
\switchcolumn
Lecture du Livre des Rois.
\switchcolumn*

2 R (2 Sam) 7, 4-16
\switchcolumn

\switchcolumn*

In diébus illis : Ecce sermo Dómini  ad Nathan, dicens : Vade et lóquere ad servum meum David : Hæc dicit Dóminus : Ego tuli te de páscuis sequéntem greges, ut esses dux super pópulum meum Israel : et fui tecum in ómnibus ubicúmque ambulásti, et interféci univérsos inimícos tuos a fácie tua : fecíque tibi nomen grande, juxta nomen magnórum qui sunt in terra. Et ponam locum pópulo meo Israel, et réquiem dabo tibi ab ómnibus inimícis tuis ; prædicítque tibi Dóminus, quod domum fáciat tibi Dóminus. Cumque compléti fúerint dies tui, et dormíeris cum pátribus tuis, suscitábo semen tuum post te, quod egrediétur de útero tuo, et firmábo regnum ejus : ipse ædificábit domum nómini meo, et stabíliam thronum ejus usque in sempitérnum. Et ego ero ei in patrem , et ipse erit mihi in fílium : qui si iníque áliquid gésserit, árguam eum in virga virórum, et in plagis filiórum hóminum. Misericórdiam autem meam non áuferam ab eo, sicut ábstuli a Saul, quem amóvi a fácie mea. Et fidélis erit domus tua, et regnum tuum usque in ætérnum ante fáciem tuam, et thronus tuus erit firmus júgiter.
\switchcolumn
En ces jours-là, la parole du Seigneur fut  adressée à Nathan : Va dire à mon serviteur David : Ainsi parle le Seigneur : Je t’ai pris au pâturage de derrière les brebis, pour être prince sur mon peuple Israël. J’ai été avec toi partout où tu es allé, j’ai exterminé tous tes ennemis devant toi et je t’ai fait un grand nom, semblable au nom des grands de la terre. J’ai assigné un lieu à mon peuple Israël. Je t’ai accordé du repos en te délivrant de tous tes ennemis. Le Seigneur t’annonce en outre qu’il te fera une maison. Quand tes jours seront accomplis et que tu seras couché avec tes pères, je susciterai après toi la postérité qui sortira de tes entrailles, et j’affermirai son règne. C’est lui qui bâtira une maison à mon nom, et j’affermirai pour toujours son trône. Je serai pour lui un père, et il sera pour moi un fils. S’il fait le mal, je le châtierai avec une verge d’homme et des coups de fils d’homme. Mais ma grâce ne se retirera point de lui, come je l’ai retirée de Saül, que j’ai ôté de devant ma face. Ta maison et ton règne seront pour toujours assurés devant toi : ton trône sera ferme pour toujours.
\switchcolumn*

Graduale. Ps 149, 5. Exsultábunt sancti in glória : lætabúntur in cubílibus suis. √~Ibid. 1. Cantáte Dómino cánticum novum : laus ejus in ecclésia sanctórum.
\switchcolumn
Graduel. Les saints exulteront dans la gloire, ils se réjouiront sur leur couche. √~Chantez au Seigneur un chant nouveau, sa louange dans l’assemblée des saints.
\switchcolumn*

Allelúia, allelúia. √~O Jóachim sancte, conjux Annæ, pater almæ Vírginis, hic fámulis confer salútis opem. Allelúia.
\switchcolumn
Alléluia, alléluia. √~Ô saint Joachim, époux d’Anne, père de la sainte Vierge, donnez à vos serviteurs ici présents un secours qui les sauve. Alléluia.
\switchcolumn*

Inítium sancti Evangélii secúndum Matthǽum.
\switchcolumn
Commencement du saint évangile selon saint Matthieu.
\switchcolumn*

Mt 1, 1-16
\switchcolumn

\switchcolumn*

Liber generatiónis Jesu Christi fílii  David, fílii Abraham. Abraham génuit Isaac. Isaac autem génuit Jacob. Jacob autem génuit Judam et fratres ejus. Judas autem génuit Phares et Zaram de Thamar. Phares autem génuit Esron. Esron autem génuit Aram. Aram autem génuit Amínadab. Amínadab autem génuit Naásson. Naásson autem génuit Salmon. Salmon autem génuit Booz de Rahab. Booz autem génuit Obed ex Ruth. Obed autem génuit Jesse. Jesse autem génuit David regem. David autem rex génuit Salomónen ex ea quæ fuit Uríæ. Sálomon autem génuit Róboam. Róboam autem génuit Abíam. Abías autem génuit Asa. Asa autem génuit Jósaphat. Jósaphat autem génuit Joram. Joram autem génuit Ozíam. Ozías autem génuit Jóatham. Jóatham autem génuit Achaz. Achaz autem génuit Ezechíam. Ezechías autem génuit Manássen. Manásses autem génuit Amon. Amon autem génuit Josíam. Josías autem génuit Jechoníam, et fratres ejus in transmigratióne Babylónis. Et post transmigratiónem Babylónis : Jechonías génuit Saláthiel. Saláthiel autem génuit Zoróbabel. Zoróbabel autem génuit Abiud. Abiud autem génuit Elíacim. Elíacim autem génuit Azor. Azor autem génuit Sadoc. Sadoc autem génuit Achim. Achim autem génuit Eliud. Eliud autem génuit Eleázar. Eleázar autem génuit Mathan. Mathan autem génuit Jacob. Jacob autem génuit Joseph, virum Maríæ, de qua natus est Jesus, qui vocátur Christus.
\switchcolumn
Livre de la génération de Jésus-Christ,  fils de David, fils d’Abraham. Abraham engendra Isaac. Isaac engendra Jacob. Jacob engendra Judas et ses frères. Judas engendra Pharès et Zara, de Thamar. Pharès engendra Esron. Esron engendra Aram. Aram engendra Aminadab. Aminadab engendra Naason. Naason engendra Salmon. Salmon engendra Booz, de Rahab. Booz engendra Obed, de Ruth. Obed engendra Jessé. Jessé engendra le roi David. Le roi David engendra Salomon, de celle qui fut l’épouse d’Urie. Salomon engendra Roboam. Roboam engendra Abia. Abia engendra Asa. Asa engendra Josaphat. Josaphat engendra Joram. Joram engendra Ozias. Ozias engendra Joatham. Joatham engendra Achaz. Achaz engendra Ézéchias. Ézéchias engendra Manassé. Manassé engendra Amon. Amon engendra Josias. Josias engendra Jéchonias et ses frères, lors de la captivité à Babylone. Et après la captivité à Babylone, Jéchonias engendra Salathiel. Salathiel engendra Zorobabel. Zorobabel engendra Abiud. Abiud engendra Éliakim. Éliakim engendra Azor. Azor engendra Sadoc. Sadoc engendra Achim. Achim engendra Éliud. Éliud engendra Éleazar. Éleazar engendra Mathan. Mathan engendra Jacob. Jacob engendra Joseph, l’époux de Marie, de laquelle est né Jésus, qui est appelé Christ.
\switchcolumn*

Ant. ad offertorium.\hfill Ps 31, 11
\switchcolumn
Antienne d’offertoire.
\switchcolumn*

Lætámini in Dómino et exsultáte justi : et gloriámini omnes recti corde.
\switchcolumn
Réjouissez-vous dans le Seigneur et exultez, justes ; soyez glorifiés, vous tous qui avez le cœur droit.
\switchcolumn*

Secreta
\switchcolumn
Secrète
\switchcolumn*

Sacrifíciis præséntibus, quǽsumus,  Dómine, placátus inténde : ut per intercessiónem beatórum Jóachim et Annæ, qui Matris Unigéniti tui paréntes exstitérunt, et devotióni nostræ profíciant et salúti. Per eúndem Dóminum nostrum.
\switchcolumn
Nous vous en supplions, Seigneur, re gardez favorablement ce sacrifice que voici, afin que par l’intercession des bienheureux Joachim et Anne, qui furent les parents de la Mère de votre Fils, il serve à l’accroissement de notre dévotion et à notre salut. Par le même Jésus-Christ.
\switchcolumn*

Ant. ad Comm.\hfill Ba 5, 5 et 4, 36
\switchcolumn
Antienne de communion.
\switchcolumn*

Jerúsalem, surge, et sta in excélso : et vide jucunditátem, quæ véniet tibi a Deo tuo.
\switchcolumn
Lève-toi, Jérusalem, et tiens-toi sur la hauteur ; vois la joie qui te vient de ton Dieu.
\switchcolumn*

Postcommunio
\switchcolumn
Postcommunion
\switchcolumn*

Córporis sacri et pretiósi sánguinis  tui repléti libámine, quǽsumus, Dómine Deus noster : ut quod pia devotióne gérimus, patrocíniis beatórum Jóachim et Annæ certa redemptióne capiámus. Qui vivis et regnas.
\switchcolumn
Remplis par la réception de votre saint  Corps et de votre précieux Sang, nous vous en prions, Seigneur : ce que nous célébrons avec une sainte dévotion, puissions-nous, par les suffrages des bienheureux Joachim et Anne, l’obtenir par une rédemption certaine. Vous qui vivez.
\switchcolumn*

Die 4 augusti
\switchcolumn
Le 4 août
\switchcolumn*

S. Ioannis M. Vianney
\switchcolumn
S. Jean-Marie Vianney
\switchcolumn*

Confessoris
\switchcolumn
Confesseur
\switchcolumn*

Ant. ad Introitum.\hfill Ga 6, 14
\switchcolumn
Antienne d’introït.
\switchcolumn*

Mihi absit gloriári, nisi in cruce  Dómini nostri Jesu Christi, per quem mihi mundus crucifíxus est, et ego mundo. √~Ps 20, 2. In te Dómine sperávi, non confúndar in ætérnum : in justítia tua líbera me. √~Glória Patri. Mihi absit.
\switchcolumn
Pour moi, que jamais je ne me glorifie  que dans la croix de Notre-Seigneur Jésus-Christ, par qui le monde est crucifié pour moi, et moi pour le monde. √~En vous Seigneur j’ai espéré, je ne serai pas confondu pour toujours ; dans votre justice, libérez-moi. √~Gloire au Père. Pour moi.
\switchcolumn*

Oratio
\switchcolumn
Collecte
\switchcolumn*

Omnípotens et miséricors Deus,  qui sanctum Joánnem Maríam pastoráli stúdio et jugi oratiónis ac pœniténtiæ ardóre mirábilem effecísti : da, quǽsumus, ut ejus exémplo et intercessióne, ánimas fratrum lucrári Christo, et cum eis ætérnam glóriam cónsequi valeámus. Per eúmdem Dóminum.
\switchcolumn
Dieu tout-puissant et miséricordieux,  qui avez rendu saint Jean-Marie admirable par son zèle pastoral et son incessante ardeur pour la prière et la pénitence, accordez-nous, nous vous en prions, que par son exemple et son inter-cession, nous puissions gagner au Christ les âmes de nos frères, et obtenir avec eux la gloire éternelle. Par le même Jésus-Christ.
\switchcolumn*

Léctio Ezechiélis Prophétæ.
\switchcolumn
Lecture du prophète Ézéchiel.
\switchcolumn*

Ez 33, 7. 10-12
\switchcolumn

\switchcolumn*

Et tu, fili hóminis, speculatórem  dedi te dómui Israel. Audiens ergo ex ore meo sermónem, annuntiábis eis ex me. Tu ergo, fili hóminis, dic ad domum Israel : Sic locúti estis, dicéntes : Iniquitátes nostræ et peccáta nostra super nos sunt, et in ipsis nos tabéscimus : quómodo ergo vívere potérimus ? Dic ad eos : Vivo ego, dicit Dóminus Deus : nolo mortem ímpii, sed ut convertátur ímpius a via sua et vivat. Convertímini, convertímini a viis vestris péssimis : et quare moriémini, domus Israel ? Tu ítaque, fili hóminis, dic ad fílios pópuli tui : Impíetas ímpii non nocébit ei in quacúmque die convérsus fúerit ab impietáte sua.
\switchcolumn
Et toi, fils d’homme, j’ai fait de toi une  sentinelle pour la maison d’Israël. Quand tu entendras une parole sortie de ma bouche, tu la leur annonceras de ma part. Toi donc, fils d’homme, dis à la maison d’Israël : Vous avez parlé ainsi : Nos iniquités et nos péchés sont sur nous, et nous périssons à cause d’eux. Comment donc pourrons-nous vivre ? Tu leur diras : Aussi vrai que je suis vivant, dit le Seigneur Dieu, je ne veux pas la mort de l’impie, mais que l’impie se détourne de sa voie et qu’il vive. Revenez, revenez de vos voies mauvaises. Pourquoi devriez-vous mourir, maison d’Israël ? Toi donc, fils d’homme, dis aux fils de ton peuple : L’impiété de l’impie ne lui sera plus nuisible à partir du jour où il se sera détourné de son impiété.
\switchcolumn*

Graduale. Ps 44, 2. Eructávit cor meum verbum bonum : dico ego ópera mea Regi. √~Ps 38, 4. Concáluit cor meum intra me : et in meditatióne mea exardéscet ignis.
\switchcolumn
Graduel. De mon cœur sort une belle parole, je dis mes œuvres au Roi. √~Mon cœur s’est réchauffé au-dedans de moi, et dans ma méditation un feu s’embrasera.
\switchcolumn*

Allelúia, allelúia. √~Si 48, 1. Surréxit quasi ignis, et verbum ipsíus quasi fácula ardébat. Allelúia.
\switchcolumn
Alléluia, alléluia. √~[Élie] s’est levé comme un feu, et sa parole brûlait comme une torche. Alléluia.
\switchcolumn*

Sequéntia sancti Evangélii secúndum Matthǽum
\switchcolumn
Lecture du saint évangile selon saint Matthieu
\switchcolumn*

Mt 9, 35-38 ; 10,1
\switchcolumn

\switchcolumn*

In illo témpore : Circuíbat Jesus  omnes civitátes et castélla, docens in synagógis eórum et prǽdicans Evangélium regni, et curans omnem languórem et omnem infirmitátem. Videns autem turbas, misértus est eis, quia erant vexáti et jacéntes sicut oves non habéntes pastórem. Tunc dicit discípulis suis : Messis quidem multa, operárii autem pauci. Rogáte ergo Dóminum messis, ut mittat operários in messem suam. Et convocátis duódecim discípulis suis, dedit illis potestátem spirítuum immundórum, ut ejícerent eos, et curárent omnem languórem et omnem infirmitátem.
\switchcolumn
En ce temps-là, Jésus passait dans toutes  les villes et les villages, enseignant dans leurs synagogues, prêchant l’Évangile du Royaume et guérissant toute maladie et toute infirmité. Voyant les foules il eut pitié d’elles, car elles étaient tourmentées et gisantes comme des brebis sans pasteur. Alors il dit à ses disciples : La moisson est abondante, mais les ouvriers peu nombreux. Priez donc le Maître de la moisson d’envoyer des ouvriers dans sa moisson. Et ayant appelé ses douze disciples, il leur donna pouvoir sur les esprits impurs afin qu’ils les chassent, et qu’ils guérissent toute maladie et toute infirmité.
\switchcolumn*

Ant. ad Offertorium.\hfill Col 1, 24-25
\switchcolumn
Antienne d’offertoire.
\switchcolumn*

Gáudeo in passiónibus, et adímpleo ea quæ desunt passiónum Christi in carne mea, pro córpore ejus quod est Ecclésia, cujus factus sum ego miníster.
\switchcolumn
Je me réjouis dans mes souffrances et je complète ce qui manque aux souffrances du Christ en ma chair, pour son corps qui est l’Église, dont j’ai été fait le ministre.
\switchcolumn*

Secreta
\switchcolumn
Secrète
\switchcolumn*

Super hanc illibátam hóstiam,  omnípotens sempitérne Deus, descéndat invisíbilis plenitúdo Spíritus Sancti, et præsta, ut intercedénte beáto Joánne María, casto córpore et mundo corde ad tantum semper mystérium accedámus. Per Dóminum.
\switchcolumn
Sur cette hostie sans tache, ô Dieu  tout-puissant, que descende l’invisible plénitude du Saint-Esprit, et accordez-nous que, par l’intercession du bienheureux Jean-Marie, nous nous approchions toujours d’un tel mystère avec un corps chaste et un cœur pur. Par Notre-Seigneur.
\switchcolumn*

Ant. ad Comm.\hfill Lc 6, 18-19
\switchcolumn
Antienne de communion.
\switchcolumn*

Multitúdo languéntium et qui vexabántur a spirítibus immúndis veniébant ad Jesum, quia virtus de illo exíbat et sanábat omnes.
\switchcolumn
La multitude des malades et ceux qui étaient tourmentés par des esprits impurs venaient vers Jésus, car une force sortait de lui et les guérissait tous.
\switchcolumn*

Postcommunio
\switchcolumn
Postcommunion
\switchcolumn*

Angelórum dape refécti, te Dómine  deprecámur, ut sicut in fortitúdine hujus panis beátus Joánnes María advérsa ómnia invícta constántia tolerávit, ita nos ejus méritis et imitatióne, de virtúte in virtútem eúntes ad te felíciter perducámur. Per Dóminum.
\switchcolumn
Réconfortés par le festin des anges, nous  vous en prions, Seigneur : de même que le bienheureux Jean-Marie trouva dans de ce pain la force de supporter avec une constance invincible toutes les adversités, ainsi puissions-nous être conduits heureusement vers vous en marchant de vertu en vertu. Par Notre-Seigneur.
\switchcolumn*

Die 19 augusti
\switchcolumn
Le 19 août
\switchcolumn*

S. Bernardi Tolomæi
\switchcolumn
S. Bernard Tolomei
\switchcolumn*

Abbatis
\switchcolumn
Abbé
\switchcolumn*

Ant. ad Introitum.\hfill Ps 54, 8-10
\switchcolumn
Antienne d’introït.
\switchcolumn*

Ecce elongávi fúgiens et mansi  in solitúdine ; quóniam vidi iniquitátem et contradictiónem in civitáte. √~Ps 54, 2-3. Exaúdi, Deus, oratiónem meam, et ne despéxeris deprecatiónem meam : inténde mihi et exaúdi me. √~Glória Patri. Ecce elongávi.
\switchcolumn
Voici que je me suis éloigné en fuyant,  et j’ai demeuré au désert ; car j’ai vu l’iniquité et la contradiction dans la ville. √~Exaucez, ô Dieu, ma prière, et ne méprisez pas ma supplication : Écoutez-moi, et exaucez-moi. √~Gloire au Père. Voici que je me suis éloigné.
\switchcolumn*

Oratio
\switchcolumn
Collecte
\switchcolumn*

Deus, qui beátum Bernárdum  Abbátem a sǽculi pompa ad amó-rem solitúdinis evocátum in miseránda pópuli contagióne, caritátis víctimam effecísti : quǽsumus, ut per ejus vestígia gradiéntes, ejúsdem caritátis et glóriæ tríbuas esse consórtes. Per Dóminum.
\switchcolumn
Dieu, qui avez fait du bienheureux abbé  Bernard, appelé des vanités du siècle à l’amour de la solitude, une victime de la charité qu’il a exercée envers la population atteinte de la peste, faites que, marchant sur ses traces, nous méritions de partager sa charité et sa gloire. Par Notre-Seigneur.
\switchcolumn*

Léctio Epístolæ beáti Pauli Apóstoli ad Philippénses.
\switchcolumn
Lecture de la lettre du bienheureux Apôtre Paul aux Philippiens.
\switchcolumn*

Ph 2, 12-18
\switchcolumn

\switchcolumn*

Caríssimi : sicut semper obedístis,  non ut in præséntia mei tantum, sed multo magis nunc in abséntia mea, cum metu et tremóre vestram salútem operámini. Deus est enim, qui operátur in vobis et velle et perfícere pro bona voluntáte. Omnia autem fácite sine murmuratiónibus et hæsitatiónibus : ut sitis sine queréla, et símplices fílii Dei, sine reprehensióne in médio natiónis pravæ et pervérsæ : inter quos lucétis sicut luminária in mundo, verbum vitæ continéntes ad glóriam meam in die Christi, quia non in vácuum cucúrri, neque in vácuum laborávi. Sed et si ímmolor supra sacríficium, et obséquium fídei vestræ, gáudeo, et congrátulor ómnibus vobis. Idípsum autem et vos gaudéte, et congratulámini mihi.
\switchcolumn
Mes bien-aimés, comme vous avez tou jours été obéissants, ayez soin, non seulement en ma présence, mais beaucoup plus maintenant en mon absence, d’opérer votre salut avec crainte et tremblement. Car c’est Dieu qui opère en vous et le vouloir et le faire, selon son bon plaisir. Faites toutes choses sans murmures et sans hésitations,  afin que vous soyez irrépréhensibles et des enfants de Dieu sincères et sans tache au milieu d’une nation dépravée et perverse, parmi laquelle vous brillez comme des astres dans le monde ; tenant ferme la parole de vie, en sorte que je puisse me glorifier, au jour du Christ, de n’avoir pas couru en vain, ni travaillé en vain. Mais, dussé-je servir de libation pour le sacrifice et l’offrande de votre foi, je m’en réjouis, et je vous en félicite tous. Vous aussi, réjouissez-vous, et félicitez-moi.
\switchcolumn*

Graduale. Ps 51, 10. Ego autem sicut olíva fructífera in domo Dei : sperávi in misericórdia Dei in ætérnum, et in sǽculum sǽculi. √~Ps 83, 5. Beáti qui hábitant in domo tua Dómine ; in sǽcula sæculórum laudábunt te.
\switchcolumn
Graduel. Mais moi, je suis comme un olivier fertile dans la maison de Dieu. J’espère en la miséricorde de Dieu éternellement et à jamais.√~Heureux ceux qui habitent dans votre maison, Seigneur ; ils vous loueront dans les siècles des siècles.
\switchcolumn*

Allelúia, allelúia. √~Duxit Dóminus Bernárdum in solitúdinem, et locútus est ad cor ejus. Allelúia.
\switchcolumn
Alléluia, alléluia. √~Le Seigneur a conduit Bernard dans la solitude, et il a parlé à son cœur. Alléluia.
\switchcolumn*

Sequéntia sancti Evangélii secúndum Matthǽum.
\switchcolumn
Lecture du saint évangile selon saint Matthieu.
\switchcolumn*

Mt 19, 27-29
\switchcolumn

\switchcolumn*

In illo témpore : Dixit Petrus ad  Jesum : Ecce nos relíquimus ómnia, et secúti sumus te : quid ergo erit nobis ? Jesus autem dixit illis : Amen dico vobis, quod vos, qui secúti estis me, in regeneratióne cum séderit Fílius hóminis in sede majestátis suæ, sedébitis et vos super sedes duódecim, judicántes duódecim tribus Israel. Et omnis qui relíquerit domum, vel fratres, aut soróres, aut patrem, aut matrem, aut uxórem, aut fílios, aut agros propter nomen meum, céntuplum accípiet, et vitam ætérnam possidébit.
\switchcolumn
En ce temps là : Pierre, prenant la parole,  dit à Jésus : Voici que nous avons tout quitté et que nous vous avons suivi ; qu’y aura-il-donc pour nous ? Jésus leur dit : En vérité je vous le dis, vous qui m’avez suivi, lorsque, au temps de la régénération, le Fils de l’homme siégera su-r le trône de sa gloire, vous siégerez, vous aussi, sur douze trônes, et vous jugerez les douze tribus d’Israël.  Et quiconque aura quitté sa maison, ou ses frères, ou ses soeurs, ou son père, ou sa mère, ou sa femme, ou ses enfants, ou ses champs, à cause de mon nom, recevra le centuple,et possédera la vie éternelle.
\switchcolumn*

Ant. ad Offertorium.\hfill Ps 118, 47.48
\switchcolumn
Antienne d’offertoire.
\switchcolumn*

Meditábar in mandátis tuis, quæ diléxi ; et exercébar in justificatiónibus tuis.
\switchcolumn
Je méditais sur vos commandements, car je les aime : et je m’exerçais dans vos ordonnances.
\switchcolumn*

Secreta
\switchcolumn
Secrète
\switchcolumn*

Ut tuis, Dómine, digne famulémur  altáribus, beáti Bernárdi Patris nostri fac hortaménta sectémur et imitémur exémpla, in cujus honórem hanc immaculátam hóstiam tuæ offérimus majestáti. Per Dóminum.
\switchcolumn
Afin que nous servions dignement  vos autels, faites, Seigneur, que nous suivions les exhortations du bienheureux Bernard, notre Père, en l’honneur duquel nous offrons cette hostie immaculée à votre divine Majesté,  et que nous imitions ses exemples. Par Notre-Seigneur.
\switchcolumn*

Ant. ad Comm.\hfill Ps 83, 11
\switchcolumn
Antienne de communion.
\switchcolumn*

Elégi abjéctus esse in domo Dei mei, magis quam habitáre in tabernáculis peccatórum.
\switchcolumn
J’ai choisi d’être des derniers dans la maison de mon Dieu, plutôt que d’habiter dans les tentes des pécheurs.
\switchcolumn*

Postcommunio
\switchcolumn
Postcommunion
\switchcolumn*

Cœlésti convívio recreáti, súpplices  te Dómine deprecámur : ut cáritas, quæ in corde beáti Bernárdi júgiter ardébat, eódem interveniénte in córdibus quoque nostris lárgiter diffundátur. Per Dóminum.
\switchcolumn
Recréés par ce céleste banquet, nous  vous supplions, Seigneur : que la charité qui brûlait continuellement dans le cœur du bienheureux Bernard, se diffuse largement, par son intercession, aussi dans nos cœurs. Par Notre-Seigneur.
\switchcolumn*

Die 20 augusti
\switchcolumn
Le 20 août
\switchcolumn*

S. Bernardi
\switchcolumn
S. Bernard
\switchcolumn*

Abb. et Eccl. Doct.
\switchcolumn
Abbé et Docteur de l’Église
\switchcolumn*

Ant. ad Introitum.\hfill Si 15, 5
\switchcolumn
Antienne d’introït.
\switchcolumn*

In médio Ecclésiæ apéruit os ejus :  et implévit eum Dóminus spíritu sapiéntiæ et intelléctus : stolam glóriæ índuit eum. √~Ps 91, 2. Bonum est confitéri Dómino : et psállere nómini tuo, Altíssime. √~Glória Patri. In médio.
\switchcolumn
Au milieu de l’Église, le Seigneur a ouvert sa bouche et il l’a rempli de l’esprit de sagesse et d’intelligence ; il l’a revêtu du vêtement de gloire. √~Il est bon de célébrer le Seigneur, et de psalmodier pour votre Nom, ô Très-Haut. √~Gloire au Père. Au milieu.
\switchcolumn*

Oratio
\switchcolumn
Collecte
\switchcolumn*

Perfice, quǽsumus, Dómine, pium in  nobis sanctæ religiónis afféctum : et ad obtinéndam  tuæ grátiæ largitátem, beátus Bernárdus, Abbas et Doctor egrégius, suis apud te semper pro nobis méritis et précibus intercédat. Per Dóminum.
\switchcolumn
Portez en nous à sa perfection, nous vous  en prions, Seigneur, un sentiment de dévotion et de religion, et pour que nous obtenions les largesses de votre grâce, que le bienheureux Bernard, Abbé et Docteur éminenent, intercède toujours pour nous auprès de vous par ses mérites et ses prières. Par Notre-Seigneur.
\switchcolumn*

Léctio libri Sapiéntiæ.
\switchcolumn
Lecture du livre de la Sagesse.
\switchcolumn*

Si 39, 6-14
\switchcolumn

\switchcolumn*

Justus cor suum tradet ad  vigilándum dilúculo ad Dóminum, qui fecit illum, et in conspéctu Altíssimi deprecábitur. Apériet os suum in oratióne et pro delíctis suis deprecábitur. Si enim Dóminus magnus volúerit, spíritu intellegéntiæ replébit illum : et ipse tamquam imbres mittet elóquia sapiéntiæ suæ, et in oratióne confitébitur Dómino : et ipse díriget consílium ejus et disciplínam, et in abscónditis suis consiliábitur. Ipse palam faciet disciplínam doctrínæ suæ, et in lege testaménti Dómini gloriábitur. Collaudábunt multi sapiéntiam ejus, et usque in sǽculum non delébitur. Non recédet memória ejus, et nomen ejus requirétur a generatióne in generatiónem. Sapiéntiam ejus enarrábunt gentes, et laudem ejus enuntiábit ecclésia.
\switchcolumn
Le juste occupera son cœur dès le matin  à se tourner vers le Seigneur qui le créa, et en présence du Très-Haut il fera monter sa prière. Il ouvrira sa bouche dans la prière, et il suppliera pour ses péchés. Car si le Seigneur qui est grand le veut, il le remplira de l’esprit d’intelligence. Quant à lui, il répandra comme une pluie les paroles de sa sagesse, et dans sa prière il célébrera le Seigneur. Il acquerra la droiture du jugement et de la connaissance, et il méditera sur ses mystères cachés. Il fera paraître l’instruction qu’il a reçue, et il se glorifiera dans la loi de l’alliance du Seigneur. Beaucoup loueront sa sagesse, et il ne sera pas effacé à jamais. Sa mémoire ne sera pas oubliée, et son nom sera recherché de génération en génération. Les peuples raconteront sa sagesse, et l’assemblée prononcera sa louange.
\switchcolumn*

Graduale. Sap 8, 2. Sapiéntiam amávi, et exquisívi a juventúte mea, et quǽsivi sponsam mihi eam assúmere, et amátor factus sum formæ illíus. √~Ibid. 10.11. Habébo propter sapiéntiam claritátem ad turbas, et in conspéctu poténtium admirábilis ero, et fácies príncipum mirabúntur me.
\switchcolumn
Graduel. J’ai aimé la sagesse et je l’ai recherchée depuis ma jeunesse ; j’ai cherché à en faire mon épouse, et je suis devenu amoureux de sa beauté. √~Grâce à la sagesse, j’aurai la gloire devant les foules, et en présence des puissants je serai admirable, et les visages des princes m’admireront.
\switchcolumn*

Allelúia, allelúia. √~Caritáte vulnerátus, castitáte dealbátus, verbo vitæ laureátus est Bernárdus, sublimátus in glória. Allelúia.
\switchcolumn
Alléluia, alléluia. √~Blessé d’amour, rayonnant de chasteté, couronné par la parole de vie, tel est Bernard, élevé dans la gloire. Alléluia.
\switchcolumn*

Sequéntia sancti Evangélii secúndum Matthǽum.
\switchcolumn
Lecture du saint évangile selon saint Matthieu.
\switchcolumn*

Mt 5, 13-19
\switchcolumn

\switchcolumn*

In illo témpore : Dixit Jesus discípulis  suis : Vos estis sal terræ. Quod si sal evanúerit, in quo saliétur ? Ad níhilum valet ultra, nisi ut mittátur foras, et conculcétur ab homínibus. Vos estis lux mundi. Non potest cívitas abscóndi supra montem pósita. Neque accéndunt lucérnam, et ponunt eam sub módio, sed super candelábrum, ut lúceat ómnibus qui in domo sunt. Sic lúceat lux vestra coram homínibus, ut vídeant ópera vestra bona, et gloríficent Patrem vestrum, qui in cælis est. Nolíte putáre, quóniam veni sólvere legem, aut prophétas : non veni sólvere, sed adimplére. Amen quippe dico vobis, donec tránseat cælum et terra, iota unum, aut unus apex non præteríbit a lege, donec ómnia fiant. Qui ergo sólverit unum de mandátis istis mínimis, et docúerit sic hómines, mínimus vocábitur in regno cælórum : qui autem fécerit, et docúerit, hic magnus vocábitur in regno cælórum.
\switchcolumn
En ce temps-là, Jésus dit à ses disciples :  Vous êtes le sel de la terre. Si le sel s’affadit, avec quoi le salera-t-on ? Il n’est plus bon à rien, sinon à être jeté dehors et piétiné par les hommes. Vous êtes la lumière du monde. Une ville placée sur une montagne ne peut être cachée. Et quand on allume une lampe, ce n’est pas pour la placer sous le boisseau mais sur le candélabre, afin qu’elle brille pour tous ceux qui sont dans la maison. Qu’ainsi votre lumière brille devant les hommes, afin qu’ils voient vos bonnes œuvres et qu’ils glorifient votre Père qui est dans les cieux. Ne pensez pas que je sois venu abolir la loi ou les prophètes. Je ne suis pas venu abolir, mais accomplir. Amen je vous le dis : Jusqu’à ce que passent le ciel et la terre, pas un iota ou un trait de la loi ne passera : tout sera accompli. Celui donc qui abolira l’un de ces plus petits commandements et enseignera aux hommes à faire ainsi, sera appelé le plus petit dans le royaume des cieux. Mais celui qui les accomplira et les enseignera, celui-là sera appelé grand dans le royaume des cieux.
\switchcolumn*

Ant. ad Offertorium.\hfill Ct 1, 12
\switchcolumn
Antienne d’offertoire.
\switchcolumn*

Fascículus myrrhæ diléctus meus mihi, inter úbera mea commorábitur.
\switchcolumn
Mon Bien-Aimé est comme un bouquet de myrrhe, il demeurera entre mes seins.
\switchcolumn*

Secreta
\switchcolumn
Secrète
\switchcolumn*

Grata tibi sit, Deus, intercedénte  beáto Bernárdo, hujus oblátio Sacraménti, quod in memóriam passiónis Unigéniti tui, tuæ offérimus majestáti. Per eúmdem Dóminum.
\switchcolumn
Que vous soit agréable, ô Dieu, grâce  à l’intercession du bienheureux Bernard, l’oblation de ce sacrement, que nous offrons à votre Majesté en mémoire de la passion de votre Fils unique. Par le même.
\switchcolumn*

Ant. ad Comm.\hfill Si 51, 30
\switchcolumn
Antienne de communion.
\switchcolumn*

Dedit mihi Dóminus linguam mércedem meam : et in ipsa laudábo eum.
\switchcolumn
Le Seigneur m’a donné en récompense une langue par laquelle je le louerai.
\switchcolumn*

Postcommunio
\switchcolumn
Postcommunion
\switchcolumn*

Suum in nobis, omnípotens Deus, intercedénte beáto Bernárdo, cibus quem súmpsimus operétur efféctum : ut incórporet nos sibi esus edéntes. Per Dóminum.
\switchcolumn
Que cette nourriture que nous avons reçue, ô Dieu tout-puissant, opère en nous son effet, par l’intercession du bienheureux Bernard : une fois mangée, qu’elle nous incorpore à elle-même. Par Notre-Seigneur.
\switchcolumn*

Die 28 augusti
\switchcolumn
Le 28 août
\switchcolumn*

S. Augustini
\switchcolumn
S. Augustin
\switchcolumn*

Ep., Conf.
\switchcolumn
Évêque, Confesseur
\switchcolumn*

et Eccl. Doctoris
\switchcolumn
et Docteur de l’Église
\switchcolumn*

Ant. ad Introitum.\hfill Si 15, 5
\switchcolumn
Antienne d’introït.
\switchcolumn*

In médio Ecclésiæ apéruit os ejus :  et implévit eum Dóminus spíritu sapiéntiæ et intelléctus : stolam glóriæ índuit eum. √~Ps 91, 2. Bonum est confitéri Dómino : et psállere nómini tuo, Altíssime. √~Glória Patri. In médio.
\switchcolumn
Au milieu de l’Église, le Seigneur a ou vert sa bouche et il l’a rempli de l’esprit de sagesse et d’intelligence ; il l’a revêtu du vêtement de gloire. √~Il est bon de célébrer le Seigneur, et de psalmodier pour votre Nom, ô Très-Haut. √~Gloire au Père. Au milieu.
\switchcolumn*

Oratio
\switchcolumn
Collecte
\switchcolumn*

Deus, qui abditióra sapiéntiæ arcána  beáto Augustíno revelándo, et divínæ caritátis flammas in ejus corde excitándo, miráculum colúmnæ nubis et ignis in Ecclésia tua renovásti : concéde ; ut ejus ductu mundi vórtices felíciter transeámus, et ad ætérnam promissiónis pátriam perveníre mereámur. Per Dóminum.
\switchcolumn
Ô Dieu qui, en révélant au bienheureux  Augustin les secrets les plus cachés de la sagesse et en suscitant dans son cœur les flammes de l’amour divin, avez renouvelé dans votre Église le miracle de la colonne de nuée et de feu, accordez-nous de passer heureusement sous sa conduite les tourbillons du monde présent, et de mériter de parvenir à la patrie de l’éternelle promesse. Par Notre-Seigneur.
\switchcolumn*

Léctio libri Sapientiæ.
\switchcolumn
Lecture du Livre de la Sagesse.
\switchcolumn*

Si 50, 1-14
\switchcolumn

\switchcolumn*

Ecce sacérdos magnus, qui in vita  sua suffúlsit domum, et in diébus suis corroborávit templum. Templi étiam altitúdo ab ipso fundáta est, duplex ædificátio, et excélsi paríetes templi. In diébus ipsíus emanavérunt pútei aquárum, et quasi mare adimpléti sunt supra modum. Qui curávit gentem suam, et liberávit eam a perditióne. Qui præváluit amplificáre civitátem, qui adéptus est glóriam in conversatióne gentis : et ingréssum domus et átrii amplificávit. Quasi stella matutína in médio nébulæ, et quasi luna plena, in diébus suis lucet ; et quasi sol refúlgens, sic ille effúlsit in templo Dei. Quasi arcus refúlgens inter nébulas glóriæ, et quasi flos rosárum in diébus vernis, et quasi lília quæ sunt in tránsitu aquæ, et quasi thus rédolens in diébus æstátis ; quasi ignis effúlgens et thus ardens in igne ; quasi vas auri sólidum ornátum omni lápide pretióso ; quasi olíva púllulans, et cypréssus in altitúdinem se extóllens, in accipiéndo ipsum stolam glóriæ, et vestíri eum in consummatiónem virtútis. In ascénsu altáris sancti, glóriam dedit sanctitátis amíctum. In accipiéndo autem partes de manu sacerdótum, et ipse stans juxta aram ; et circa illum coróna fratrum : quasi plantátio cedri in monte Líbano, sic circa illum stetérunt quasi rami palmæ ; et omnes fílii Aaron in glória sua.
\switchcolumn
Voici le grand prêtre qui a soutenu la  maison et fortifié le temple pendant sa vie. C’est par lui que fut fondée la hauteur double, le haut contrefort de l’enceinte du Temple. En son temps  les puits des eaux se remplirent comme la mer, au delà de toute mesure. Il guérit son peuple et le libéra de la perdition. Il réussit à agrandir la cité, il a obtenu la gloire avec son peuple autour de lui, et il a agrandi l’entrée du temple et du parvis. Il brille comme l’étoile du matin au milieu de la nuée, et comme la lune en son plein. Comme le soleil resplendissant, ainsi a-t-il brillé dans le temple de Dieu, comme l’arc-en-ciel brillant parmi les nuées glorieuses, comme la rose au printemps, comme les lis au bord des eaux, comme l’encens qui répand son parfum en été, comme le feu brillant et l’encens brûlant dans le feu, comme un solide vase d’or orné de toutes les pierres précieuses, comme l’olivier qui étend ses pousses et le cyprès qui s’élève ; ainsi était-il lorsqu’il prenait le vêtement de gloire et qu’il revêtait ses superbes ornements ; en montant l’autel saint, il remplissait de gloire l’enceinte du sanctuaire, lorsqu’il recevait les portions des sacrifices de la main des prêtres, et qu’il se tenait près de l’autel, avec autour de lui une couronne de frères. Comme une plantation de cèdres sur le Liban, ainsi se tenaient autour de lui comme des branches de palmier tous les fils d’Aaron dans leur gloire.
\switchcolumn*

Graduale. Ps 36, 30-31. Os justi meditábitur sapiéntiam, et lingua ejus loquétur judícium. √~Lex Dei ejus  in corde ipsíus : et non supplantabúntur gressus ejus.
\switchcolumn
Graduel. La bouche du juste méditera la sagesse, et sa langue prononcera le jugement. √~La loi de son Dieu est dans son cœur, et ses pas ne chancelleront point.
\switchcolumn*

Allelúia, allelúia. √~Si 45, 9. Amávit eum Dóminus, et ornávit eum : stolam glóriæ índuit eum. Allelúia.
\switchcolumn
Alléluia, alléluia. √~Le Seigneur l’a aimé et l’a orné ; il l’a revêtu du vêtement de gloire. Alléluia.
\switchcolumn*

Sequéntia sancti Evangélii secúndum Matthǽum.
\switchcolumn
Lecture du saint évangile selon saint Matthieu.
\switchcolumn*

Mt 5, 13-19
\switchcolumn

\switchcolumn*

In illo témpore : Dixit Jesus discípulis  suis : Vos estis sal terræ. Quod si sal evanúerit, in quo saliétur ? Ad níhilum valet ultra, nisi ut mittátur foras, et conculcétur ab homínibus. Vos estis lux mundi. Non potest cívitas abscóndi supra montem pósita. Neque accéndunt lucérnam, et ponunt eam sub módio, sed super candelábrum, ut lúceat ómnibus qui in domo sunt. Sic lúceat lux vestra coram homínibus, ut vídeant ópera vestra bona, et gloríficent Patrem vestrum, qui in cælis est. Nolíte putáre, quóniam veni sólvere legem, aut prophétas : non veni sólvere, sed adimplére. Amen quippe dico vobis, donec tránseat cælum et terra, iota unum, aut unus apex non præteríbit a lege, donec ómnia fiant. Qui ergo sólverit unum de mandátis istis mínimis, et docúerit sic hómines, mínimus vocábitur in regno cælórum : qui autem fécerit, et docúerit, hic magnus vocábitur in regno cælórum.
\switchcolumn
En ce temps-là, Jésus dit à ses disciples :  Vous êtes le sel de la terre. Si le sel s’affadit, avec quoi le salera-t-on ? Il n’est plus bon à rien, sinon à être jeté dehors et piétiné par les hommes. Vous êtes la lumière du monde. Une ville placée sur une montagne ne peut être cachée Et quand on allume une lampe, ce n’est pas pour la placer sous le boisseau mais sur le candélabre, afin qu’elle brille pour tous ceux qui sont dans la maison. Qu’ainsi votre lumière brille devant les hommes, afin qu’ils voient vos bonnes œuvres et qu’ils glorifient votre Père qui est dans les cieux. Ne pensez pas que je sois venu abolir la loi ou les prophètes. Je ne suis pas venu abolir, mais accomplir. Amen je vous le dis : Jusqu’à ce que passent le ciel et la terre, pas un iota ou un trait de la loi ne passera : tout sera accompli. Celui donc qui abolira l’un de ces plus petits commandements et enseignera aux hommes à faire ainsi, sera appelé le plus petit dans le royaume des cieux. Mais celui qui les accomplira et les enseignera, celui-là sera appelé grand dans le royaume des cieux.
\switchcolumn*

Ant. ad offertorium.\hfill Ps 91, 13
\switchcolumn
Antienne d’offertoire.
\switchcolumn*

Justus ut palma florébit : sicut cedrus, quæ in Líbano est, multiplicábitur.
\switchcolumn
Le juste fleurira comme le palmier ; il se multipliera comme le cèdre du Liban.
\switchcolumn*

Secreta
\switchcolumn
Secrète
\switchcolumn*

Omnípotens sempitérne Deus,  qui præcláro sapiéntiæ lúmine beáti Augustíni mentem illustrásti, et sancti amóris jáculo ejúsdem cor transverberasti : da nobis fámulis tuis ; ut illíus doctrínæ et caritátis partícipes éffici mereámur. Per Dóminum.
\switchcolumn
Dieu tout-puissant et éternel, qui avez  éclairé l’esprit du bienheureux Augustin par un admirable rayon de la sagesse, et qui par un trait du saint amour avez blessé son cœur, accordez-nous, à nous qui sommes vos serviteurs, de mériter de participer à sa doctrine et à sa charité. Par N. S.
\switchcolumn*

Ant. ad Comm.\hfill Lc 12, 42
\switchcolumn
Antienne de communion.
\switchcolumn*

Fidélis servus et prudens, quem constítuit Dóminus super famíliam suam : ut det illis in témpore trítici mensúram.
\switchcolumn
Voici le serviteur fidèle et prudent que son seigneur a établi sur sa famille, afin de leur donner en temps opportun leur mesure de blé.
\switchcolumn*

Postcommunio
\switchcolumn
Postcommunion
\switchcolumn*

Fove, Dómine, famíliam tuam  munéribus sacris, quam cælésti libámine recreásti : et, ut solémnia sancti Augustíni devóte concélebret ; infúnde lumen supérnæ cognitiónis, et flammam ætérnæ caritátis. Per Dóminum.
\switchcolumn
Réchauffez par les saints dons, Seigneur,  votre famille que vous avez nourrie de la nourriture céleste ; et afin qu’elle célèbre avec dévotion la solennité de saint Augustin, infusez en nous la lumière de la connaissance d’en-haut et la flamme de l’éternelle charité. Par Notre-Seigneur.
\switchcolumn*

Die 17 Septembris
\switchcolumn
Le 17 septembre
\switchcolumn*

S. Hildegardis
\switchcolumn
S. Hildegarde
\switchcolumn*

Virginis
\switchcolumn
Vierge
\switchcolumn*

Ant. ad Introitum.\hfill Ps 44, 8
\switchcolumn
Antienne d’introït.
\switchcolumn*

Dilexísti justítiam, et odísti  iniquitátem : proptérea unxit te Deus, Deus tuus, óleo lætítiæ præ consórtibus tuis. √~Ibid., 2. Eructávit cor meum verbum bonum : dico ego ópera mea Regi. √~Glória Patri. Dilexísti.
\switchcolumn
Tu as aimé la justice et haï l’iniquité, c’est  pourquoi Dieu, ton Dieu, t’a ointe de l’huile d’allégresse plus que tes compagnes. √~De mon cœur jaillit une belle parole, je dis mes œuvres au Roi. √~Gloire au Père. Tu as aimé.
\switchcolumn*

Oratio
\switchcolumn
Collecte
\switchcolumn*

Deus, qui beátam Hildegárdem  Vírginem tuam donis cæléstibus decorásti : tríbue quǽsumus ; ut ejus vestígiis et documéntis insisténtes, a præséntis hujus sǽculi calígine ad lucem tuam delectábilem transíre mereámur. Per Dóminum.
\switchcolumn
Dieu qui avez orné de dons célestes  votre Vierge Hildegarde, accordez-nous, nous vous en prions, que, assidus à suivre ses traces et ses enseignements, nous méritions de passer des ténèbres du monde présent à votre délectable lumière. Par Notre-Seigneur.
\switchcolumn*

Léctio epístolæ beáti Pauli Apóstoli ad Corínthios.
\switchcolumn
Lecture de la lettre du bienheureux Apôtre Paul aux Corinthiens.
\switchcolumn*

2 Co 10, 17-18 ; 11, 1-2
\switchcolumn

\switchcolumn*

Fratres : Qui glóriatur, in Dómino  gloriétur. Non enim qui seípsum comméndat, ille probátus est : sed quem Deus comméndat. Utinam sustinerétis módicum quid insipiéntiæ meæ, sed et supportáte me : ǽmulor enim vos Dei æmulatióne. Despóndi enim vos uni viro vírginem castam exhibére Christo.
\switchcolumn
Frères, que celui qui se glorifie se glorifie  dans le Seigneur. Car ce n’est pas celui qui se recommande lui-même qui a fait ses preuves, mais celui que Dieu recommande. Si seulement vous supportiez de ma part un peu de folie ! Mais bien sûr vous me supportez. Car je suis jaloux de vous d’une jalousie divine. Car je vous ai fiancés à un seul homme, comme une vierge chaste à présenter au Christ.
\switchcolumn*

Graduale. Ps 44, 5. Spécie tua, et pulchritúdine tua inténde, próspere procéde, et regna. √~Propter veritátem, et mansuetúdinem, et justítiam : et dedúcet te mirabíliter déxtera tua.
\switchcolumn
Graduel. Dans ta dignité et ta beauté, tends ton arc, marche avec succès et règne √~pour la vérité, la douceur et la justice, et ta main droite te conduira admirablement.
\switchcolumn*

Allelúia, allelúia. √~Ps 44, 15-16. Adducéntur Regi vírgines post eam : próximæ ejus afferéntur tibi. Allelúia.
\switchcolumn
Alléluia, alléluia. √~Après elle, des vierges seront conduites au Roi ; ses proches te seront amenées. Alléluia.
\switchcolumn*

Sequéntia sancti Evangélii secúndum Matthǽum.
\switchcolumn
Lecture du saint évangile selon saint Matthieu
\switchcolumn*

Mt 25, 1-13
\switchcolumn

\switchcolumn*

In illo témpore : Dixit Jesus discípulis  suis parábolam hanc : Símile erit regnum cælórum decem virgínibus : quæ, accipiéntes lámpades suas, exiérunt óbviam sponso et sponsæ. Quinque autem ex eis erant fátuæ, et quinque prudéntes : sed quinque fátuæ, accéptis lampádibus, non sumpsérunt óleum secum : prudéntes vero accepérunt óleum in vasis suis cum lampádibus. Moram autem faciénte sponso, dormitavérunt omnes, et domiérunt. Média autem nocte clamor factus est : Ecce, sponsus venit, exíte óbviam ei. Tunc surrexérunt omnes vírgines illæ, et ornavérunt lámpades suas. Fátuæ autem sapiéntibus dixérunt : Date nobis de óleo vestro : quia lámpades nostræ exstinguúntur. Respondérunt prudéntes, dicéntes : Ne forte non suffíciat nobis, et vobis, ite pótius ad vendéntes, et émite vobis. Dum autem irent émere, venit sponsus : et quæ parátæ erant, intravérunt cum eo ad núptias, et clausa est jánua. Novíssime vero véniunt et réliquæ vírgines, dicéntes : Dómine, Dómine, áperi nobis. At ille respóndens, ait : Amen, dico vobis, néscio vos. Vigiláte ítaque, quia néscitis diem neque horam.
\switchcolumn
En ce temps-là, Jésus dit à ses disciples  cette parabole : Le royaume des cieux sera semblable à dix vierges qui, ayant pris leurs lampes, s’en allèrent à la rencontre de l’époux et de l’épouse. Or cinq d’entre elles étaient insensées et cinq étaient sages. Les cinq insensées, ayant pris leurs lampes, ne prirent point d’huile avec elles, tandis que les sages prirent de l’huile dans leurs vases avec les lampes. Mais comme l’époux tardait, elles s’assoupirent toutes et s’endormirent. Or au milieu de la nuit, une clameur se fit entendre : Voici que l’époux arrive, sortez à sa rencontre. Alors toutes ces vierges se levèrent et préparèrent leurs lampes. Les insensées dirent aux sages : Donnez-nous de votre huile, car nos lampes s’éteignent. Les sages répondirent : De peur que cela ne suffise pour nous et pour vous, allez plutôt chez les marchands et achetez-vous en. Mais tandis qu’elles allaient en acheter, l’époux arriva. Celles qui étaient prêtes entrèrent avec lui dans la salle des noces, et la porte fut fermée. À la fin arrivent les autre vierges, disant : Seigneur, Seigneur, ouvre-nous. Mais lui leur répondit : Amen, je vous le dis, je ne vous connais pas. Veillez donc, car vous ne savez ni le jour ni l’heure.
\switchcolumn*

Ant. ad offertorium.\hfill Ps 44, 10
\switchcolumn
Antienne d’offertoire.
\switchcolumn*

Fíliæ regum in honóre tuo, ástitit regína a dextris tuis in vestítu deauráto, circúmdata varietáte.
\switchcolumn
Les filles des rois en ton honneur. La reine se tient à ta droite dans un habit d’or, couverte de vêtements variés.
\switchcolumn*

Secreta
\switchcolumn
Secrète
\switchcolumn*

Sanctæ Hildegárdis Vírginis tuæ  festa recoléntes, preces offérimus et hóstias immolámus : præsta, Dómine, quǽsumus ; ut cum præsídio temporáli, vitæ nobis prǽbeant increménta perpétuæ. Per Dóminum.
\switchcolumn
En célébrant la fête de sainte Hildegarde  votre vierge, nous offrons des prières et nous immolons des hosties ; accordez-nous, Seigneur, nous vous en prions, qu’ils nous obtiennent, avec le secours temporel, un accroissement de vie éternelle. Par Notre-Seigneur.
\switchcolumn*

Ant. ad Comm.\hfill Mt 25, 4.6
\switchcolumn
Antienne de communion.
\switchcolumn*

Quinque prudéntes vírgines accepérunt óleum in vasis suis cum lampádibus : média autem nocte clamor factus est : Ecce sponsus venit : exíte óbviam Christo Dómino.
\switchcolumn
Les cinq vierges prudentes prirent de l’huile dans les fioles avec leurs lampes. Au milieu de la nuit un cri se fit entendre : Voici que l’époux arrive, sortez à la rencontre du Christ Seigneur.
\switchcolumn*

Postcommunio
\switchcolumn
Postcommunion
\switchcolumn*

Perpétuo, Dómine, favóre perséquere  quos réficis divíno mystério ; et quos imbuísti cæléstibus institútis, eósdem intercedénte beáta Hildegárde Vírgine tua, salutáribus comitáre solátiis. Per Dóminum.
\switchcolumn
Accompagnez, Seigneur, d’une bien veillance éternelle ceux que vous réconfortez par ces divins mystères, et ceux que vous avez remplis des enseignements célestes, soutenez-les, par l’intercession de la bienheureuse Hildegarde votre Vierge, de salutaires consolations. Par Notre-Seigneur.
\switchcolumn*

Die 4 novembris
\switchcolumn
Le 4 novembre
\switchcolumn*

S. Caroli
\switchcolumn
S. Charles Borromée
\switchcolumn*

Ep. et Conf.
\switchcolumn
Évêque et Confesseur
\switchcolumn*

Ant. ad Introitum.\hfill Si 45, 29-30
\switchcolumn
Antienne d’introït.
\switchcolumn*

In bonitáte et alacritáte ánimæ  suæ plácuit Deo : ídeo státuit illi testaméntum pacis, príncipem sanctórum et gentis suæ, ut sit illi sacerdótii dígnitas in ætérnum. √~Ps 72, 1. Quam bonus Israel Deus his, qui recto sunt corde ! √~Glória Patri. In bonitáte.
\switchcolumn
Par sa bonté et l’ardeur de son âme, il a  plu à Dieu. C’est pourquoi celui-ci lui a confié l’alliance de paix, il en a fait le prince du sanctuaire et de son peuple, afin qu’il possède à jamais la dignité du sacerdoce. √~Que Dieu est bon, ô Israël, pour ceux qui ont le cœur droit ! √~Gloire au Père. Par sa bonté.
\switchcolumn*

Oratio
\switchcolumn
Collecte
\switchcolumn*

Ecclésiam tuam, Dómine, sancti  Cároli Confessóris tui atque Pontíficis contínua protectióne custódi : ut, sicut illum pastorális sollicitúdo gloriósum réddidit ; ita nos ejus intercéssio in tuo semper fáciat amóre fervéntes. Per Dóminum.
\switchcolumn
Gardez votre Église, Seigneur, par la protection continuelle de saint Charles, votre Confesseur et Pontife, afin que, de même que sa sollicitude pastorale l’a lui-même rendu glorieux, ainsi son intercession nous rende toujours fervents dans votre amour. Par Notre-Seigneur.
\switchcolumn*

Léctio libri Sapiéntiæ.
\switchcolumn
Lecture du livre de la Sagesse.
\switchcolumn*

Si 50, 1.4-11
\switchcolumn

\switchcolumn*

Ecce sacérdos magnus, qui in vita  sua suffúlsit domum, et in diébus suis corroborávit templum. Qui curávit gentem suam, et liberávit eam a perditióne. Qui præváluit amplificáre civitátem, qui adéptus est glóriam in conversatióne gentis : et ingréssum domus et átrii amplificávit. Quasi stella matutína in médio nébulæ, et quasi luna plena, in diébus suis lucet ; et quasi sol refúlgens, sic ille effúlsit in templo Dei. Quasi arcus refúlgens inter nébulas glóriæ, et quasi flos rosárum in diébus vernis, et quasi lília quæ sunt in tránsitu aquæ, et quasi thus rédolens in diébus æstátis ; quasi ignis effúlgens et thus ardens in igne ; quasi vas auri sólidum ornátum omni lápide pretióso ; quasi olíva púllulans, et cypréssus in altitúdinem se extóllens, in accipiéndo ipsum stolam glóriæ, et vestíri eum in consummatiónem virtútis.
\switchcolumn
Voici le grand prêtre qui a soutenu la  maison et fortifié le temple pendant sa vie. Il guérit son peuple et le libéra de la perdition. Il réussit à agrandir la cité, il a obtenu la gloire avec son peuple autour de lui, et il a agrandi l’entrée du temple et du parvis. Il brille comme l’étoile du matin au milieu de la nuée, et comme la lune en son plein. Comme le soleil resplendissant, ainsi a-t-il brillé dans le temple de Dieu, comme l’arc-en-ciel brillant parmi les nuées glorieuses, comme la rose au printemps, comme les lis au bord des eaux, comme l’encens qui répand son parfum en été, comme le feu brillant et l’encens brûlant dans le feu, comme un solide vase d’or orné de toutes les pierres précieuses, comme l’olivier qui étend ses pousses et le cyprès qui s’élève ; ainsi était-il lorsqu’il prenait le vêtement de gloire et qu’il revêtait ses superbes ornements.
\switchcolumn*

Graduale. Sap 4, 13-14. Consummátus in brevi explévit témpora multa : plácita enim erat Deo ánima illíus. √~Propter hoc properávit edúcere illum de médio iniquitátis.
\switchcolumn
Graduel. Porté à sa perfection en peu de temps, il a accompli des temps nombreux, car son âme était agréable à Dieu. √~C’est pourquoi Dieu s’est hâté de l’enlever du milieu de l’iniquité.
\switchcolumn*

Allelúia, allelúia. √~Si 45, 9. Induit eum stolam glóriæ, et coronávit eum in vasis virtútis. Allelúia.
\switchcolumn
Alléluia, alléluia. √~Le Seigneur l’a revêtu du vêtement de gloire, et il l’a couronné avec des insignes de force. Alléluia.
\switchcolumn*

Sequéntia sancti Evangélii secúndum Joánnem/
\switchcolumn
Lecture du saint évangile selon saint Jean.
\switchcolumn*

Jo 10, 11-16
\switchcolumn

\switchcolumn*

In illo témpore : Dixit Jesus pharisǽis :  Ego sum pastor bonus. Bonus pastor ánimam suam dat pro óvibus suis. Mercenárius autem, et qui non est pastor, cujus non sunt oves própriæ, videt lupum veniéntem, et dimíttit oves, et fugit : et lupus rapit, et dispérgit oves ; mercenárius autem fugit, quia mercenárius est, et non pértinet ad eum de óvibus. Ego sum pastor bonus : et cognósco meas, et cognóscunt me meæ. Sicut novit me Pater, et ego agnósco Patrem : et ánimam meam pono pro óvibus meis. Et álias oves hábeo, quæ non sunt ex hoc ovíli : et illas opórtet me addúcere, et vocem meam áudient, et fiet unum ovíle et unus pastor.
\switchcolumn
En ce temps-là, Jésus dit aux Pharisiens :  Je suis le bon Pasteur. Le bon pasteur donne sa vie pour ses brebis. Mais le mercenaire, et celui qui n’est point pasteur, celui à qui les brebis n’appartiennent pas, voyant venir le loup, abandonne les brebis, et s’enfuit ; et le loup les enlève et disperse le troupeau. Or le mercenaire s’enfuit, parce qu’il est mercenaire, et qu’il ne se met point en peine des brebis. Moi, je suis le bon Pasteur : je connais mes brebis, et mes brebis me connaissent, comme mon Père me connaît, et que je connais mon Père ; et je donne ma vie pour mes brebis. J’ai encore d’autres brebis qui ne sont pas de cette bergerie ; il faut aussi que je les amène. Elles écouteront ma voix ; et il y aura un seul troupeau et un seul Pasteur.
\switchcolumn*

Ant. ad offertorium.\hfill Sap 8, 2
\switchcolumn
Antienne d’offertoire.
\switchcolumn*

Amávit sapiéntiam a juventúte sua, et quæsívit sibi sponsam eam assúmere, et amátor factus est formæ illíus.
\switchcolumn
Il a aimé la Sagesse depuis sa jeunesse, il a cherché à la prendre pour épouse, et il est devenu amoureux de sa beauté.
\switchcolumn*

Secreta
\switchcolumn
Secrète
\switchcolumn*

Sancti Cároli Confessóris tui  atque Pontíficis pastorális offícii vigilántiam et præcláras virtútes admirántibus : præsta, quǽsumus, [Dómine] ; ut, ipsíus inhæréntes vestígiis, tibi digne múnera deferámus. Per Dóminum.
\switchcolumn
À nous qui admirons la vigilance de saint  Charles votre Confesseur et Pontife dans son devoir de pasteur, ainsi que ses hautes vertus, accordez, Seigneur, qu’en suivant ses traces, nous vous fassions de dignes offrandes. Par Notre-Seigneur.
\switchcolumn*

Ant. ad Comm.\hfill Si 39, 13
\switchcolumn
Antienne de communion.
\switchcolumn*

Non recédet memória ejus, et nomen ejus requirétur a generatióne in generatiónem.
\switchcolumn
Sa mémoire ne disparaîtra pas, et son nom sera recherché de génération en génération.
\switchcolumn*

Postcommunio
\switchcolumn
Postcommunion
\switchcolumn*

Sanctíficent nos, quǽsumus, Dómine,  sumpta mystéria : et sancto Cárolo Confessóre tuo atque Pontífice intercedénte, nostrórum puríficent máculas delictórum. Per Dóminum.
\switchcolumn
Que les mystères que nous avons reçus  nous sanctifient, nous vous en prions, Seigneur, et par l’intercession de saint Charles votre Confesseur et Pontife, qu’ils purifient les souillures de nos péchés. Par Notre-Seigneur.
\switchcolumn*

Die 13 novembris
\switchcolumn
Le 13 novembre
\switchcolumn*

S. Benigni
\switchcolumn
S. Bénigne
\switchcolumn*

Ep. et mart.
\switchcolumn
Évêque et martyr
\switchcolumn*

Ant. ad Introitum.\hfill Ps 95, 3.5
\switchcolumn
Antienne d’introït.
\switchcolumn*

Annuntiáte inter gentes, quóniam  omnes dii géntium dæmónia : Dóminus autem cœlos fecit, allelúia, allelúia. √~Ps 95, 1. Cantáte Dómino cánticum novum ; cantáte Dómino omnis terra. √~Glória Patri. Annuntiate.
\switchcolumn
Annoncez parmi les peuples : tous les  dieux des peuples sont des démons, mais le Seigneur à fait les cieux, alléluia, alléluia. √~Chantez au Seigneur un cantique nouveau ; chantez au Seigneur, toute la terre. √~Gloire au Père. Annoncez.
\switchcolumn*

Oratio
\switchcolumn
Collecte
\switchcolumn*

Deus, qui nos beáti Benígni Mártyris  tui prædicatióne, de infidelitátis ténebris in admirábile Evangélii lumen transférre dignátus es : fac ut ejus intercessióne crescámus in grátia et cognitióne Dómini nostri et Salvatóris Jesu Christi Fílii tui. Qui tecum.
\switchcolumn
Ô Dieu qui, par la prédication du bien heureux Bénigne, votre martyr, avez daigné nous faire passer des ténèbres de l’infidélité à l’admirable lumière de l’Évangile, faites que par son intercession nous grandissions dans la grâce et la connaissance de notre Seigneur et Sauveur Jésus-Christ votre Fils. Lui qui vit et règne.
\switchcolumn*

Léctio epístolæ beáti Pauli Apóstoli ad Thessalonicénses.
\switchcolumn
Lecture de l’épître du bienheureux Apôtre Paul aux Thessaloniciens.
\switchcolumn*

1 Th 2, 2-13
\switchcolumn

\switchcolumn*

Fratres : Fidúciam habúimus in Deo  nostro loqui ad vos Evangélium Dei in multa sollicitúdine. Exhortátio enim nostra non de erróre, neque de immundítia, neque in dolo : sed sicut probáti sumus a Deo, ut crederétur nobis Evangélium, ita lóquimur ; non quasi homínibus placéntes, sed Deo, qui probat corda nostra. Neque enim aliquándo fúimus in sermóne adulatiónis, sicut scitis ; neque in occasióne avarítiæ, Deus testis est ; nec quæréntes ab homínibus glóriam neque a vobis, neque ab áliis. Cum possémus vobis óneri esse, ut Christi Apóstoli : sed facti sumus párvuli in médio vestrum, tamquam si nutrix fóveat fílios suos. Ita desiderántes vos, cúpide volebámus trádere vobis non solum Evangélium Dei, sed étiam ánimas nostras, quóniam caríssimi nobis facti estis. Mémores enim estis, fratres, labóris nostri, et fatigatiónis : nocte ac die operántes, ne quem vestrum gravarémus, et prædicávimus in vobis Evangélium Dei. Vos testes estis et Deus, quam sancte, et juste, et sine queréla, vobis qui credidístis, fúimus : sicut scitis, quáliter unumquémque vestrum, sicut pater fílios suos, deprecántes vos et consolántes, testificáti sumus, ut ambularétis digne Deo, qui vocávit vos in suum regnum et glóriam. Ideo et nos grátias ágimus Deo sine intermissióne ; quóniam cum accepissétis a nobis verbum audítus Dei, accepístis illud, non ut verbum hóminum, sed, sicut est vere, verbum Dei, qui operátur in vobis, qui credidístis.
\switchcolumn
Frères, en nous confiant en notre Dieu,  nous vous avons prêché hardiment l’Évangile de Dieu parmi beaucoup de peines et de combats. Car nous ne vous avons point prêché une doctrine d’erreur ou d’impureté, ou avec tromperie. Mais comme Dieu nous a choisis pour nous confier son Évangile, ainsi nous parlons, non pour plaire aux hommes, mais à Dieu, qui voit le fond de nos coeurs. Car nous n’avons usé d’aucune parole de flatterie, comme vous le savez ; et notre ministère n’a point servi de prétexte à notre avarice, Dieu en est témoin ; et nous n’avons point non plus recherché aucune gloire de la part des hommes, ni de vous, ni d’aucun autre, quoique nous eussions pu, comme apôtres de Jésus-Christ, vous charger de notre subsistance ; mais nous nous sommes conduits parmi vous avec une douceur d’enfants, comme une nourrice qui a soin de ses enfants. Aussi dans l’affection que nous ressentions pour vous, nous aurions souhaité de vous donner, non seulement la connaissance de l’Évangile de Dieu, mais aussi notre propre vie, tant était grand l’amour que nous vous portions. Car vous n’avez pas oublié, mes frères, quelle peine et quelle fatigue nous avons soufferte, et comme nous vous avons prêché l’Évangile de Dieu en travaillant jour et nuit, pour n’être à charge à aucun de vous. Vous êtes témoins vous-mêmes, et Dieu l’est aussi, combien la manière dont je me suis conduit envers vous qui avez embrassé la foi, a été sainte, juste et irréprochable. Et vous savez que j’ai agi envers chacun de vous comme un père envers ses enfants,  vous exhortant, vous consolant, et vous conjurant de vous conduire d’une manière digne de Dieu, qui vous a appelés à son royaume et à sa gloire. C’est pourquoi aussi, nous rendons à Dieu de continuelles actions de grâces, de ce qu’ayant entendu la parole de Dieu que nous vous prêchions, vous l’avez reçue, non comme une parole d’homme, mais comme ce qu’elle est vraiment : la parole de Dieu, qui agit efficacement en vous qui avez cru.
\switchcolumn*

Graduale. 2 Co 1, 14. Glória vestra sumus sicut et vos nostra, in die Dómini nostri Jesu Christi. √~1 Co 9, 1 et 4, 15. Opus meum vos estis in Dómino ; nam in Christo Jesu per Evangélium ego vos génui.
\switchcolumn
Graduel. Nous sommes votre gloire, comme vous la nôtre, au jour de notre Seigneur Jésus-Christ. √~Vous êtes mon œuvre dans le Seigneur ; car dans le Christ Jésus, je vous ai engendrés par l’Évangile.
\switchcolumn*

Allelúia, allelúia. √~2 Co 2, 14. Grátias Deo, qui triúmphat nos in Christo Jesu, et odórem notítiæ suæ maniféstat per nos.
\switchcolumn
Alléluia, alléluia. √~Rendons grâces à Dieu qui nous fait triompher dans le Christ Jésus, et manifeste par nous la bonne odeur de sa connaissance.
\switchcolumn*

Sequentia
\switchcolumn
Séquence
\switchcolumn*

Exsúlta, felix Dívio,
\switchcolumn
Réjouis-toi, heureuse ville de Dijon,
\switchcolumn*

Quæ Benígni glorióso
\switchcolumn
Toi qui es embellie du sang
\switchcolumn*

Decoráris sánguine.
\switchcolumn
Du glorieux Bénigne.
\switchcolumn*

In Grǽcia sanctis satus,
\switchcolumn
Semé en Grèce par les saints,
\switchcolumn*

Polycárpi cura doctus,
\switchcolumn
Enseigné par les soins de Polycarpe,
\switchcolumn*

Christi fulsit lúmine.
\switchcolumn
Il brille de la lumière du Christ.
\switchcolumn*

Cunctis rebus abdicátis,
\switchcolumn
Ayant quitté toute chose,
\switchcolumn*

Zelo flagrans veritátis,
\switchcolumn
Brûlant du zèle de la vérité,
\switchcolumn*

Venit in Burgúndiam.
\switchcolumn
Il vient en Bourgogne.
\switchcolumn*

Sacri verbi seminátor
\switchcolumn
Il sème la Parole sainte,
\switchcolumn*

Et errórum extirpátor,
\switchcolumn
Il extirpe les erreurs,
\switchcolumn*

Fundávit Ecclésiam.
\switchcolumn
Il fonde une Église.
\switchcolumn*

Multis claret miráculis :
\switchcolumn
Il brille par de nombreux miracles,
\switchcolumn*

Sanat morbos, dira mortis
\switchcolumn
Il guérit les malades,
\switchcolumn*

Dissólvit impéria.
\switchcolumn
Il détruit le cruel pouvoir de la mort.
\switchcolumn*

Fremit dæmon, Cæsar furit ;
\switchcolumn
Le démon gronde, César se déchaîne,
\switchcolumn*

Sed Martyr fídei reddit
\switchcolumn
Mais le Martyr, invaincu,
\switchcolumn*

Invíctus obséquia.
\switchcolumn
Obéit à la foi.
\switchcolumn*

Quantus tormentórum
\switchcolumn
Combien lui sont présentés
\switchcolumn*

Prodit apparátus,
\switchcolumn
D’instruments de torture !
\switchcolumn*

Stridet flagellórum
\switchcolumn
Les fouets sifflent
\switchcolumn*

Ictus repétitus.
\switchcolumn
À coups redoublés.
\switchcolumn*

Bestiárum dentes,
\switchcolumn
Les dents des bêtes,
\switchcolumn*

Plumbum liquefáctum,
\switchcolumn
Le plomb fondu,
\switchcolumn*

Lánceæ, mucrónes,
\switchcolumn
Les lances, les épées
\switchcolumn*

Parant exítium.
\switchcolumn
Cherchent à le mettre à mort.
\switchcolumn*

Transvérsa fossus gémina
\switchcolumn
Percé en travers du corps
\switchcolumn*

Latus utrúmque láncea,
\switchcolumn
Par une double lance,
\switchcolumn*

Crucem gestit intra corda
\switchcolumn
Il porte dans son cœur
\switchcolumn*

Quam semper amáverat.
\switchcolumn
Cette croix qu’il a toujours aimée.
\switchcolumn*

Mox spinis caput púngitur,
\switchcolumn
Bientôt sa tête est piquée par les épines
\switchcolumn*

Cérebrum vecte frángitur,
\switchcolumn
Puis brisée par une barre ;
\switchcolumn*

Sicque pugil coronátur
\switchcolumn
C’est ainsi que le lutteur est couronné
\switchcolumn*

Ritu quo cupíerat.
\switchcolumn
De la manière qu’il désirait.
\switchcolumn*

O Benígne, Pater care !
\switchcolumn
Ô Bénigne, Père aimé,
\switchcolumn*

Apóstole gentis nostræ,
\switchcolumn
Apôtre de notre peuple,
\switchcolumn*

Fídei Martyr præcláre,
\switchcolumn
Témoin insigne de la foi,
\switchcolumn*

Quos amásti, nunc tuére
\switchcolumn
Ceux que tu as aimés, défends-les
\switchcolumn*

De cœlo poténtius.
\switchcolumn
Plus puissamment aujourd’hui du haut du ciel.
\switchcolumn*

Credámus quæ docuísti ;
\switchcolumn
Puissions-nous croire ce que tu as enseigné,
\switchcolumn*

Pergámus quo pervenísti ;
\switchcolumn
Nous rendre où tu es parvenu ;
\switchcolumn*

Tecum mereámur frui
\switchcolumn
Puissions-nous mériter de jouir avec toi
\switchcolumn*

Corónis perénnibus.
\switchcolumn
Des couronnes éternelles.
\switchcolumn*

Amen. Allelúia.
\switchcolumn
Amen. Alléluia.
\switchcolumn*

Sequéntia sancti Evangélii secúndum Marcum.
\switchcolumn
Lecture du saint évangile selon saint Marc.
\switchcolumn*

Mc 16, 15-18
\switchcolumn

\switchcolumn*

In illo témpore : Dixit Jesus discípulis  suis : Eúntes in mundum univérsum, prædicáte Evangélium omni creatúræ. Qui credíderit et baptizátus fúerit, salvus erit : qui vero non credíderit, condemnábitur. Signa autem eos, qui credíderint, hæc sequéntur : in nómine meo dæmónia ejícient, linguis loquéntur novis ; serpéntes tollent, et si mortíferum quid bíberint, non eis nocébit : super ægros manus impónent, et bene habébunt.
\switchcolumn
En ce temps-là, Jésus dit à ses disciples :  Allez dans le monde entier, prêchez l’Évangile à toute créature. Celui qui croira et sera baptisé sera sauvé ; mais celui qui ne croira pas sera condamné. Voici les signes qui accompagneront ceux qui auront cru : en mon nom, ils chasseront les démons ; ils parleront de nouvelles langues ; ils prendront des serpents avec la main ; et s’ils boivent quelque breuvage mortel, il ne leur fera point de mal ; ils imposeront les mains sur les malades, et les malades seront guéris.
\switchcolumn*

Ant. ad offertorium.\hfill 1 Th 2, 8
\switchcolumn
Antienne d’offertoire.
\switchcolumn*

Tradídimus vobis non solum Evangélium Dei, sed étiam ánimas nostras ; quóniam caríssimi nobis facti estis. Allelúia.
\switchcolumn
Nous vous avons livré non seulement l’Évangile de Dieu, mais encore nos vies, car vous nous êtes devenus très chers. Alléluia.
\switchcolumn*

Secreta
\switchcolumn
Secrète
\switchcolumn*

Glóriam tibi, Dómine, et lætítiam  nobis salutáris tui reddat sacrifícium solémne, quod tibi hac sacra die offérimus, gaudéntes et congratulántes sancto Benígno Mártyri tuo, cui gáudium fuit immolári supra sacrifícium et obséquium fídei nostræ. Per Dóminum.
\switchcolumn
Que le sacrifice solennel vous rende  gloire, Seigneur, et nous donne la joie de votre salut. Nous vous l’offrons en ce jour saint, en nous réjouissant en l’honneur de saint Bénigne, votre martyr, qui s’est réjoui de pouvoir s’immoler sur le sacrifice et l’oblation de notre foi. Par Notre-Seigneur.
\switchcolumn*

Præfatio
\switchcolumn
Préface
\switchcolumn*

Vere dignum et justum est, æquum et salutáre, nos tibi semper et ubíque grátias ágere, Dómine sancte, Pater omnípotens, ætérne Deus : qui nos secúndum misericórdiam tuam magnam, de ténebris ad lucem vocáre dignátus es, et de potestáte Sátanæ eréptos, in fílios adoptiónis assúmere. Tua enim, Dómine, misericórdia, tua grátia, verbum fídei in nobis Mártyris tui labóre seminátum est, et sánguine fœcundátum. Nunc ergo, Pater sancte, confírma hoc quod operátus es in nobis ; et gregem istum, quem Fílio tuo donásti, consérva tuæ virtútis auxílio : ut sanctificátum in veritáte, perféctum in unitáte consummáre dignéris in glória, per eúmdem Christum Dóminum nostum. Per quem majestátem tuam laudant Angeli, adórant Dominatiónes, tremunt Potestátes. Cœli cœlorúmque Virtútes, ac beáta Séraphim sócia exultatióne concélebrant. Cum quibus et nostras voces ut admítti júbeas deprecámur, súpplici confessióne dicéntes : Sanctus.
\switchcolumn
Il est vraiment digne et juste, équitable et salutaire, de toujours et partout vous rendre grâces, Seigneur saint, Père tout-puissant, Dieu éternel, qui avez daigné nous appeler, dans votre grande miséricorde, des ténèbres à la lumière, et de nous recevoir comme des fils d’adoption, après nous avoir arrachés à la puissance de Satan. Car c’est par votre miséricorde, Seigneur, et par votre grâce, que la parole de la foi a été semée parmi nous grâce au labeur de votre martyr, et qu’elle a été fécondée par son sang. Maintenant donc, Père saint, affermissez ce que vous avez opéré en nous, et conservez par le secours de votre force ce troupeau que vous avez donné à votre Fils ; afin que sanctifié par la vérité, parfaitement uni, il soit par vous porté à sa perfection dans la gloire, par le même Christ notre Seigneur, par qui les Anges louent votre Majesté, les Dominations l’adorent, les Puissances tremblent devant elle, et les Cieux et les Vertus des Cieux, ainsi que les bienheureux Séraphins, l’acclament d’une commune jubilation. Nous vous prions d’accepter que nos voix se mêlent aux leurs par cette humble louange : Saint.
\switchcolumn*

Ant. ad Comm.\hfill Ac 9, 31
\switchcolumn
Antienne de communion.
\switchcolumn*

Ecclésia ædificabátur, ámbulans in timóre Dómini, et consolatióne Sancti Spíritus replebátur. Allelúia.
\switchcolumn
L’Église s’édifiait, marchant dans la crainte du Seigneur, et elle était remplie de la consolation du Saint Esprit.
\switchcolumn*

Postcommunio
\switchcolumn
Postcommunion
\switchcolumn*

Deus, qui beáti Benígni ministério  eam nobis grátiam contulísti, ut ex infidélibus fidéles efficerémur : adésto opéribus tuis, adésto munéribus ; ut quibus inest fides, non desit fídei fortitúdo. Per Dóminum nostrum.
\switchcolumn
Ô Dieu qui, par le ministère du bien- heureux Bénigne, nous avez fait cette grâce de passer de l’infidélité à la foi, regardez favorablement votre œuvre et nos dons, afin que la force de la foi ne manque pas à ceux qui croient. Par Notre-Seigneur.
\switchcolumn*

Die 16 novembris
\switchcolumn
Le 16 novembre
\switchcolumn*

S. Gertrudis Magnæ
\switchcolumn
S. Gertrude
\switchcolumn*

Virginis
\switchcolumn
Vierge
\switchcolumn*

Ant. ad Introitum.\hfill Ps 72, 28
\switchcolumn
Antienne d’introït.
\switchcolumn*

Mihi autem adhærére Deo bonum  est, pónere in Dómino Deo spem meam : ut annúntiem omnes prædicatiónes tuas in portis fíliæ Sion. √~Ps 72, 1. Quam bonus Israel Deus his qui recto sunt corde. √~Glória Patri. Mihi autem.
\switchcolumn
Pour moi il est bon d’adhérer à Dieu,  de mettre en Dieu mon espérance ; afin d’annoncer toutes vos prédications aux portes de la Fille de Sion. √~Que Dieu est bon, ô Israël, pour ceux qui ont le cœur droit ! √~Gloire au Père. Pour moi il est bon.
\switchcolumn*

Oratio
\switchcolumn
Collecte
\switchcolumn*

Deus, qui in puríssimo corde beátæ  Gertrúdis Vírginis tuæ jucúndam tibi habitatiónem præparásti : ejus méritis et intercessióne cordis nostri máculas cleménter abstérge ; ut digna divínæ majestátis tuæ habitátio éffici mereátur. Per Dóminum.
\switchcolumn
Ô Dieu qui vous vous êtes préparé une  demeure agréable dans le cœur très pur de la bienheureuse Gertrude votre Vierge ; par ses mérites et son intercession, lavez avec bonté les souillures de notre cœur, afin qu’il mérite de devenir une digne demeure de votre divine Majesté. Par Notre-Seigneur.
\switchcolumn*

Léctio libri Sapiéntiæ.
\switchcolumn
Lecture du livre de la Sagesse.
\switchcolumn*

Ct 2, 1-10
\switchcolumn

\switchcolumn*

Ego flos campi et lílium convállium.  Sicut lílium inter spinas, sic amíca mea inter fílias. Sicut malus inter ligna silvárum, sic diléctus meus inter fílios. Sub umbra illíus quem desideráveram sedi, et fructus ejus dulcis gútturi meo. Introdúxit me in cellam vináriam, ordinávit in me caritátem. Fulcíte me flóribus, stipáte me malis : quia amóre lángueo. Læva ejus sub cápite meo, et déxtera illíus amplexábitur me. Adjúro vos, fíliæ Jerúsalem, per cápreas cervósque campórum, ne suscitétis, neque evigiláre faciátis diléctam, quoadúsque ipsa velit. Símilis est diléctus meus cápreæ hinnulóque cervórum. En ipse stat post paríetem nostrum, respíciens per fenéstras, prospíciens per cancéllos. En diléctus meus lóquitur mihi : Surge, própera, amíca mea, colúmba mea, formósa mea, et veni.
\switchcolumn
Je suis la fleur des champs et le lis des  vallées. Comme le lis parmi les épines, ainsi ma bien-aimée parmi les filles. Comme le pommier parmi les arbres des forêts, ainsi mon bien-aimé parmi les fils. À l’ombre de celui que j’avais désiré je me suis assise, et son fruit est doux à ma bouche. Il m’a introduite dans le cellier à vin, il a ordonné en moi la charité. Soutenez-moi avec des fleurs, entourez-moi de pommes, car je languis d’amour. Sa main gauche est sous ma tête, et sa droite me tiendra embrassée. Je vous adjure, filles de Jérusalem, par les chevreuils et les cerfs des champs : ne réveillez ni ne faites s’éveiller la bien-aimée avant qu’elle le veuille elle-même. Mon bien-aimé est semblable au chevreuil et au petit des cerfs. Voici qu’il se tient derrière notre mur, regardant par les fenêtres, observant à travers les treillages. Voici que mon bien-aimé me parle : Lève-toi, hâte-toi, ma bien-aimée, ma colombe, ma toute-belle, et viens.
\switchcolumn*

Graduale. Ps 30, 8. Quóniam respexísti humilitátem meam, salvásti de necessitátibus ánimam meam. √~Ibid., 22. Benedíctus Dóminus : quóniam mirificávit misericórdiam suam mihi in civitáte muníta.
\switchcolumn
Graduel. Parce que tu as regardé ma bassesse, tu as sauvé mon âme de ses angoisses. √~Béni soit le Seigneur, car il a magnifié sa miséricorde envers moi dans la cité fortifiée.
\switchcolumn*

Allelúia, allelúia. √~Ibid., 25. Viríliter ágite, et confortétur cor vestrum, omnes qui sperátis in Dómino. Allelúia.
\switchcolumn
Alléluia, alléluia. √~Agissez virilement, afin que votre cœur se fortifie, vous tous qui espérez dans le Seigneur. Alléluia.
\switchcolumn*

Sequéntia sancti Evangélii secúndum Matthǽum.
\switchcolumn
Lecture du saint évangile selon saint Matthieu.
\switchcolumn*

Mt 25, 1-13
\switchcolumn

\switchcolumn*

In illo témpore : Dixit Jesus discípulis  suis parábolam hanc : Símile erit regnum cælórum decem virgínibus : quæ, accipiéntes lámpades suas, exiérunt óbviam sponso et sponsæ. Quinque autem ex eis erant fátuæ, et quinque prudéntes : sed quinque fátuæ, accéptis lampádibus, non sumpsérunt óleum secum : prudéntes vero accepérunt óleum in vasis suis cum lampádibus. Moram autem faciénte sponso, dormitavérunt omnes, et domiérunt. Média autem nocte clamor factus est : Ecce, sponsus venit, exíte óbviam ei. Tunc surrexérunt omnes vírgines illæ, et ornavérunt lámpades suas. Fátuæ autem sapiéntibus dixérunt : Date nobis de óleo vestro : quia lámpades nostræ exstinguúntur. Respondérunt prudéntes, dicéntes : Ne forte non suffíciat nobis, et vobis, ite pótius ad vendéntes, et émite vobis. Dum autem irent émere, venit sponsus : et quæ parátæ erant, intravérunt cum eo ad núptias, et clausa est jánua. Novíssime vero véniunt et réliquæ vírgines, dicéntes : Dómine, Dómine, áperi nobis. At ille respóndens, ait : Amen, dico vobis, néscio vos. Vigiláte ítaque, quia néscitis diem neque horam.
\switchcolumn
En ce temps-là, Jésus dit à ses disciples  cette parabole : Le royaume des cieux sera semblable à dix vierges qui, ayant pris leurs lampes, s’en allèrent à la rencontre de l’époux et de l’épouse. Or cinq d’entre elles étaient insensées et cinq étaient sages. Les cinq insensées, ayant pris leurs lampes, ne prirent point d’huile avec elles, tandis que les sages prirent de l’huile dans leurs vases avec les lampes. Mais comme l’époux tardait, elles s’assoupirent toutes et s’endormirent. Or au milieu de la nuit, une clameur se fit entendre : Voici que l’époux arrive, sortez à sa rencontre. Alors toutes ces vierges se levèrent et préparèrent leurs lampes. Les insensées dirent aux sages : Donnez-nous de votre huile, car nos lampes s’éteignent. Les sages répondirent : De peur que cela ne suffise pour nous et pour vous, allez plutôt chez les marchands et achetez-vous en. Mais tandis qu’elles allaient en acheter, l’époux arriva. Celles qui étaient prêtes entrèrent avec lui dans la salle des noces, et la porte fut fermée. À la fin arrivent les autre vierges, disant : Seigneur, Seigneur, ouvre-nous. Mais lui leur répondit : Amen, je vous le dis, je ne vous connais pas. Veillez donc, car vous ne savez ni le jour ni l’heure.
\switchcolumn*

Ant. ad offertorium.\hfill Ps 100, 6
\switchcolumn
Antienne d’offertoire.
\switchcolumn*

Oculi mei ad fidéles terræ, ut sédeant mecum : ámbulans in via immaculáta, hic mihi ministrábat.
\switchcolumn
Mes yeux sont vers les fidèles de la terre, afin qu’ils trônent avec moi. Celui qui marche par un chemin sans tache, c’est celui-là qui me sert.
\switchcolumn*

Secreta
\switchcolumn
Secrète
\switchcolumn*

Sanctífica, quǽsumus, Dómine Deus,  hæc múnera, quæ in solemnitáte beátæ Gertrúdis Vírginis offérimus : ut per ea vita nostra inter advérsa et próspera salúbriter dirigátur. Per Dóminum.
\switchcolumn
Sanctifiez, nous vous en prions, Seigneur  Dieu, ces dons que nous offrons en la solennité de la bienheureuse Gertrude, afin que par eux notre vie reste dans la voie du salut parmi les prospérités et les adversités. Par Notre-Seigneur.
\switchcolumn*

Ant. ad Comm.\hfill Ps 23, 4
\switchcolumn
Antienne de communion.
\switchcolumn*

Innocens mánibus et mundo corde, qui non accépit in vano ánimam suam.
\switchcolumn
Mains innocentes et cœur pur, il n’a pas reçu son âme en vain.
\switchcolumn*

Postcommunio
\switchcolumn
Postcommunion
\switchcolumn*

Sacris munéribus reféctos illo nos  igne, quǽsumus Dómine, Spíritus Sanctus inflámmet : quem Dóminus noster Jesus Christus misit in terram, et in corde beátæ Gertrúdis vóluit veheménter accéndi : Qui tecum vivit.
\switchcolumn
Nous avons été réconfortés, Seigneur,  par les saints mystères. Que le Saint Esprit, nous vous en prions, nous enflamme de ce feu que Notre-Seigneur Jésus-Christ jeta sur la terre, et dont il voulut qu’il s’enflamme avec force dans le cœur de la bienheureuse Gertrude. Lui qui vit et règne.
\switchcolumn*

\end{paracol}

\end{document}











