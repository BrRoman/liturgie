FERIA SECUNDA

AD SEXTAM



Hymne

Rector potens verax Deus,

Qui témperas rerum vices,

Splendóre mane ínstruis,

Et ígnibus merídiem :



Exstíngue flammas lítium,

Aufer calórem nóxium,

Confer salútem córporum,

Verámque pacem córdium.



Præsta, Páter piíssime,

Patríque compar Unice,

Cum Spíritu Paráclito

Regnans per omne sǽculum.

Amen.


A. Aspice in me, et miserére mei Dómine.


Psalmus 118, 17[t](1)

Mirabília testimónia tua : * ídeo scrutáta est ea ánima mea.

Declarátio sermónum tuórum illúminat  : * et intelléctum dat párvulis.

Os meum apérui, et attráxi spíritum : * quia mandáta tua desiderábam.

Aspice in me, et miserére mei, * secúndum judícium diligéntium nomen tuum.

Gressus meos dírige secúndum elóquium tuum : * et non dominétur mei omnis injustítia.

Rédime me a calúmniis hóminum : * ut custódiam mandáta tua.

Fáciem tuam illúmina super servum tuum : * et doce me justificatiónes tuas.

Exitus aquárum deduxérunt óculi mei : * quia non custodiérunt legem tuam.



Psalmus 118, 18[t](2)

Justus es, Dómine : * et rectum judícium tuum.

Mandásti justítiam testimónia tua : * et veritátem tuam nimis.

Tabéscere me fecit zelus meus : * quia oblíti sunt verba tua inimíci mei.

Ignítum elóquium tuum veheménter : * et servus tuus diléxit illud.

Adolescéntulus sum ego et contémptus : * justificatiónes tuas non sum oblítus.

Justítia tua, justítia in ætérnum : * et lex tua véritas.

Tribulátio, et angústia invenérunt me : * mandáta tua meditátio mea est.

Æquitas testimónia tua in ætérnum : * intelléctum da mihi, et vivam.



Psalmus 118, 19[t](3)

Clamávi in toto corde meo, exáudi me, Dómine : * justificatiónes tuas requíram.

Clamávi ad te, salvum me fac : * ut custódiam mandáta tua.

Prævéni in maturitáte, et clamávi : * quia in verba tua supersperávi.

Prævenérunt óculi mei ad te dilúculo : * ut meditárer elóquia tua.

Vocem meam audi secúndum misericórdiam tuam, Dómine : * et secúndum judícium tuum vivífica me.

Appropinquavérunt persequéntes me iniquitáti : * a lege autem tua longe facti sunt.

Prope es tu, Dómine : * et omnes viæ tuæ véritas.

Inítio cognóvi de testimóniis tuis : * quia in ætérnum fundásti ea.


Capitule

Alter altérius ónera portáte, * et sic adimplébitis legem Christi.


V. Dóminus regit me, et nihil mihi déerit.

R. In loco páscuæ ibi me collocávit.



FERIA SECUNDA

AD NONAM 


Hymne

Rerum, Deus, tenax vigor,

Immótus in te pérmanens,

Lucis diúrnæ témpora

Succéssibus detérminans:


Largíre clarum véspere,

Quo vita nusquam décidat,

Sed prǽmium mortis sacræ

Perénnis instet glória.


Præsta, Pater piíssime,

Patríque compar Únice,

Cum Spíritu Paráclito

Regnans per omne sǽculum.

Amen.


A. Fiat manus tua Dómine, ut salvum me fácias, quia mandáta tua concupívi.


Psalmus 118, 20[t](1)

Vide humilitátem meam, et éripe me: * quia legem tuam non sum oblítus.

Júdica judícium meum, et rédime me: * propter elóquium tuum vivífica me.

Longe a peccatóribus salus: * quia justificatiónes tuas non exquisiérunt.

Misericórdiæ tuæ multæ, Dómine: * secúndum judícium tuum vivífica me.

Multi qui persequúntur me, et tríbulant me: * a testimóniis tuis non declinávi.

Vidi prævaricántes, et tabescébam: * quia elóquia tua non custodiérunt.

Vide quóniam mandáta tua diléxi, Dómine: * in misericórdia tua vivífica me.

Psalmus 118, 21[t](2)

Princípium verbórum tuórum, véritas: * in ætérnum ómnia judícia justítiæ tuæ.

Príncipes persecúti sunt me gratis: * et a verbis tuis formidávit cor meum.

Lætábor ego super elóquia tua: * sicut qui invénit spólia multa.

Iniquitátem ódio hábui, et abominátus sum: * legem autem tuam diléxi.

Sépties in die laudem dixi tibi, * super judícia justítiæ tuæ.

Pax multa diligéntibus legem tuam: * et non est illis scándalum.

Exspectábam salutáre tuum, Dómine: * et mandáta tua diléxi.

Custodívit ánima mea testimónia tua: * et diléxit ea veheménter.

Servávi mandáta tua, et testimónia tua: * quia omnes viæ meæ in conspéctu tuo.

Psalmus 118, 22[t](3)

Appropínquet deprecátio mea in conspéctu tuo, Dómine: * juxta elóquium tuum da mihi intelléctum.

Intret postulátio mea in conspéctu tuo: * secúndum elóquium tuum éripe me.

Eructábunt lábia mea hymnum, * cum docúeris me justificatiónes tuas.

Pronuntiábit lingua mea elóquium tuum: * quia ómnia mandáta tua æquitas.

Fiat manus tua ut salvet me: * quóniam mandáta tua elégi.

Concupívi salutáre tuum, Dómine: * et lex tua meditátio mea est.

Vivet ánima mea, et laudábit te: * et judícia tua adjuvábunt me.

Errávi, sicut ovis, quæ périit: * quǽre servum tuum, quia mandáta tua non sum oblítus.


Capitule

Empti enim estis prétio magno~: * glorificáte et portáte Deum in córpore vestro.


V. Ab occúltis meis munda me, Dómine.

R. Et ab aliénis parce servo tuo.









PER HEBDOMADAM

AD SEXTAM



Hymne

Rector potens verax Deus,

Qui témperas rerum vices,

Splendóre mane ínstruis,

Et ígnibus merídiem:


Exstíngue flammas lítium,

Aufer calórem nóxium,

Confer salútem córporum,

Verámque pacem córdium.


Præsta, Páter piíssime,

Patríque compar Únice,

Cum Spíritu Paráclito

Regnans per omne sǽculum.

Amen.


A. Qui hábitas in cælis, miserére nobis.


Psaume 122

Ad te levávi óculos meos, * qui hábitas in cælis.

Ecce, sicut óculi servórum * in mánibus dominórum suórum,

Sicut óculi ancíllæ in mánibus dóminæ suæ: * ita óculi nostri ad Dóminum, Deum nostrum, donec misereátur nostri.

Miserére nostri, Dómine, miserére nostri: * quia multum repléti sumus despectióne:

Quia multum repléta est ánima nostra: * oppróbrium abundántibus, et despéctio supérbis.


Psaume 123

Nisi quia Dóminus erat in nobis, dicat nunc Israël: * nisi quia Dóminus erat in nobis,

Cum exsúrgerent hómines in nos, * forte vivos deglutíssent nos:

Cum irascerétur furor eórum in nos, * fórsitan aqua absorbuísset nos.

Torréntem pertransívit ánima nostra: * fórsitan pertransísset ánima nostra aquam intolerábilem.

Benedíctus Dóminus * qui non dedit nos in captiónem déntibus eórum.

Anima nostra sicut passer erépta est * de láqueo venántium:

Láqueus contrítus est, * et nos liberáti sumus.

Adjutórium nostrum in nómine Dómini, * qui fecit cælum et terram.


Psaume 124

Qui confídunt in Dómino, sicut mons Sion: * non commovébitur in ætérnum, qui hábitat in Jerúsalem.

Montes in circúitu ejus: * et Dóminus in circúitu pópuli sui, ex hoc nunc et usque in sǽculum.

Quia non relínquet Dóminus virgam peccatórum super sortem justórum: * ut non exténdant justi ad iniquitátem manus suas.

Bénefac, Dómine, bonis, * et rectis corde.

Declinántes autem in obligatiónes addúcet Dóminus cum operántibus iniquitátem: * pax super Israël.


Capitule

Alter altérius ónera portáte * et sic adimplébitis legem Christi.


V. Dóminus regit me, et nihil mihi déerit.

R. In loco páscuæ ibi me collocávit.





Per hebdomadam ad Nonam


Hymne

Rerum, Deus, tenax vigor,

Immótus in te pérmanens,

Lucis diúrnæ témpora

Succéssibus detérminans:


Largíre clarum véspere,

Quo vita nusquam décidat,

Sed prǽmium mortis sacræ

Perénnis instet glória.


Præsta, Pater piíssime,

Patríque compar Únice,

Cum Spíritu Paráclito

Regnans per omne sǽculum.

Amen.


A. Beáti omnes, qui timent Dóminum.


Psaume 125

In converténdo Dóminus captivitátem Sion: * facti sumus sicut consoláti:

Tunc replétum est gáudio os nostrum: * et lingua nostra exsultatióne.

Tunc dicent inter gentes: * Magnificávit Dóminus fácere cum eis.

Magnificávit Dóminus fácere nobíscum: * facti sumus lætántes.

Convérte, Dómine, captivitátem nostram, * sicut torrens in Austro.

Qui séminant in lácrimis, * in exsultatióne metent.

Eúntes ibant et flebant, * mitténtes sémina sua.

Veniéntes autem vénient cum exsultatióne, * portántes manípulos suos.


Psaume 126

Nisi Dóminus ædificáverit domum, * in vanum laboravérunt qui ædíficant eam.

Nisi Dóminus custodíerit civitátem, * frustra vígilat qui custódit eam.

Vanum est vobis ante lucem súrgere: * súrgite postquam sedéritis, qui manducátis panem dolóris.

Cum déderit diléctis suis somnum: * ecce heréditas Dómini fílii: merces, fructus ventris.

Sicut sagíttæ in manu poténtis: * ita fílii excussórum.

Beátus vir, qui implévit desidérium suum ex ipsis: * non confundétur cum loquétur inimícis suis in porta.


Psaume 127

Beáti, omnes, qui timent Dóminum, * qui ámbulant in viis ejus.

Labóres mánuum tuárum quia manducábis: * beátus es, et bene tibi erit.

Uxor tua sicut vitis abúndans, * in latéribus domus tuæ.

Fílii tui sicut novéllæ olivárum, * in circúitu mensæ tuæ.

Ecce, sic benedicétur homo, * qui timet Dóminum.

Benedícat tibi Dóminus ex Sion: * et vídeas bona Jerúsalem ómnibus diébus vitæ tuæ.

Et vídeas fílios filiórum tuórum, * pacem super Israel.


Capitule

Empti enim estis prétio magno~: glorificáte et portáte Deum in córpore vestro.


V. Ab occúltis meis munda me, Dómine.

R. Et ab aliénis parce servo tuo.




Oratio.

A cunctis nos, quǽsumus Dómine, mentis et córporis defénde perículis~; + et intercedénte beáta et gloriósa semper Vírgine Dei Genitríce María, cum beáto Joseph, beátis Apóstolis Petro et Paulo, atque beáto Patre nostro Benedícto et ómnibus Sanctis, salútem nobis tríbue benígnus et pacem, * ut destrúctis adversitátibus et erróribus univérsis, Ecclésia tua secúra tibi sérviat libertáte. Per eúmdem.






