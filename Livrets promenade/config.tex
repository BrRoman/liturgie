\documentclass{report}

% Géométrie de la page :
\usepackage[twoside]{geometry}
\geometry{paperwidth=10.5cm, paperheight=14.85cm, inner=0.7cm, outer=1cm, tmargin=0.9cm, bmargin=1cm, includehead}

% Modification temporaire des marges :
%\usepackage{scrextend}

% Choix d'une police principale :
\usepackage{luatextra}
\setmainfont[Ligatures=Rare]{Arno Pro}

% Langues latin et français (Césures, accents etc.) :
\usepackage[latin]{babel}
%\frenchbsetup{ThinColonSpace=true}

% Accès aux polices Opentype :
\usepackage{fontspec}

% Définition de polices :
\newfontfamily{\GregPlantin}{GregPlantin}
\newfontfamily{\Titre}[Style=Alternate]{Brioso Pro SemiboldDisp}
\newfontfamily{\FlavGaramond}{FlavGaramond}
\newfontfamily{\DropCapFont}{Plantin Std}

% Pour pouvoir insérer des expressions (et non pas seulement des valeurs) dans des commandes telles que \setlength, \addtolength, etc. :
\usepackage{calc}

% Mise en forme des titres de sections :
\usepackage[explicit]{titlesec}
\titleformat{\section}{\filcenter}{}{0cm}
{
\fontsize{14}{16}\selectfont
\parbox{.9\columnwidth}{\centering\textbf{{\textsc{#1}}}}}
\titlespacing{\section}{0cm}{*4}{*4}
%\titlespacing{\section}{0cm}{1.7cm}{1.7cm}

% Entêtes et pieds de pages :
\usepackage{fancyhdr}
\pagestyle{fancy}
\fancyhf{}
\cfoot{\thepage}
\chead{\textsc{\rightmark}}
\renewcommand{\sectionmark}[1]{\markright{#1}}
\renewcommand{\headrulewidth}{0.5pt}
\renewcommand{\footrulewidth}{0pt}
\setlength{\parindent}{0cm}
\setlength{\headsep}{0.5cm} % Distance entre le header et le corps du texte.
% Redefine plain text:
\fancypagestyle{plain}{%
\fancyhf{} % clear all header and footer fields
\setlength{\headheight}{0cm}
\renewcommand{\headrulewidth}{0pt}
\renewcommand{\footrulewidth}{0pt}}

% Symboles :
\usepackage{pifont}

% Lettrines :
\usepackage{lettrine}

% Textes en parallèle :
\usepackage{parallel}

% Pour augmenter l'approche des caractères :
\usepackage{soul}

% Tableaux (litanies des saints) :
%\usepackage{longtable}
%\setlength{\tabcolsep}{0cm}

% Alignement des cellules de tableaux :
%\usepackage{makecell}

% Multicolonne :
\usepackage{multicol}
\setlength{\columnseprule}{0.5pt}
\multicoltolerance = 5000


%%%%%%%%%%%%%%%%%%%%%%%%%%%%%%%%%%%
%%%%%%%%%%%% Gregorio %%%%%%%%%%%%%
%%%%%%%%%%%%%%%%%%%%%%%%%%%%%%%%%%%

\usepackage[autocompile]{gregoriotex}
\grechangedim{commentaryraise}{.4cm}{scalable}
\grechangestyle{modeline}{\fontsize{11}{11}\selectfont\scshape}
\grechangestyle{initial}{\fontsize{36}{36}\color{rougeliturgique}\DropCapFont}
\grechangedim{afterinitialshift}{2.2mm}{fixed}
\grechangedim{beforeinitialshift}{3mm}{fixed}
\newbox\scorebox
\gresetheadercapture{commentary}{grecommentary}{string}
\let\grevanillacommentary\grecommentary
\def\grecommentary#1{\grevanillacommentary{#1\kern 0.3mm}}
\grechangestaffsize{16}% Taille des portées
\gresetbarspacing{new}
\grechangedim{bar@maior@standalone@notext}{0.3 cm}{scalable}
\grechangedim{spacebeforeeolcustos}{0.3 cm}{scalable}

% Couleur rouge :
\definecolor{rougeliturgique}{cmyk}{0.15,1,1,0}

% Style pour les références des partitions :
\grechangestyle{commentary}{\color{rougeliturgique}\itshape\fontsize{9}{8}\selectfont}

% Verset :
\renewcommand{\Vbar}{\textbf{\color{rougeliturgique}\GregPlantin\symbol{8730}}}
\catcode`\↑=\active % Alt-Maj-b
\def↑{\Vbar}

% Répons :
\renewcommand{\Rbar}{\textbf{\color{rougeliturgique}\GregPlantin\symbol{164}}}
\catcode`\®=\active % Alt-Maj-c
\def®{\Rbar}

% Croix haute :
\renewcommand{\GreDagger}{\textrm{\color{rougeliturgique}\FlavGaramond \symbol{8224}}}
\catcode`\Þ=\active % Alt-Maj-t
\defÞ{\GreDagger}

% Croix de Malte :
\catcode`\±=\active % Alt-Maj-+
\def±{{\fontspec{Menlo} \color{rougeliturgique}\symbol{10016}}}

% Étoile dans les partitions :
\def\GreStar{{\color{rougeliturgique}\tiny\raisebox{1.5ex}{\ding{72}}}\relax}

% Étoile des Kyrie et Gloria étoilés :
\catcode`\„=\active % Alt-Maj-s
\def„{\raisebox{0.9ex}{\tiny{~\color{rougeliturgique}\ding{107}}}}

% Antienne :
\catcode`\¥=\active % Alt-Maj-g
\def¥{{\color{rougeliturgique}{\fontspec{FlavGaramond} \symbol{8721}}}}

% Caractères composés (æ, œ, y) + accent aigu :
\makeatletter
\def\accentaigucaractere{\makebox[0pt][c]{´}}
\newcommand\accentaigu[1]{\setlength{\@tempdima}{\widthof{#1}}\hbox{#1\kern-0.5\@tempdima\accentaigucaractere\kern0.5\@tempdima}}
\makeatother
\catcode`\Ð=\active% Alt-Maj-h
\defÐ{\accentaigu{æ}}
\catcode`\Ï=\active% Alt-Maj-k
\defÏ{\accentaigu{œ}}
\catcode`\Ÿ=\active% Alt-Maj-y
\defŸ{\accentaigu{y}}

% Divers symboles liturgiques :
\catcode`\™=\active% Alt-Maj-8 % Temps pascal.
\def™{\textit{\color{rougeliturgique}T.P.}}
\catcode`\Ø=\active% Alt-Maj-$ % Psaume.
\defØ{\textit{\color{rougeliturgique}Ps.}}
\catcode`\→=\active% Alt-Maj-n % N. (in "Pontifici nostro N.").
\def→{{\color{rougeliturgique}\itshape\small{N.}}}

% Pour pouvoir aligner un texte sur les notes d'une partition (Fili Redémptor mundi etc.) :
% (cf. site Gregorio/Tips and Tricks/Creating a Litany)
\usepackage[savepos]{zref}
\makeatletter 
\newcounter{score}
\newcounter{tabstop}[score]
\newcommand{\grealign}{%
	\@bsphack%
	\ifgre@boxing\else%
		\kern\gre@dimen@begindifference%
		\stepcounter{tabstop}%
		\expandafter\zsavepos{stop-\thescore-\thetabstop}%
		\kern-\gre@dimen@begindifference%
	\fi%
	\@esphack%
}
\newcommand{\setstops}{%
  \gdef\nstabbing@stops{%
    \hspace*{-\oddsidemargin}\hspace{-1in}%
    \hspace*{\zposx{stop-\thescore-1} sp}\=%
  }%
  \count@=\@ne
  \loop\ifnum\count@<\value{tabstop}%
  \begingroup\edef\x{\endgroup
    \noexpand\g@addto@macro\noexpand\nstabbing@stops{%
      \noexpand\hspace{-\noexpand\zposx{stop-\thescore-\the\count@} sp}%
      \noexpand\hspace{\noexpand\zposx{stop-\thescore-\the\numexpr\count@+1} sp}\noexpand\=%
    }%
  }\x
  \advance\count@\@ne
  \repeat
  \nstabbing@stops\kill
}
\makeatother
\newenvironment{nstabbing}{
  \setlength{\topsep}{0pt}%
  \setlength{\partopsep}{0pt}%
  \tabbing%
  \setstops
}
{\endtabbing\stepcounter{score}}

\gresetlastline{justified}

%%%%%%%%%%%%%%%%%%%%%%%%%%%%%%%%%%%%%%%%
%%%%%%%%%%%%% Fin Gregorio %%%%%%%%%%%%%
%%%%%%%%%%%%%%%%%%%%%%%%%%%%%%%%%%%%%%%%


%%%%%%%%%%%%%%%%%%%%%%%%%%%%%%%
% Commandes et environnements :
%%%%%%%%%%%%%%%%%%%%%%%%%%%%%%%

% Espace fine :
\DeclareRobustCommand{\mynobreakthinspace}{%
\leavevmode\nobreak\hspace{0.08em}}
\def~{\mynobreakthinspace{}}

% Lettrines rouges [Lettrine, reste du mot, coeff. de dilatation au-dessus de la 1ère ligne, coeff. d'élévation au-dessus de la 1ère ligne] :
\newcommand\DropCapRed[4]{\vspace{-\baselineskip}\lettrine[realheight, nindent=0.5em, slope=-.5em, loversize=#3, lraise=#4]{{\color{rougeliturgique}#1}}{#2}}

% Lettrines noires [Lettrine, reste du mot, coeff. de dilatation au-dessus de la 1ère ligne, coeff. d'élévation au-dessus de la 1ère ligne] :
\newcommand\DropCapBlack[4]{\vspace{-\baselineskip}\lettrine[realheight, nindent=0.5em, slope=-.5em, loversize=#3, lraise=#4]{#1}{#2}}

% Étoile avec espace insécable avant et après :
\newcommand\GreStarNbsp{\enspace\GreStar\enspace}

% Environnement Boîte (espace avant, contenu) :
\newenvironment{ParBox}[2]{
  \setlength{\parindent}{0cm}
  \begin{center}
  \parbox[t]{14.85cm}{\vspace{#1} #2}
  \end{center}
  \par
}

% Style de paragraphe TitreA :
\newenvironment{TitreA}[1]{
  \setlength{\parindent}{0cm}
  \setlength{\leftskip}{0cm}
  \setlength{\rightskip}{0cm}
  \setlength{\parskip}{-0.3cm}
  \fontsize{32}{32}\selectfont
  \begin{center}
  \Titre\textsc{#1}
  \end{center}
}

% Style de paragraphe TitreB :
\newenvironment{TitreB}[1]{
  \setlength{\parindent}{0cm}
  \setlength{\leftskip}{0cm}
  \setlength{\rightskip}{0cm}
  \setlength{\parskip}{-0.3cm}
  \fontsize{16}{18}\selectfont
  \begin{center}
    \textbf{\textsc{#1}}
  \end{center}
}

% Style de paragraphe TitreC :
\newenvironment{TitreC}[1]{
  \setlength{\parindent}{0cm}
  \setlength{\leftskip}{0cm}
  \setlength{\rightskip}{0cm}
  \setlength{\parskip}{0cm}
  \fontsize{11}{13}\selectfont
  \begin{center}
  {\color{rougeliturgique}\textsc{#1}}
  \end{center}
  \vspace{0.3cm}
}

% Normal :
\newcommand\Normal{
  \setlength{\parindent}{0.3cm}
  \setlength{\leftskip}{0cm}
  \setlength{\rightskip}{0cm}
  \setlength{\parskip}{0cm}
  \selectlanguage{latin}
  \fontsize{11.5}{12}\selectfont
}

% Style de paragraphe Hymne :
\newenvironment{Hymne}{
  \setlength{\parindent}{-0.5cm}
  \setlength{\leftskip}{0.5cm}
  \setlength{\rightskip}{0cm}
  \setlength{\parskip}{0cm}
  \selectlanguage{latin}
  \fontsize{11.5}{12}\selectfont
}

% Style de paragraphe Antienne :
\newenvironment{Antienne}{
  \setlength{\parindent}{0cm}
  \setlength{\leftskip}{0cm}
  \setlength{\rightskip}{0cm}
  \setlength{\parskip}{0cm}
  \selectlanguage{latin}
  \fontsize{11.5}{12}\selectfont
  \vspace{0.5cm}
}

% Style de paragraphe Rubrique :
%\newenvironment{Rubrique}[1]{
%  \selectlanguage{french}
%  \fontsize{11.5}{12}\selectfont
%  {\color{rougeliturgique}\textit{#1}}
%  \Normal
%}

% Capitule :
\newenvironment{Capitule}[1]{
  \setlength{\parindent}{0cm}
  \setlength{\leftskip}{0cm}
  \setlength{\rightskip}{0cm}
  \setlength{\parskip}{0cm}
  \fontsize{11}{12}\selectfont
  \vspace{0.5cm}
  {\color{rougeliturgique}\textsc{Capitulum}\hfill \small{#1}}
  \vspace{0.5cm}
}

% Latin/francais :
%\newenvironment{LF}[2]{
%  \begin{Parallel}[]{5cm}{5cm}
%  \selectlanguage{latin}
%  \ParallelLText{\fontsize{12}{13}\selectfont#1}
%  \selectlanguage{french}
%  \ParallelRText{\fontsize{11}{12}\selectfont#2}
%  \end{Parallel}
%  \Normal
%}

% Ligne de séparation :
\newcommand\Ligne{
  \begin{center}
  \greseparator{2}{10}
  \end{center}
}

% Hyphenations :
\hyphenation{al-le-lu-ia}
\hyphenation{exau-cez}
\hyphenation{jus-ti-fi-ca-ti-ó-nes}

